\documentclass[oneside,english]{amsart}
%%\documentclass[oneside,english]{scrreprt}
\usepackage[T1]{fontenc}
%%\usepackage[latin9]{inputenc} %ermöglicht die direkte Eingabe von Sonderzeichen
\usepackage[utf8]{inputenc} %ermöglicht die direkte Eingabe von Sonderzeichen
\usepackage{textcomp}
\usepackage{amsthm}
\usepackage{amstext}
\usepackage{graphicx}
\usepackage{here}
\usepackage{ellipsis} % kümmert sich um den Leerraum rund um die Auslassungspunkte
\usepackage[english]{babel} % Anpassung der Überschriften (chapter, chapitre, Kapitel …) und die richtige Silbentrennung

\makeatletter

\providecommand{\tabularnewline}{\\}

\newcommand{\ImageFormat}{.png}

\numberwithin{equation}{section}
\numberwithin{figure}{section}
\newenvironment{lyxlist}[1]
{\begin{list}{}
{\settowidth{\labelwidth}{#1}
 \setlength{\leftmargin}{\labelwidth}
 \addtolength{\leftmargin}{\labelsep}
 \renewcommand{\makelabel}[1]{##1\hfil}}}
{\end{list}}

\def\ScaleIfNeeded{%
	\ifdim\Gin@nat@height> 0.9\textheight
		0.9\textheight
	\else
		\Gin@nat@height
	\fi
}

\def\ConstForImageWidth{9cm}

%\newcommand{\loadImage}[1]{\IfFileExists{#1}{\begin{figure}[H] \begin{center} \includegraphics[height=\ScaleIfNeeded,width=\ConstForImageWidth, keepaspectratio]{#1} \end{center} \end{figure}}{\begin{figure}[!htb] \centering \includegraphics[height=\ScaleIfNeeded,width=\ConstForImageWidth, keepaspectratio]{noValues.pdf} \end{figure}}}
%\newcommand{\loadTex}[1]{\IfFileExists{#1}{\input{#1}}{\begin{figure}[H] \begin{center} \includegraphics[height=\ScaleIfNeeded,width=\ConstForImageWidth, keepaspectratio]{noValues.pdf} \end{center} \end{figure}}}
\newcommand{\loadImage}[1]{\IfFileExists{#1}{\begin{OwnLoadImage} \includegraphics[height=\ScaleIfNeeded,width=\ConstForImageWidth, keepaspectratio]{#1} \end{OwnLoadImage}}{\begin{OwnLoadImage} \includegraphics[height=\ScaleIfNeeded,width=\ConstForImageWidth, keepaspectratio]{noValues.pdf} \end{OwnLoadImage}}}
\newcommand{\loadTex}[1]{\IfFileExists{#1}{\input{#1}}{\begin{OwnLoadImage} \includegraphics[height=\ScaleIfNeeded,width=\ConstForImageWidth, keepaspectratio]{noValues.pdf} \end{OwnLoadImage}}}

\newenvironment{OwnLoadImage}
  {\par\raggedbottom\null\vfill\noindent\minipage{\textwidth}\centering}
  {\endminipage\par\vfill\vfill}


\renewcommand{\topfraction}{.85} % maximaler Abstand von Seitenoberseite, bis zu welchem Gleitumgebungen noch plaziert werden dürfen
\renewcommand{\bottomfraction}{.7} % maximaler Anteil welchen Gleitumgebungen am unteren Seitenrand einnehmen dürfen
\renewcommand{\textfraction}{.15} % Anteil einer Seite mit Gleitumgebungen, welcher mindestens von Text belegt sein muss -> ansonsten kein Text auf der Seite
\renewcommand{\floatpagefraction}{.66} % minimaler Seitenanteil welcher besetzt sein muss, bevor eine neue Seite für Gleitumgebungen angelegt wird
\setcounter{topnumber}{3} % maximale Anzahl Gleitobjekte am oberen Seitenrand
\setcounter{bottomnumber}{3} % maximale Anzahl Gleitobjekte am unteren Seitenrand
\setcounter{totalnumber}{6} % maximale Anzahl Gleitobjekte pro Seite


\makeatother


\begin{document}

\title{Analysis of Experiment --experimentname--}


\author{Research Group Image Analysis\\ \\ IPK-Gatersleben\\ \\ C. Klukas \& A. Entzian}


\date{\today}


\address{Leibniz Institute of Plant Genetics and Crop Plant Research, Group
Image Analysis, Corrensstr. 3, 06466 Gatersleben, Germany}


\email{klukas@ipk-gatersleben.de, entzian@ipk-gatersleben.de}


\urladdr{http://ba-13.ipk-gatersleben.de}
\begin{abstract}
This document is automatically created based on the automated image analysis performed 
inside the integrated analysis platform IAP. Currently the pipelines and data analysis tools are in beta status, 
which means that we are working on makeing the analysis procedures, statistical operations, diagram output and report
generation more stable and suitable to the needs of any user of the automated imaging infrastructure at the IPK.

If you have any hints, comments or suggestions for improvements, please don't hesitate to contact any member
of the group image analysis.
\end{abstract}

\keywords{image analysis, phenotyping, maize }

\maketitle

\clearpage
\tableofcontents

\clearpage

\section{Experiment info}
\subsection{Experiment properties}

\begin{lyxlist}{00.00.0000}
\item [{%
\begin{tabular}{|c|c|c|}
\hline 
Experiment & --experimentname--\tabularnewline
\hline 
\hline 
Start & --StartExp--\tabularnewline
\hline 
End & --EndExp--\tabularnewline
\hline 
Numeric values & --NumExp-- \tabularnewline
\hline 
Images & --ImagesExp-- \tabularnewline
\hline 
Storage & --StorageExp-- \tabularnewline
\hline 
\end{tabular}}]~
\end{lyxlist}

\subsection{Remarks}

\begin{lyxlist}{00.00.0000}
\item [{%
\begin{tabular}{|c|c|}
\hline 
Notes\tabularnewline
\hline 
\hline 
--RemarkExp--\tabularnewline
\hline 
\end{tabular}}]~
\end{lyxlist}

\subsection{Experiment factors}

\begin{lyxlist}{00.00.0000}
\item [{%
\begin{tabular}{|c|c|c|c|c|c|}
\hline 
ID & Genotype & Variety & Treatment & Sequence & xyz\tabularnewline
\hline 
\hline 
--ID-- & --Genotype-- & --Variety-- & --Treatment-- & --Sequence-- & --xyz--\tabularnewline
\hline 
\end{tabular}}]~
\end{lyxlist}

%%\newpage
\clearpage
\section{\noindent Weight and water consumption}

\subsection{Weights (before and after watering)}
\begin{itemize}
\item Weight before and after watering.
\item Colnum name: Weight A (g), Weight B (g)
\item Unit: g
\end{itemize}
\loadImage{Weight_A__g_nboxplot.pdf}
\loadImage{Weight_B__g_nboxplot.pdf}



\subsection{Watering amounts / consumption}
\begin{itemize}
\item Consumption of water based of weight measurments.
\item Colnum name: Watering (weight-diff)
\item Unit: g
\end{itemize}
\loadImage{Water__weight_diff_nboxplot.pdf}


\clearpage
\section{\noindent Biomass}
\begin{itemize}
\item Digital Biomass based on the side.area and top.area
\item Unit: pixel\textthreesuperior{}
\end{itemize}

\subsection{IAP based formula - Visible Light}
\begin{itemize}
\item Equation: $Biomass_{IAP}=\sqrt{side.area_{average}^{2}*top.area_{average}}$
\item Column name: volume.iap
\end{itemize}
\loadImage{volume_iap__px3_boxplot.pdf}
\loadImage{volume_iap__px3_nboxplot.pdf}

\subsection{Volume (LT-Formel)}
\begin{itemize}
\item Equation: $Biomass_{LemnaTec}=\sqrt{side.area_{0^{\circ}}*side.area_{90^{\circ}}*top.area}$
\item Column name: volume.lt
\end{itemize}
\loadImage{volume_lt__px3_boxplot.pdf}
\loadImage{volume_lt__px3_nboxplot.pdf}


\subsection{KeyGene formula}
\begin{itemize}
\item Equation: $Biomass_{KeyGene}=side.area_{0^{\circ}}+side.area_{90^{\circ}}+\log(\frac{top.area}{3})$
\item Column name: digital.biomass.keygene.norm
\end{itemize}

%\loadImage{digital_biomass_keygene_normnboxplot.pdf}


\clearpage
\section{\noindent Water use efficiency}
\begin{itemize}
\item Ratio of plant growth and water use.
\end{itemize}

\subsection{Volume based}
\begin{itemize}
\item Growth per day divided by water usage per day.
\item Colnum name: volume.iap.wue
\item Unit: .
\end{itemize}

\loadImage{volume_iap_wuenboxplot.pdf}



\clearpage
\section{General growth related plant properties}

\subsection{Plant height}
\begin{itemize}
\item Plant height (px)
\item Column name: side.height
\item Unit: px
\end{itemize}
\loadImage{side_height__mm_boxplot.pdf}
\loadImage{side_height__mm_nboxplot.pdf}


\subsection{Normalized plant height}
\begin{itemize}
\item Plant height (mm) (normalized to distance of left and right marker)
\item Column name: side.height.norm
\item Unit: mm
\end{itemize}
\loadImage{side_height_norm__mm_boxplot.pdf}
\loadImage{side_height_norm__mm_nboxplot.pdf}


\subsection{Plant width}
\begin{itemize}
\item Plant width (px) 
\item Column name: side.width
\item Unit: px
\end{itemize}
\loadImage{side_width__mm_boxplot.pdf}
\loadImage{side_width__mm_nboxplot.pdf}


\subsection{Normalized plant width}
\begin{itemize}
\item Plant width (mm) (normalized to distance of left and right marker) 
\item Column name: side.width.norm
\item Unit: mm
\end{itemize}
\loadImage{side_width_norm__mm_boxplot.pdf}
\loadImage{side_width_norm__mm_nboxplot.pdf}


\subsection{Projected side area}
\begin{itemize}
\item Number of foreground pixels
\item Colum name: side.area
\item Unit: px
\end{itemize}
\loadImage{side_area__px_boxplot.pdf}
\loadImage{side_area__px_nboxplot.pdf}


\subsection{Normalized projected side area}
\begin{itemize}
\item Number of foreground pixels from side camera (normalized to distance of left and right marker)
\item Colum name: side.area.norm
\item Unit: mm\texttwosuperior{}
\end{itemize}
\loadImage{side_area_norm__mm2_boxplot.pdf}
\loadImage{side_area_norm__mm2_nboxplot.pdf}


\subsection{Projected top area}
\begin{itemize}
\item Number of foreground pixels
\item Colum name: top.area
\item Unit: px
\end{itemize}
\loadImage{top_area__px_boxplot.pdf}
\loadImage{top_area__px_nboxplot.pdf}


\subsection{Normalized projected top area}
\begin{itemize}
\item Number of foreground pixels from top camera (normalized to distance of left and right marker)
\item Colum name: top.area.norm
\item Unit: mm\texttwosuperior{}
\end{itemize}

\loadImage{top_area_norm__mm2_nboxplot.pdf}


\clearpage
\section{\noindent Relative values (changes in percent, per day)}
\begin{itemize}
\item Presented values in relative dependence.
\item Unit: \%
\end{itemize}

\subsection{Relative projected side area growth}
\begin{itemize}
\item Relative growth of number of foreground pixels from side camera
\item Colnum name: side.area.relative
\end{itemize}

\loadImage{side_area_relativenboxplot.pdf}



\subsection{Relative plant height growth}
\begin{itemize}
\item Relative growth of plant height in percent (normalized to distance of left and right marker)
\item Colnum name: side.height.norm.relative
\end{itemize}

\loadImage{side_height_norm_relativenboxplot.pdf}


\subsection{Relative plant width growth}
\begin{itemize}
\item Relative plant width growth in percent (normalized to distance of left and right marker) 
\item Colnum name: side.width.norm.relative
\end{itemize}

\loadImage{side_width_norm_relativenboxplot.pdf}


\subsection{Relative projected top area growth}
\begin{itemize}
\item Relative growth of number of foreground pixels from top camera
\item Colnum name: top.area.relative
\end{itemize}

\loadImage{top_area_relativenboxplot.pdf}


\subsection{Relative projected side area growth}
\begin{itemize}
\item Relative growth of number of foreground pixels from side camera
\item Colnum name: side.area.relative
\end{itemize}

\loadImage{side_area_relativenboxplot.pdf}


\subsection{Relative growth of volume (IAP based formular)}
\begin{itemize}
\item Digital biomass growth in percent (per day) based on the side.area and top.area observed from the visible light camera.
\item Colnum name: volume.iap.relative
\end{itemize}

\loadImage{volume_iap_relativenboxplot.pdf}

\clearpage
\section{VIS intensity analysis}

\subsection{VIS normalized intensity}
\begin{itemize}
\item Average activity of the normalized VIS images.
\item The darker the higher acitivity
\item Column name: xxx
\item Unit: %
\end{itemize}
\loadTex{side.vis.normalized.histogram.ratio.bin.1.0_25;side.vis.normalized.his}

\subsection{VIS ratio intensity}
\begin{itemize}
\item Average activity of the normalized VIS images.
\item The darker the higher acitivity
\item Column name: xxx
\item Unit: %
\end{itemize}
\loadTex{side.vis.hue.histogram.ratio.bin.1.0_25;side.vis.hue.histogram.ratio.n}
\loadTex{top.vis.hue.histogram.ratio.bin.1.0_25;top.vis.hue.histogram.ratio.bin}


\clearpage
\section{Fluorescence intensity analysis}

\subsection{Red intensity}
\begin{itemize}
\item Average activity of the red reflection.
\item The darker the higher acitivity
\item Column name: xxx
\item Unit: %
\end{itemize}
\loadTex{side_fluo_histogram_bin_1_0_25_side_fluo_histogram_bin_2_25_51_side_flstackedOverallImage}
\loadTex{top_fluo_histogram_bin_1_0_25_top_fluo_histogram_bin_2_25_51_top_fluo_stackedOverallImage}


\subsection{Red normalized intensity}
\begin{itemize}
\item Average activity of the normalized red reflection.
\item The darker the higher acitivity
\item Column name: xxx
\item Unit: %
\end{itemize}
\loadTex{side_fluo_normalized_histogram_bin_1_0_25_side_fluo_normalized_histogrstackedOverallImage}

\subsection{Fluo ratio intensity}
\begin{itemize}
\item xxx
\item The darker the higher acitivity
\item Column name: xxx
\item Unit: %
\end{itemize}
\loadTex{side_fluo_histogram_ratio_bin_1_0_25_side_fluo_histogram_ratio_bin_2_2stackedOverallImage}
\loadTex{top_fluo_histogram_ratio_bin_1_0_25_top_fluo_histogram_ratio_bin_2_25_stackedOverallImage}

\subsection{Fluo normalized ratio intensity}
\begin{itemize}
\item xxx
\item The darker the higher acitivity
\item Column name: xxx
\item Unit: %
\end{itemize}
\loadTex{side_fluo_normalized_histogram_ratio_bin_1_0_25_side_fluo_normalized_hstackedOverallImage}

\subsection{Near-infrared intensity}
\begin{itemize}
\item Average intensity of near infrared (NIR)
\item Represents the water content of the plant
\item Column name: xxx
\item Unit: %
\end{itemize}
\loadTex{side_nir_histogram_bin_1_0_25_side_nir_histogram_bin_2_25_51_side_nhisstackedOverallImage}
\loadTex{top_nir_histogram_bin_1_0_25_top_nir_histogram_bin_2_25_51_top_nir_hisstackedOverallImage}


\subsection{Near-infrared normalized intensity}
\begin{itemize}
\item Average intensity of normalized near infrared (NIR)
\item Represents the water content of the plant
\item Column name: xxx
\item Unit: %
\end{itemize}
\loadTex{side_nir_normalized_histogram_bin_1_0_25_side_nir_normalized_histogramstackedOverallImage}

\clearpage
\section{Plant structures}

\subsection{Number of leafs}
\begin{itemize}
\item Number of leafs-tips
\item Colum name: side.leaf.count.median
\item Unit: leafs
\item Hint: leaf-tips are shown as red retangles in the image
\end{itemize}

\loadImage{side_leaf_count_median__leafs_nboxplot.pdf}


\subsection{Tassel (for maize plants)}
\begin{itemize}
\item Male flower count
\item Colum name: side.bloom.count
\item Unit: tassel
\item Hint: the number of blue retangles in the image
\end{itemize}

\loadImage{side_bloom_count__tassel_nboxplot.pdf}


\subsection{Leaf lengths}
\begin{itemize}
\item Length of all leafs plus stem
\item Column name: side.leaf.length.sum.norm.max
\item Unit: mm
\item Hint: the yellow line in the image
\end{itemize}

\loadImage{side_leaf_length_sum_norm_max__mm_nboxplot.pdf}


\clearpage
\section{\noindent Wetness}
\begin{itemize}
\item Vales based on the NIR images
\item Unit: .
\end{itemize}

\subsection{Average wetness of side image}
\begin{itemize}
\item Average wetness of the plants from NIR side camera
\item Column name: side.nir.wetness.av
\item Unit: .
\end{itemize}

\loadImage{side_nir_wetness_av.pdf}


\subsection{Average wetness of top image}
\begin{itemize}
\item Average wetness of the plants from NIR top camera
\item Column name: top.nir.wetness.av
\item Unit: .
\end{itemize}

\loadImage{top_nir_wetness_avnboxplot.pdf}


\subsection{Weighted loss through drought stress - side image}
\begin{itemize}
\item Number of foreground pixels from NIR side camera minus the weighted value of the plant
\item weightOfPlant = fully wet: 1 unit, fully dry: 1/7 unit
\item Column name: side.nir.wetness.plant\_weight\_drought\_loss
\item Unit: .
\end{itemize}

\loadImage{side_nir_wetness_plant_weight_drought_lossnboxplot.pdf}



\subsection{Weighted loss through drought stress - top image}
\begin{itemize}
\item Number of foreground pixels from NIR top camera minus the weighted value of the plant
\item weightOfPlant = fully wet: 1 unit, fully dry: 1/7 unit
\item Column name: top.nir.wetness.plant\_weight\_drought\_loss
\item Unit: .
\end{itemize}

\loadImage{top_nir_wetness_plant_weight_drought_lossnboxplot.pdf}


\clearpage
\section{Appendix}

\loadTex{appendixImage}

\end{document}