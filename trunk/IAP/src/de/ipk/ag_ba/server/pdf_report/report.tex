%%%%%%%%%%%%% Klassen-Optionen
\documentclass[%
  paper=a4, % Stellt auf A4-Papier ein
  pagesize, % Diese Option reicht die Papiergröße an
            % alle Ausgabeformate weiter
  DIV=calc, % Errechnet einen guten Satzspiegel
%  BCOR=1cm, % Weil bei mir immer 1cm in der Bindung
            % der Klemmmappe verschwindet
  headings=small,% Für etwas kleinere Überschriften
  english,   % englische Rechtschreibung d.\,h. (Silbentrennung)
  oneside
]{scrartcl}  % Eine Klasse für einseitige Texte mit Kapiteln  
            %
% \documentclass[oneside,english]{article}
                  
%%\documentclass[oneside,english]{scrreprt}
\usepackage[T1]{fontenc}
\usepackage[utf8]{inputenc} %ermöglicht die direkte Eingabe von Sonderzeichen
%\usepackage[utf8]{inputenc} %ermöglicht die direkte Eingabe von Sonderzeichen
\usepackage{fixltx2e}
\usepackage[english]{babel} % Anpassung der Überschriften (chapter, chapitre,
% Kapitel …) und die richtige Silbentrennung
\usepackage{hyperref}
\usepackage[left=2.5cm,right=2.5cm,top=2.5cm,bottom=2.5cm]{geometry}
\usepackage{datetime}

%%%%%%%%%%%%% Typografisch empfehlenswerte Pakete
\usepackage{% 
  %ellipsis, % Korrigiert den Weißraum um Auslassungspunkte
  ragged2e, % Ermöglicht Flattersatz mit Silbentrennung
 %marginnote,% Für bessere Randnotizen mit \marginnote statt
            % \marginline 
}
% \usepackage[tracking=true]{microtype}% 
%             % Microtype ist einfach super, aber lesen Sie
%             % unbedingt die Anleitung um das Folgende zu
%             % verstehen.
% \DeclareMicrotypeSet*[tracking]{my}% 
%   { font = */*/*/sc/* }% 
% \SetTracking{ encoding = *, shape = sc }{ 45 }% Hier wird festgelegt,
%             % dass alle Passagen in Kapitälchen automatisch leicht
%             % gesperrt werden. Das Paket soul, das ich früher empfohlen
%             % habe ist damit für diese Zwecke nicht mehr nötig.
%             %

\usepackage{%
  lmodern, % A) Latin Modern Fonts sind die Nachfolger von Computer
            % Modern, den LaTeX-Standardfonts
%  hfoldsty % B) Diese Schrift stellt alle Ziffern, außer
            % im Mathemodus, auf Minuskel- oder Mediäval-Ziffern um.
            % Wenn Ihre pdfs unscharf aussehen installieren Sie bitte
            % die cm-super-Fonts (Type1-Fonts).
% charter   % C) Diese Zeile lädt die Charter als Schriftart
}
% \usepackage[osf,sc]{mathpazo}% D) So erreichen Sie Palatino als
            % Schrift mit Minuskel-Ziffern und echten Kapitälchen
            %

%\usepackage{here}
%\usepackage{scrpage2} 
%\usepackage{tocloft}
\usepackage{graphicx}
%\usepackage{amsmath} 
\usepackage{ifthen}
\usepackage{longtable}

% —– Start: Angaben zur Formatierung von Überschriften —– %
\usepackage{titletoc}
% \titlecontents{Abschnitt}[Links]{Gesamtformatierung}{Vor dem %
% Label}{Nach dem Label}{Abstandsfüller & Seitenzahl}[Rechts]
%4.5 and 3.0
%7.5 and 3.2
%10.6 and 3.7
\titlecontents{subsection}[4.5em]{}{\contentslabel{3.0em}}{}{\titlerule*[0.6pc]{.}\contentspage}
\titlecontents{subsubsection}[7.5em]{}{\contentslabel{3.2em}}{}{\titlerule*[0.6pc]{.}\contentspage}
\titlecontents{paragraph}[10.6em]{}{\contentslabel{3.7em}}{}{\titlerule*[0.6pc]{.}\contentspage}
\titlecontents{subparagraph}[13.6em]{}{\contentslabel{4.1em}}{}{\titlerule*[0.6pc]{.}\contentspage}
% —– Ende: Angaben zur Formatierung von Überschriften —– %

\setcounter{tocdepth}{5}
\setcounter{secnumdepth}{5}

\makeatletter

\providecommand{\tabularnewline}{\\}

% \numberwithin{equation}{section}
% \numberwithin{figure}{section}

\def\ScaleIfNeeded{%
	\ifdim\Gin@nat@height> 0.9\textheight
		0.9\textheight
	\else
		\Gin@nat@height
	\fi 
}


\def\ConstForImageWidth{\textwidth}

%\newcommand{\loadImage}[1]{\IfFileExists{#1}{\begin{figure}[H] \begin{center} \includegraphics[height=\ScaleIfNeeded,width=\ConstForImageWidth, keepaspectratio]{#1} \end{center} \end{figure}}{\begin{figure}[!htb] \centering \includegraphics[height=\ScaleIfNeeded,width=\ConstForImageWidth, keepaspectratio]{noValues.pdf} \end{figure}}}
%\newcommand{\loadTex}[1]{\IfFileExists{#1}{\input{#1}}{\begin{figure}[H] \begin{center} \includegraphics[height=\ScaleIfNeeded,width=\ConstForImageWidth, keepaspectratio]{noValues.pdf} \end{center} \end{figure}}}
\newcommand{\loadImage}[1]{\IfFileExists{#1}{\begin{OwnLoadImage} \includegraphics[height=\ScaleIfNeeded,width=\ConstForImageWidth, keepaspectratio]{#1} \end{OwnLoadImage}}{\begin{OwnLoadImage} \includegraphics[height=\ScaleIfNeeded,width=\ConstForImageWidth, keepaspectratio]{noValues.pdf} \end{OwnLoadImage}}}
\newcommand{\loadTex}[1]{\IfFileExists{#1}{\input{#1}}{\begin{OwnLoadImage} \includegraphics[height=\ScaleIfNeeded,width=\ConstForImageWidth, keepaspectratio]{noValues.pdf} \end{OwnLoadImage}}}

\newenvironment{OwnLoadImage}
  {\par\noindent\minipage{\textwidth}\centering}
  {\endminipage}

\newboolean{isFile1}
\newboolean{isFile2}
\newboolean{isFile3}

\newcommand{\resetBoolean}{
	\setboolean{isFile1}{false}
	\setboolean{isFile2}{false}
	\setboolean{isFile3}{false}
	}

\newcommand{\checkFileNoReset}[3]{
	\IfFileExists{#1}{\setboolean{isFile1}{true}}{}
	\IfFileExists{#2}{\setboolean{isFile2}{true}}{}
	\IfFileExists{#3}{\setboolean{isFile3}{true}}{}
	}

\newcommand{\checkFile}[3]{
	\resetBoolean
	\checkFileNoReset{#1}{#2}{#3}
	%\setboolean{isFile1}{false}
	%\setboolean{isFile2}{false}
	%\IfFileExists{#1}{\setboolean{isFile1}{true}}{}
	%\IfFileExists{#2}{\setboolean{isFile2}{true}}{}
	}

\newboolean{isClearPage}
\setboolean{isClearPage}{true}

\newboolean{isClearPageSub}
\setboolean{isClearPageSub}{true}

\newcommand{\resetClear}{
	\setboolean{isClearPage}{true}
}

\newcommand{\resetClearSub}{
	\setboolean{isClearPageSub}{true}
}


\newcommand{\ownClearPage}{

	\ifthenelse{\boolean{isClearPage}}{
		\setboolean{isClearPage}{false}
	}{
		\clearpage
	}
}

\newcommand{\ownClearPageSub}{

	\ifthenelse{\boolean{isClearPageSub}}{
		\setboolean{isClearPageSub}{false}
	}{
		\clearpage
	}
}

% \usepackage{array}
% \newcolumntype{C}[1]{>{\begin{center}}p{#1}<{\end{center}}}

% \newboolean{forceClearPage}
% \setboolean{forceClearPage}{true}
% 
% \newboolean{noForceClearPage}
% \setboolean{noForceClearPage}{false}

% \newcommand{\ownClearPage}[1]{
% 	
% 	\ifthenelse{#1}{
% 		\setboolean{isClearPage}{true}
% 		\clearpage	
% 	}{
% 		\ifthenelse{\boolean{isClearPage}}{
% 			\setboolean{isClearPage}{false}
% 		}{
% 			\setboolean{isClearPage}{true}
% 			\clearpage
% 		}
% 	}
% }
	

% \newenvironment{OwnLoadImage}
%   {\par\raggedbottom\null\vfill\noindent\minipage{\textwidth}\centering}
%   {\endminipage\par\vfill\vfill}
  
\renewcommand{\topfraction}{.85} % maximaler Abstand von Seitenoberseite, bis zu welchem Gleitumgebungen noch plaziert werden dürfen
\renewcommand{\bottomfraction}{.7} % maximaler Anteil welchen Gleitumgebungen am unteren Seitenrand einnehmen dürfen
\renewcommand{\textfraction}{.15} % Anteil einer Seite mit Gleitumgebungen, welcher mindestens von Text belegt sein muss -> ansonsten kein Text auf der Seite
\renewcommand{\floatpagefraction}{.66} % minimaler Seitenanteil welcher besetzt sein muss, bevor eine neue Seite für Gleitumgebungen angelegt wird
\setcounter{topnumber}{3} % maximale Anzahl Gleitobjekte am oberen Seitenrand
\setcounter{bottomnumber}{3} % maximale Anzahl Gleitobjekte am unteren Seitenrand
\setcounter{totalnumber}{6} % maximale Anzahl Gleitobjekte pro Seite

 
\makeatother


\begin{document}


\title{--experimentname-- \tabularnewline \vspace{10 mm} \large{Experiment coordinator: --coordinator--}}

\author{Automated image analysis report by\tabularnewline Research
Group Image Analysis - IPK-Gatersleben\tabularnewline \tabularnewline C. Klukas
\& A. Entzian}

\date{\today ~(\currenttime )}

\maketitle
\thispagestyle{empty}  
\begin{abstract}
This document is automatically created based on the automated image analysis performed 
inside the integrated analysis platform IAP. Currently the pipelines and data analysis tools are in beta status, 
which means that we are working on makeing the analysis procedures, statistical operations, diagram output and report
generation more stable and suitable to the needs of any user of the automated imaging infrastructure at the IPK.

If you have any hints, comments or suggestions for improvements, please don't hesitate to contact any member
of the group image analysis.
\end{abstract}
\vfill
\small{\textit{E-mail address:} klukas@ipk-gatersleben.de, entzian@ipk-gatersleben.de}
\newline 
\small{\textit{Key words and phrases:} image analysis, phenotyping, \textit{Arabidopsis}, barley, maize }
 
\addtocounter{page}{-1}
\clearpage
\tableofcontents

\clearpage
\pagestyle{headings}
\section{Experiment info} 
\subsection{Experiment properties}
\begin{center}
	\begin{tabular}{|p{3cm}|p{13cm}|}
	\hline
	{\textbf{Experiment}} & --experimentnameShort--\tabularnewline
	\hline
	\hline
	{\textbf{Start}} & --StartExp--\tabularnewline
	\hline
	{\textbf{End}} & --EndExp--\tabularnewline
	\hline
	{\textbf{Numeric values}} & --NumExp-- \tabularnewline
	\hline
	{\textbf{Images}} & --ImagesExp-- \tabularnewline
	\hline
	{\textbf{Storage}} & --StorageExp-- \tabularnewline
	\hline
	{\textbf{Sequence}} & --SequenceEXP-- \tabularnewline
	\hline
	{\textbf{Defined Outliers}} & --Outliers-- \tabularnewline
	\hline
	\hline 
	\end{tabular}
\end{center}

\subsection{Remarks}
\begin{center}
	\begin{tabular}{|p{16.5cm}|}
	\hline 
	{\textbf{Notes}}\tabularnewline
	\hline 
	\hline 
	--RemarkExp--\tabularnewline
	\hline
	\hline  
	\end{tabular}
\end{center}


\resetBoolean
\checkFileNoReset{conditionsFirstFilter.tex}{}{}
\checkFileNoReset{conditionsSecondFilter.tex}{}{}
\ifthenelse{\boolean{isFile1}}{
	\subsection{Grouping of selected experiment factors}
	\checkFile{conditionsFirstFilter.tex}{conditionsSecondFilter.tex}{}
	\ifthenelse{\boolean{isFile1} \and \boolean{isFile2}}{
		\subsubsection{Primary filter}
		\begin{center}
		\loadTex{conditionsFirstFilter}
		\end{center}
		
		\subsubsection{Secondary filter}
		\begin{center}
		\loadTex{conditionsSecondFilter}
		\end{center}
	}{
		\ifthenelse{\boolean{isFile1}}{
			\begin{center}
			\loadTex{conditionsFirstFilter}
			\end{center}
		}{
			\begin{center}
			\loadTex{conditionsSecondFilter}
			\end{center}	
		}
	}
}

\subsection{Full list of experiment factors}
\begin{center}
	\begin{longtable}{|p{0.7cm}|p{2.2cm}|p{2.2cm}|p{2.2cm}|p{2.2cm}|p{2.2cm}|}
	%|p{2.2cm}
	
	%This is the header for the first page of the table...
	
	\hline
	{\textbf{ID}} & 
	{\textbf{Species}} & 
	{\textbf{Genotype}} & 
	{\textbf{Variety}} &
	{\textbf{Treatment}} & 
% 	{\textbf{Sequence}} & 
	{\textbf{Growth conditions}}
	\tabularnewline
	\hline
	\hline
	\endfirsthead
	
	 %This is the header for the remaining page(s) of the table...
	 
	\hline
	{\textbf{ID}} & 
	{\textbf{Species}} & 
	{\textbf{Genotype}} & 
	{\textbf{Variety}} &
	{\textbf{Treatment}} & 
% 	{\textbf{Sequence}} & 
	{\textbf{Growth conditions}}
	\tabularnewline
	\hline
	\hline
	\endhead
	
	%This is the footer for all pages except the last page of the table...
	
	\multicolumn{6}{l}{{Continued on Next Page\ldots}} 
	\tabularnewline
	\endfoot
	
	%This is the footer for the last page of the table...
	
	\hline \hline
	\endlastfoot

	--factorlist--

	\end{longtable}
\end{center}

% \begin{center}
% 	\begin{tabular}{|p{1cm}|p{2.2cm}|p{2.2cm}|p{2.2cm}|p{2.2cm}|p{2.2cm}|p{2.2cm}|}
% 	\hline
% 	ID & Species & Genotype & Variety & Treatment & Sequence & Treatment
% 	\tabularnewline
% 	\hline
% 	\hline
% 	-##-factorlist--
% 
% 	\end{tabular}
% \end{center}


\resetBoolean
\checkFileNoReset{Weight_A__g_nboxplotOverallImage.tex}{}{}
\checkFileNoReset{Weight_B__g_nboxplotOverallImage.tex}{}{}
\checkFileNoReset{Water__sum_of_day_nboxplotOverallImage.tex}{}{}
%\checkFileNoReset{Water__weight_diff_nboxplotOverallImage.tex}{}{}
\ifthenelse{\boolean{isFile1}}{

	\clearpage
	\section{\noindent Weights and water consumption}
		
	\checkFile{Weight_A__g_nboxplotOverallImage.tex}{Weight_B__g_nboxplotOverallImage.tex}{}
	\ifthenelse{\boolean{isFile1} \or \boolean{isFile2}}{
		\subsection{Weights (before and after watering)}
		\begin{itemize}
		\item Weight before and after watering.
		\item Colnum name: Weight A (g), Weight B (g)
		\item Unit: g
		\end{itemize}
		\ifthenelse{\boolean{isFile1}}{
			\loadTex{Weight_A__g_nboxplotOverallImage}
		}{}
		\ifthenelse{\boolean{isFile2}}{
			\loadTex{Weight_B__g_nboxplotOverallImage}
		}{}
	}{}
	
	\checkFile{Water__sum_of_day_nboxplotOverallImage.tex}{}{}
	\ifthenelse{\boolean{isFile1}}{
		\subsection{Daily watering amounts}
		\begin{itemize}
		\item The sum of the watering amount of the day.
		\item Colnum name: Water (sum of day)
		\item Unit: g
		\end{itemize}
		\loadTex{Water__sum_of_day_nboxplotOverallImage}
	}{}
%   \checkFile{Water__weight_diff_nboxplotOverallImage.tex}{}{}
% 	\ifthenelse{\boolean{isFile1}}{
% 		\subsection{Watering amounts / consumption}
% 		\begin{itemize}
% 		\item Consumption of water based on weight differences.
% 		\item Colnum name: Watering (weight-diff)
% 		\item Unit: g
% 		\end{itemize}
% 		\loadTex{Water__weight_diff_nboxplotOverallImage}
% 	}{}
}{}

\resetBoolean
\checkFileNoReset{clusters.pdf}{}{}
\ifthenelse{\boolean{isFile1}}{
	\clearpage
	\section{\noindent Clustering of data}
		
	\subsection{Clustering based on watering data}
	\begin{itemize}
	\item The clustering is based on the similarity of the watering data for the different conditions over time.
	\item For the bootstrapping, the option n is 100.
	\end{itemize}
	\loadImage{clusters.pdf}
}{}


\resetBoolean
\checkFileNoReset{side_height__px__side_width__px__side_area__px__top_area__px__side_fluspiderOverallImage.tex}{}{}
\checkFileNoReset{side_height_norm__mm__side_width_norm__mm__side_area_norm__mm2__top_aspiderOverallImage.tex}{}{}
\checkFileNoReset{side_height__px__side_width__px__side_area__px__top_area__px__side_flulineRangeOverallImage.tex}{}{}
\checkFileNoReset{side_height_norm__mm__side_width_norm__mm__side_area_norm__mm2__top_alineRangeOverallImage.tex}{}{}
\checkFileNoReset{mark1_y__percent_nboxplotOverallImage.tex}{}{}
\checkFileNoReset{mark2_y__percent_nboxplotOverallImage.tex}{}{}
\checkFileNoReset{mark3_y__percent_nboxplotOverallImage.tex}{}{}

\checkFileNoReset{side_area_norm__mm2_violinOverallImage.tex}{}{}
\checkFileNoReset{top_area_norm__mm2_violinOverallImage.tex}{}{}
\checkFileNoReset{side_fluo_intensity_average__relative_violinOverallImage.tex}{}{}
\checkFileNoReset{side_nir_intensity_average__relative_violinOverallImage.tex}{}{}
\checkFileNoReset{side_width_norm__mm_violinOverallImage.tex}{}{}
\checkFileNoReset{side_height_norm__mm_violinOverallImage.tex}{}{}
\checkFileNoReset{side_vis_hue_averageviolinOverallImage.tex}{}{}
\checkFileNoReset{top_vis_hue_averageviolinOverallImage.tex}{}{}
\ifthenelse{\boolean{isFile1}}{

	\resetClear
	\clearpage
	\section{Properties - Overview}
	
	\resetBoolean
	\checkFileNoReset{side_height__px__side_width__px__side_area__px__top_area__px__side_fluspiderOverallImage.tex}{}{}
	\checkFileNoReset{}{side_height_norm__mm__side_width_norm__mm__side_area_norm__mm2__top_aspiderOverallImage.tex}{}
	\ifthenelse{\boolean{isFile2}}{
		\ownClearPage
		\subsection{Growth parameters, fluorescence, near-infrared and visible light color (zoom corrected)}
		\loadTex{side_height_norm__mm__side_width_norm__mm__side_area_norm__mm2__top_aspiderOverallImage}
	}{
		\ifthenelse{\boolean{isFile1}}{
			\ownClearPage
			\subsection{Growth parameters, fluorescence, near-infrared and visible light color}
			\loadTex{side_height__px__side_width__px__side_area__px__top_area__px__side_fluspiderOverallImage}
			}{}
	}
	
	\resetBoolean
	\checkFileNoReset{side_height__px__side_width__px__side_area__px__top_area__px__side_flulineRangeOverallImage.tex}{}{}
	\checkFileNoReset{}{side_height_norm__mm__side_width_norm__mm__side_area_norm__mm2__top_alineRangeOverallImage.tex}{}
	\ifthenelse{\boolean{isFile2}}{
		\ownClearPage
		\subsection{Difference within special descriptors (zoom corrected)}
		\loadTex{side_height_norm__mm__side_width_norm__mm__side_area_norm__mm2__top_alineRangeOverallImage}
	}{
		\ifthenelse{\boolean{isFile1}}{
			\ownClearPage
			\subsection{Difference within special descriptors}
			\loadTex{side_height__px__side_width__px__side_area__px__top_area__px__side_flulineRangeOverallImage}
			}{}
	}
	
	\resetBoolean
	\checkFileNoReset{mark1_y__percent_nboxplotOverallImage.tex}{}{}
	\checkFileNoReset{mark2_y__percent_nboxplotOverallImage.tex}{}{}
	\checkFileNoReset{mark3_y__percent_nboxplotOverallImage.tex}{}{}
	\ifthenelse{\boolean{isFile1}}{
		\ownClearPage
		\subsection{Zoom changes}
		
		\checkFile{mark1_y__percent_nboxplotOverallImage.tex}{mark2_y__percent_nboxplotOverallImage.tex}{mark3_y__percent_nboxplotOverallImage.tex}
		\ifthenelse{\boolean{isFile1}}{
			\loadTex{mark1_y__percent_nboxplotOverallImage}
		}{}
		
		\ifthenelse{\boolean{isFile2}}{
			\loadTex{mark2_y__percent_nboxplotOverallImage}
		}{}
		
		\ifthenelse{\boolean{isFile3}}{
			\loadTex{mark3_y__percent_nboxplotOverallImage}
		}{}
	}{}
	
	\resetBoolean
	\checkFileNoReset{side_area_norm__mm2_violinOverallImage.tex}{}{}
	\checkFileNoReset{top_area_norm__mm2_violinOverallImage.tex}{}{}
	\checkFileNoReset{side_fluo_intensity_average__relative_violinOverallImage.tex}{}{}
	\checkFileNoReset{side_nir_intensity_average__relative_violinOverallImage.tex}{}{}
	\checkFileNoReset{side_width_norm__mm_violinOverallImage.tex}{}{}
	\checkFileNoReset{side_height_norm__mm_violinOverallImage.tex}{}{}
	\checkFileNoReset{side_vis_hue_averageviolinOverallImage.tex}{}{}
	\checkFileNoReset{top_vis_hue_averageviolinOverallImage.tex}{}{}
	\checkFileNoReset{top_fluo_intensity_average__relative___pix_violinOverallImage.tex}{}{}
	\checkFileNoReset{top_nir_intensity_average__relative___pix_violinOverallImage.tex}{}{}
	\ifthenelse{\boolean{isFile1}}{
		\ownClearPage
		\subsection{Stress ratio}
		
		\checkFile{side_area_norm__mm2_violinOverallImage.tex}{top_area_norm__mm2_violinOverallImage.tex}{side_fluo_intensity_average__relative_violinOverallImage.tex}
		\ifthenelse{\boolean{isFile1}}{
			\loadTex{side_area_norm__mm2_violinOverallImage}
		}{}
		
		\ifthenelse{\boolean{isFile2}}{
			\loadTex{top_area_norm__mm2_violinOverallImage}
		}{}
		
		\ifthenelse{\boolean{isFile3}}{
			\loadTex{side_fluo_intensity_average__relative_violinOverallImage}
		}{}
	
		\checkFile{top_fluo_intensity_average__relative___pix_violinOverallImage.tex}{}{}
		\ifthenelse{\boolean{isFile1}}{
			\loadTex{top_fluo_intensity_average__relative___pix_violinOverallImage}
		}{}
	
		\checkFile{side_nir_intensity_average__relative_violinOverallImage.tex}{top_nir_intensity_average__relative___pix_violinOverallImage.tex}{side_height_norm__mm_violinOverallImage.tex}
		\ifthenelse{\boolean{isFile1}}{
			\loadTex{side_nir_intensity_average__relative_violinOverallImage}
		}{}
		
		\ifthenelse{\boolean{isFile2}}{
			\loadTex{top_nir_intensity_average__relative___pix_violinOverallImage}
		}{}
		
		\ifthenelse{\boolean{isFile3}}{
			\loadTex{side_height_norm__mm_violinOverallImage}
		}{}
	
		\checkFile{side_width_norm__mm_violinOverallImage.tex}{}{}
		\ifthenelse{\boolean{isFile1}}{
			\loadTex{side_width_norm__mm_violinOverallImage}
		}{}
	
		\checkFile{side_vis_hue_averageviolinOverallImage.tex}{top_vis_hue_averageviolinOverallImage.tex}{}
		\ifthenelse{\boolean{isFile1}}{
			\loadTex{side_vis_hue_averageviolinOverallImage}
		}{}
		
		\ifthenelse{\boolean{isFile2}}{
			\loadTex{top_vis_hue_averageviolinOverallImage}
		}{}
	}{}			
}{}

\resetBoolean
\checkFileNoReset{volume_iap__px3_boxplotOverallImage.tex}{}{}
\checkFileNoReset{volume_iap__px3_nboxplotOverallImage.tex}{}{}
\checkFileNoReset{volume_lt__px3_boxplotOverallImage.tex}{}{}
\checkFileNoReset{volume_lt__px3_nboxplotOverallImage.tex}{}{}
\checkFileNoReset{digital_biomass_keygene_normnboxplotOverallImage.tex}{}{}
\checkFileNoReset{volume_iap_wuenboxplotOverallImage.tex}{}{}
\ifthenelse{\boolean{isFile1}}{

	\resetClear
	\clearpage
	\section{\noindent Digital biomass}
	\begin{itemize}
	\item Digital Biomass based on the side.area and top.area
	\item Unit: $pixel^3$
	\end{itemize}
	
	\checkFile{volume_iap__px3_boxplotOverallImage.tex}{volume_iap__px3_nboxplotOverallImage.tex}{}
	\ifthenelse{\boolean{isFile1} \or \boolean{isFile2}}{
		\ownClearPage
		\subsection{IAP formula}
		\begin{itemize}
		\item Equation: $Biomass_{IAP}=\sqrt{side.area_{average}^{2}*top.area}$
		\item Column name: volume.iap
		\end{itemize}
		\ifthenelse{\boolean{isFile1}}{
			\loadTex{volume_iap__px3_boxplotOverallImage}
		}{}
		\ifthenelse{\boolean{isFile2}}{
			\loadTex{volume_iap__px3_nboxplotOverallImage}
		}{}
	}{}
	
	
	\checkFile{volume_lt__px3_boxplotOverallImage.tex}{volume_lt__px3_nboxplotOverallImage.tex}{}
	\ifthenelse{\boolean{isFile1} \or \boolean{isFile2}}{
		\ownClearPage
		\subsection{LemnaTec formula}
		\begin{itemize}
		\item Equation: $Biomass_{LemnaTec}=\sqrt{side.area_{0^{\circ}}*side.area_{90^{\circ}}*top.area}$
		\item Column name: volume.lt
		\end{itemize}
		\ifthenelse{\boolean{isFile1}}{
			\loadTex{volume_lt__px3_boxplotOverallImage}
		}{}
		\ifthenelse{\boolean{isFile2}}{
			\loadTex{volume_lt__px3_nboxplotOverallImage}
		}{}
	}{}
	
	
	\checkFile{digital_biomass_keygene_normnboxplotOverallImage.tex}{}{}
	\ifthenelse{\boolean{isFile1}}{
		\ownClearPage
		\subsection{KeyGene formula}
		\begin{itemize}
		\item Equation: $Biomass_{KeyGene}=side.area_{0^{\circ}}+side.area_{90^{\circ}}+\log(\frac{top.area}{3})$
		\item Column name: digital.biomass.keygene.norm
		\end{itemize}
		\loadTex{digital_biomass_keygene_normnboxplotOverallImage}
	}{}
	
	\resetBoolean	
	\checkFileNoReset{volume_iap_wuenboxplotOverallImage.tex}{}{}
	\checkFileNoReset{side_area_avg_wuenboxplotOverallImage.tex}{}{}
	\ifthenelse{\boolean{isFile1}}{
		\ownClearPage
		\section{\noindent Water use efficiency}
		\begin{itemize}
		\item Ratio of daily plant growth, determined by increasing projected side
		area and/or digital biomass, and water ussaged per day.
		\end{itemize}
		
		\checkFile{volume_iap_wuenboxplotOverallImage.tex}{side_area_avg_wuenboxplotOverallImage.tex}{}
		\ifthenelse{\boolean{isFile1}}{
			\subsection{Based on digital biomass}
			\begin{itemize}
			\item Colnum name: volume.iap.wue
			\item Unit: $pixel^3/g$
			\end{itemize}
			\loadTex{volume_iap_wuenboxplotOverallImage}
		}
		
		\ifthenelse{\boolean{isFile2}}{
			\subsection{Based on projected side area}
			\begin{itemize}
			\item Colnum name: side.area.avg.wue
			\item Unit: $pixel^2/g$
			\end{itemize}
			\loadTex{side_area_avg_wuenboxplotOverallImage}
		}
	}{}
}{}


\resetBoolean
\checkFileNoReset{side_height__px_boxplotOverallImage.tex}{}{}
\checkFileNoReset{side_height__px_nboxplotOverallImage.tex}{}{}
\checkFileNoReset{side_height_norm__mm_boxplotOverallImage.tex}{}{}
\checkFileNoReset{side_height_norm__mm_nboxplotOverallImage.tex}{}{}
\checkFileNoReset{side_width__px_boxplotOverallImage.tex}{}{}
\checkFileNoReset{side_width__px_nboxplotOverallImage.tex}{}{}
\checkFileNoReset{side_width_norm__mm_boxplotOverallImage.tex}{}{}
\checkFileNoReset{side_width_norm__mm_nboxplotOverallImage.tex}{}{}
\checkFileNoReset{side_area__px_boxplotOverallImage.tex}{}{}
\checkFileNoReset{side_area__px_nboxplotOverallImage.tex}{}{}
\checkFileNoReset{side_area_norm__mm2_boxplotOverallImage.tex}{}{}
\checkFileNoReset{side_area_norm__mm2_nboxplotOverallImage.tex}{}{}
\checkFileNoReset{top_area__px_boxplotOverallImage.tex}{}{}
\checkFileNoReset{top_area__px_nboxplotOverallImage.tex}{}{}
\checkFileNoReset{top_area_norm__mm2_nboxplotOverallImage.tex}{}{}
\ifthenelse{\boolean{isFile1}}{

	\resetClear
	\clearpage
	\section{General growth related plant properties}
		
	\resetBoolean
	\checkFileNoReset{side_height__px_boxplotOverallImage.tex}{}{}
	\checkFileNoReset{side_height__px_nboxplotOverallImage.tex}{}{}
	\checkFileNoReset{}{side_height_norm__mm_boxplotOverallImage.tex}{}
	\checkFileNoReset{}{side_height_norm__mm_nboxplotOverallImage.tex}{}
	\ifthenelse{\boolean{isFile2}}{
		
		\ownClearPage
		\subsection{Height (zoom corrected)}
		\begin{itemize}
		\item Plant height (mm) (normalized to distance of left and right marker)
		\item Column name: side.height.norm
		\item Unit: mm
		\end{itemize}
		\checkFile{side_height_norm__mm_boxplotOverallImage.tex}{side_height_norm__mm_nboxplotOverallImage.tex}{}
		\ifthenelse{\boolean{isFile1}}{
			\loadTex{side_height_norm__mm_boxplotOverallImage}
		}{}	
		\ifthenelse{\boolean{isFile2}}{	
			\loadTex{side_height_norm__mm_nboxplotOverallImage}
		}{}
	}{
		\ifthenelse{\boolean{isFile1}}{
			
			\ownClearPage
			\subsection{Height}
			\begin{itemize}
			\item Plant height (px)
			\item Column name: side.height
			\item Unit: px
			\end{itemize}
			\checkFile{side_height__px_boxplotOverallImage.tex}{side_height__px_nboxplotOverallImage.tex}{}
			\ifthenelse{\boolean{isFile1}}{
				\loadTex{side_height__px_boxplotOverallImage}
			}{}
			\ifthenelse{\boolean{isFile2}}{
				\loadTex{side_height__px_nboxplotOverallImage}
			}{}
		}{}
	}	
	
	\resetBoolean
	\checkFileNoReset{side_width__px_boxplotOverallImage.tex}{}{}
	\checkFileNoReset{side_width__px_nboxplotOverallImage.tex}{}{}
	\checkFileNoReset{}{side_width_norm__mm_boxplotOverallImage.tex}{}
	\checkFileNoReset{}{side_width_norm__mm_nboxplotOverallImage.tex}{}
	\ifthenelse{\boolean{isFile2}}{
		
		\ownClearPage
		\subsection{Width (zoom corrected)}
		\begin{itemize}
		\item Plant width (mm) (normalized to distance of left and right marker) 
		\item Column name: side.width.norm
		\item Unit: mm
		\end{itemize}
		\checkFile{side_width_norm__mm_boxplotOverallImage.tex}{side_width_norm__mm_nboxplotOverallImage.tex}{}
		\ifthenelse{\boolean{isFile1}}{
			\loadTex{side_width_norm__mm_boxplotOverallImage}
		}{}
		\ifthenelse{\boolean{isFile2}}{	
			\loadTex{side_width_norm__mm_nboxplotOverallImage}
		}{}
	}{
		\ifthenelse{\boolean{isFile1}}{
	
			\ownClearPage
			\subsection{Width}
			\begin{itemize}
			\item Plant width (px) 
			\item Column name: side.width
			\item Unit: px
			\end{itemize}
			\checkFile{side_width__px_boxplotOverallImage.tex}{side_width__px_nboxplotOverallImage.tex}{}
			\ifthenelse{\boolean{isFile1}}{
				\loadTex{side_width__px_boxplotOverallImage}
			}{}
			\ifthenelse{\boolean{isFile2}}{	
				\loadTex{side_width__px_nboxplotOverallImage}
			}{}
		}{}	
	}
	
	
	\resetBoolean
	\checkFileNoReset{side_area__px_boxplotOverallImage.tex}{}{}
	\checkFileNoReset{side_area__px_nboxplotOverallImage.tex}{}{}
	\checkFileNoReset{}{side_area_norm__mm2_boxplotOverallImage.tex}{}
	\checkFileNoReset{}{side_area_norm__mm2_nboxplotOverallImage.tex}{}
	\ifthenelse{\boolean{isFile2}}{
		
		\ownClearPage
		\subsection{Projected side area (zoom corrected)}
		\begin{itemize}
		\item Number of foreground pixels from side camera (normalized to distance of left and right marker)
		\item Colum name: side.area.norm
		\item Unit: $mm^2$
		\end{itemize}
		\checkFile{side_area_norm__mm2_boxplotOverallImage.tex}{side_area_norm__mm2_nboxplotOverallImage.tex}{}
		\ifthenelse{\boolean{isFile1}}{
			\loadTex{side_area_norm__mm2_boxplotOverallImage}
		}{}
		\ifthenelse{\boolean{isFile2}}{
			\loadTex{side_area_norm__mm2_nboxplotOverallImage}
		}{}
	
	}{	
		\ifthenelse{\boolean{isFile1}}{
			
			\ownClearPage
			\subsection{Projected side area}
			\begin{itemize}
			\item Number of foreground pixels
			\item Colum name: side.area
			\item Unit: px
			\end{itemize}
			\checkFile{side_area__px_boxplotOverallImage.tex}{side_area__px_nboxplotOverallImage.tex}{}
			\ifthenelse{\boolean{isFile1}}{
				\loadTex{side_area__px_boxplotOverallImage}
			}{}
			\ifthenelse{\boolean{isFile2}}{
				\loadTex{side_area__px_nboxplotOverallImage}
			}{}
		}{}	
	}
	
	
	\resetBoolean
	\checkFileNoReset{top_area__px_boxplotOverallImage.tex}{}{}
	\checkFileNoReset{top_area__px_nboxplotOverallImage.tex}{}{}
	\checkFileNoReset{}{top_area_norm__mm2_nboxplotOverallImage.tex}{}
	%\checkFileNoReset{}{top_area_norm__mm2_boxplot.pdf}{}
	\ifthenelse{\boolean{isFile2}}{
		
		\ownClearPage
		\subsection{Projected top area (zoom corrected)}
		\begin{itemize}
		\item Number of foreground pixels from top camera (normalized to distance of left and right marker)
		\item Colum name: top.area.norm
		\item Unit: $mm^2$
		\end{itemize}
		%\checkFile{top_area_norm__mm2_nboxplot.pdf}{}{}
		\loadTex{top_area_norm__mm2_nboxplotOverallImage}
	
	}{
		\ifthenelse{\boolean{isFile1}}{
			
			\ownClearPage
			\subsection{Projected top area}
			\begin{itemize}
			\item Number of foreground pixels
			\item Colum name: top.area
			\item Unit: px
			\end{itemize}
			\checkFile{top_area__px_boxplotOverallImage.tex}{top_area__px_nboxplotOverallImage.tex}{}
			\ifthenelse{\boolean{isFile1}}{
				\loadTex{top_area__px_boxplotOverallImage}
			}{}
			\ifthenelse{\boolean{isFile2}}{
				\loadTex{top_area__px_nboxplotOverallImage}
			}{}
		}{}
	}
	
	
}{}


\resetBoolean
\checkFileNoReset{side_area_relativenboxplotOverallImage.tex}{}{}
\checkFileNoReset{side_height_norm_relativenboxplotOverallImage.tex}{}{}
\checkFileNoReset{side_width_norm_relativenboxplotOverallImage.tex}{}{}
\checkFileNoReset{top_area_relativenboxplotOverallImage.tex}{}{}
\checkFileNoReset{side_area_relativenboxplot.tex}{}{}
\checkFileNoReset{volume_iap_relativenboxplotOverallImage.tex}{}{}
\ifthenelse{\boolean{isFile1}}{

	\resetClear
	\clearpage
	\section{\noindent Relative changes per day}
	\begin{itemize}
	\item Presented values in relative dependence per day.
	\item Unit: $\%/day$
	\end{itemize}
	
	
	\checkFile{side_area_relativenboxplotOverallImage.tex}{}{}
	\ifthenelse{\boolean{isFile1}}{
		\ownClearPage
		\subsection{Projected side area}
		\begin{itemize}
		\item Relative growth of number of foreground pixels from side camera
		\item Colnum name: side.area.relative
		\end{itemize}
		\loadTex{side_area_relativenboxplotOverallImage}
	}{}
	
	
	\checkFile{side_height_norm_relativenboxplotOverallImage.tex}{}{}
	\ifthenelse{\boolean{isFile1}}{
		\ownClearPage
		\subsection{Height}
		\begin{itemize}
		\item Relative growth of plant height in percent (normalized to distance of left and right marker)
		\item Colnum name: side.height.norm.relative
		\end{itemize}
		\loadTex{side_height_norm_relativenboxplotOverallImage}
	}{}
	
	
	\checkFile{side_width_norm_relativenboxplotOverallImage.tex}{}{}
	\ifthenelse{\boolean{isFile1}}{
		\ownClearPage
		\subsection{Width}
		\begin{itemize}
		\item Relative plant width growth in percent (normalized to distance of left and right marker) 
		\item Colnum name: side.width.norm.relative
		\end{itemize}
		\loadTex{side_width_norm_relativenboxplot}
	}{}
	
	\checkFile{top_area_relativenboxplotOverallImage.tex}{}{}
	\ifthenelse{\boolean{isFile1}}{
		\ownClearPage
		\subsection{Projected top area}
		\begin{itemize}
		\item Relative growth of number of foreground pixels from top camera
		\item Colnum name: top.area.relative
		\end{itemize}
		\loadTex{top_area_relativenboxplotOverallImage.tex}
	}{}
	
	
	\checkFile{side_area_relativenboxplotOverallImage.tex}{}{}
	\ifthenelse{\boolean{isFile1}}{
		\ownClearPage
		\subsection{Projected side area}
		\begin{itemize}
		\item Relative growth of number of foreground pixels from side camera
		\item Colnum name: side.area.relative
		\end{itemize}
		\loadTex{side_area_relativenboxplotOverallImage}
	}{}
	
	\checkFile{volume_iap_relativenboxplotOverallImage.tex}{}{}
	\ifthenelse{\boolean{isFile1}}{
		\ownClearPage
		\subsection{Digital biomass (IAP formula)}
		\begin{itemize}
		\item Digital biomass growth in percent (per day) based on the side.area and top.area observed from the visible light camera.
		\item Colnum name: volume.iap.relative
		\end{itemize}
		\loadTex{volume_iap_relativenboxplotOverallImage.tex}
	}{}
}{}

\resetBoolean
\checkFileNoReset{side_vis_hsv_h_averagenboxplotOverallImage.tex}{}{}
\checkFileNoReset{side_vis_hsv_h_histogram_bin_stackedOverallImage.tex}{}{}
\checkFileNoReset{side_vis_hsv_h_kurtosisnboxplotOverallImage.tex}{}{}
\checkFileNoReset{side_vis_hsv_h_skewnessnboxplotOverallImage.tex}{}{}
\checkFileNoReset{side_vis_hsv_h_stddevnboxplotOverallImage.tex}{}{}
\checkFileNoReset{top_vis_hsv_h_averagenboxplotOverallImage.tex}{}{}
\checkFileNoReset{top_vis_hsv_h_histogram_bin_stackedOverallImage.tex}{}{}
\checkFileNoReset{top_vis_hsv_h_kurtosisnboxplotOverallImage.tex}{}{}
\checkFileNoReset{top_vis_hsv_h_skewnessnboxplotOverallImage.tex}{}{}
\checkFileNoReset{top_vis_hsv_h_stddevnboxplotOverallImage.tex}{}{}
\checkFileNoReset{side_vis_hsv_h_histogram_bin_01_0_12nboxplotOverallImage.tex}{}{}
\checkFileNoReset{side_vis_hsv_h_histogram_bin_02_12_25nboxplotOverallImage.tex}{}{}
\checkFileNoReset{side_vis_hsv_h_histogram_bin_03_25_38nboxplotOverallImage.tex}{}{}
\checkFileNoReset{side_vis_hsv_h_histogram_bin_04_38_51nboxplotOverallImage.tex}{}{}
\checkFileNoReset{side_vis_hsv_h_histogram_bin_05_51_63nboxplotOverallImage.tex}{}{}
\checkFileNoReset{side_vis_hsv_h_histogram_bin_06_63_76nboxplotOverallImage.tex}{}{}
\checkFileNoReset{top_vis_hsv_h_histogram_bin_01_0_12nboxplotOverallImage.tex}{}{}
\checkFileNoReset{top_vis_hsv_h_histogram_bin_02_12_25nboxplotOverallImage.tex}{}{}
\checkFileNoReset{top_vis_hsv_h_histogram_bin_03_25_38nboxplotOverallImage.tex}{}{}
\checkFileNoReset{top_vis_hsv_h_histogram_bin_04_38_51nboxplotOverallImage.tex}{}{}
\checkFileNoReset{top_vis_hsv_h_histogram_bin_05_51_63nboxplotOverallImage.tex}{}{}
\checkFileNoReset{top_vis_hsv_h_histogram_bin_06_63_76nboxplotOverallImage.tex}{}{}

\checkFileNoReset{side_vis_hsv_h_normalized_averagenboxplotOverallImage.tex}{}{}
\checkFileNoReset{side_vis_hsv_h_normalized_histogram_bin_stackedOverallImage.tex}{}{}
\checkFileNoReset{side_vis_hsv_h_normalized_kurtosisnboxplotOverallImage.tex}{}{}
\checkFileNoReset{side_vis_hsv_h_normalized_skewnessnboxplotOverallImage.tex}{}{}
\checkFileNoReset{side_vis_hsv_h_normalized_stddevnboxplotOverallImage.tex}{}{}
\checkFileNoReset{top_vis_hsv_h_normalized_averagenboxplotOverallImage.tex}{}{}
\checkFileNoReset{top_vis_hsv_h_normalized_histogram_bin_stackedOverallImage.tex}{}{}
\checkFileNoReset{top_vis_hsv_h_normalized_kurtosisnboxplotOverallImage.tex}{}{}
\checkFileNoReset{top_vis_hsv_h_normalized_skewnessnboxplotOverallImage.tex}{}{}
\checkFileNoReset{top_vis_hsv_h_normalized_stddevnboxplotOverallImage.tex}{}{}
\checkFileNoReset{side_vis_hsv_h_normalized_histogram_bin_01_0_12nboxplotOverallImage.tex}{}{}
\checkFileNoReset{side_vis_hsv_h_normalized_histogram_bin_02_12_25nboxplotOverallImage.tex}{}{}
\checkFileNoReset{side_vis_hsv_h_normalized_histogram_bin_03_25_38nboxplotOverallImage.tex}{}{}
\checkFileNoReset{side_vis_hsv_h_normalized_histogram_bin_04_38_51nboxplotOverallImage.tex}{}{}
\checkFileNoReset{side_vis_hsv_h_normalized_histogram_bin_05_51_63nboxplotOverallImage.tex}{}{}
\checkFileNoReset{side_vis_hsv_h_normalized_histogram_bin_06_63_76nboxplotOverallImage.tex}{}{}
\checkFileNoReset{top_vis_hsv_h_normalized_histogram_bin_01_0_12nboxplotOverallImage.tex}{}{}
\checkFileNoReset{top_vis_hsv_h_normalized_histogram_bin_02_12_25nboxplotOverallImage.tex}{}{}
\checkFileNoReset{top_vis_hsv_h_normalized_histogram_bin_03_25_38nboxplotOverallImage.tex}{}{}
\checkFileNoReset{top_vis_hsv_h_normalized_histogram_bin_04_38_51nboxplotOverallImage.tex}{}{}
\checkFileNoReset{top_vis_hsv_h_normalized_histogram_bin_05_51_63nboxplotOverallImage.tex}{}{}
\checkFileNoReset{top_vis_hsv_h_normalized_histogram_bin_06_63_76nboxplotOverallImage.tex}{}{}

\checkFileNoReset{side_vis_hsv_s_averagenboxplotOverallImage.tex}{}{}
\checkFileNoReset{side_vis_hsv_s_histogram_bin_stackedOverallImage.tex}{}{}
\checkFileNoReset{side_vis_hsv_s_kurtosisnboxplotOverallImage.tex}{}{}
\checkFileNoReset{side_vis_hsv_s_skewnessnboxplotOverallImage.tex}{}{}
\checkFileNoReset{side_vis_hsv_s_stddevnboxplotOverallImage.tex}{}{}
\checkFileNoReset{top_vis_hsv_s_averagenboxplotOverallImage.tex}{}{}
\checkFileNoReset{top_vis_hsv_s_histogram_bin_stackedOverallImage.tex}{}{}
\checkFileNoReset{top_vis_hsv_s_kurtosisnboxplotOverallImage.tex}{}{}
\checkFileNoReset{top_vis_hsv_s_skewnessnboxplotOverallImage.tex}{}{}
\checkFileNoReset{top_vis_hsv_s_stddevnboxplotOverallImage.tex}{}{}
\checkFileNoReset{side_vis_hsv_s_histogram_bin_01_0_12nboxplotOverallImage.tex}{}{}
\checkFileNoReset{side_vis_hsv_s_histogram_bin_02_12_25nboxplotOverallImage.tex}{}{}
\checkFileNoReset{side_vis_hsv_s_histogram_bin_03_25_38nboxplotOverallImage.tex}{}{}
\checkFileNoReset{side_vis_hsv_s_histogram_bin_04_38_51nboxplotOverallImage.tex}{}{}
\checkFileNoReset{side_vis_hsv_s_histogram_bin_05_51_63nboxplotOverallImage.tex}{}{}
\checkFileNoReset{side_vis_hsv_s_histogram_bin_06_63_76nboxplotOverallImage.tex}{}{}
\checkFileNoReset{top_vis_hsv_s_histogram_bin_01_0_12nboxplotOverallImage.tex}{}{}
\checkFileNoReset{top_vis_hsv_s_histogram_bin_02_12_25nboxplotOverallImage.tex}{}{}
\checkFileNoReset{top_vis_hsv_s_histogram_bin_03_25_38nboxplotOverallImage.tex}{}{}
\checkFileNoReset{top_vis_hsv_s_histogram_bin_04_38_51nboxplotOverallImage.tex}{}{}
\checkFileNoReset{top_vis_hsv_s_histogram_bin_05_51_63nboxplotOverallImage.tex}{}{}
\checkFileNoReset{top_vis_hsv_s_histogram_bin_06_63_76nboxplotOverallImage.tex}{}{}

\checkFileNoReset{side_vis_hsv_s_normalized_averagenboxplotOverallImage.tex}{}{}
\checkFileNoReset{side_vis_hsv_s_normalized_histogram_bin_stackedOverallImage.tex}{}{}
\checkFileNoReset{side_vis_hsv_s_normalized_kurtosisnboxplotOverallImage.tex}{}{}
\checkFileNoReset{side_vis_hsv_s_normalized_skewnessnboxplotOverallImage.tex}{}{}
\checkFileNoReset{side_vis_hsv_s_normalized_stddevnboxplotOverallImage.tex}{}{}
\checkFileNoReset{top_vis_hsv_s_normalized_averagenboxplotOverallImage.tex}{}{}
\checkFileNoReset{top_vis_hsv_s_normalized_histogram_bin_stackedOverallImage.tex}{}{}
\checkFileNoReset{top_vis_hsv_s_normalized_kurtosisnboxplotOverallImage.tex}{}{}
\checkFileNoReset{top_vis_hsv_s_normalized_skewnessnboxplotOverallImage.tex}{}{}
\checkFileNoReset{top_vis_hsv_s_normalized_stddevnboxplotOverallImage.tex}{}{}
\checkFileNoReset{side_vis_hsv_s_normalized_histogram_bin_01_0_12nboxplotOverallImage.tex}{}{}
\checkFileNoReset{side_vis_hsv_s_normalized_histogram_bin_02_12_25nboxplotOverallImage.tex}{}{}
\checkFileNoReset{side_vis_hsv_s_normalized_histogram_bin_03_25_38nboxplotOverallImage.tex}{}{}
\checkFileNoReset{side_vis_hsv_s_normalized_histogram_bin_04_38_51nboxplotOverallImage.tex}{}{}
\checkFileNoReset{side_vis_hsv_s_normalized_histogram_bin_05_51_63nboxplotOverallImage.tex}{}{}
\checkFileNoReset{side_vis_hsv_s_normalized_histogram_bin_06_63_76nboxplotOverallImage.tex}{}{}
\checkFileNoReset{top_vis_hsv_s_normalized_histogram_bin_01_0_12nboxplotOverallImage.tex}{}{}
\checkFileNoReset{top_vis_hsv_s_normalized_histogram_bin_02_12_25nboxplotOverallImage.tex}{}{}
\checkFileNoReset{top_vis_hsv_s_normalized_histogram_bin_03_25_38nboxplotOverallImage.tex}{}{}
\checkFileNoReset{top_vis_hsv_s_normalized_histogram_bin_04_38_51nboxplotOverallImage.tex}{}{}
\checkFileNoReset{top_vis_hsv_s_normalized_histogram_bin_05_51_63nboxplotOverallImage.tex}{}{}
\checkFileNoReset{top_vis_hsv_s_normalized_histogram_bin_06_63_76nboxplotOverallImage.tex}{}{}

\checkFileNoReset{side_vis_hsv_v_averagenboxplotOverallImage.tex}{}{}
\checkFileNoReset{side_vis_hsv_v_histogram_bin_stackedOverallImage.tex}{}{}
\checkFileNoReset{side_vis_hsv_v_kurtosisnboxplotOverallImage.tex}{}{}
\checkFileNoReset{side_vis_hsv_v_skewnessnboxplotOverallImage.tex}{}{}
\checkFileNoReset{side_vis_hsv_v_stddevnboxplotOverallImage.tex}{}{}
\checkFileNoReset{top_vis_hsv_v_averagenboxplotOverallImage.tex}{}{}
\checkFileNoReset{top_vis_hsv_v_histogram_bin_stackedOverallImage.tex}{}{}
\checkFileNoReset{top_vis_hsv_v_kurtosisnboxplotOverallImage.tex}{}{}
\checkFileNoReset{top_vis_hsv_v_skewnessnboxplotOverallImage.tex}{}{}
\checkFileNoReset{top_vis_hsv_v_stddevnboxplotOverallImage.tex}{}{}
\checkFileNoReset{side_vis_hsv_v_histogram_bin_01_0_12nboxplotOverallImage.tex}{}{}
\checkFileNoReset{side_vis_hsv_v_histogram_bin_02_12_25nboxplotOverallImage.tex}{}{}
\checkFileNoReset{side_vis_hsv_v_histogram_bin_03_25_38nboxplotOverallImage.tex}{}{}
\checkFileNoReset{side_vis_hsv_v_histogram_bin_04_38_51nboxplotOverallImage.tex}{}{}
\checkFileNoReset{side_vis_hsv_v_histogram_bin_05_51_63nboxplotOverallImage.tex}{}{}
\checkFileNoReset{side_vis_hsv_v_histogram_bin_06_63_76nboxplotOverallImage.tex}{}{}
\checkFileNoReset{top_vis_hsv_v_histogram_bin_01_0_12nboxplotOverallImage.tex}{}{}
\checkFileNoReset{top_vis_hsv_v_histogram_bin_02_12_25nboxplotOverallImage.tex}{}{}
\checkFileNoReset{top_vis_hsv_v_histogram_bin_03_25_38nboxplotOverallImage.tex}{}{}
\checkFileNoReset{top_vis_hsv_v_histogram_bin_04_38_51nboxplotOverallImage.tex}{}{}
\checkFileNoReset{top_vis_hsv_v_histogram_bin_05_51_63nboxplotOverallImage.tex}{}{}
\checkFileNoReset{top_vis_hsv_v_histogram_bin_06_63_76nboxplotOverallImage.tex}{}{}

\checkFileNoReset{side_vis_hsv_v_normalized_averagenboxplotOverallImage.tex}{}{}
\checkFileNoReset{side_vis_hsv_v_normalized_histogram_bin_stackedOverallImage.tex}{}{}
\checkFileNoReset{side_vis_hsv_v_normalized_kurtosisnboxplotOverallImage.tex}{}{}
\checkFileNoReset{side_vis_hsv_v_normalized_skewnessnboxplotOverallImage.tex}{}{}
\checkFileNoReset{side_vis_hsv_v_normalized_stddevnboxplotOverallImage.tex}{}{}
\checkFileNoReset{top_vis_hsv_v_normalized_averagenboxplotOverallImage.tex}{}{}
\checkFileNoReset{top_vis_hsv_v_normalized_histogram_bin_stackedOverallImage.tex}{}{}
\checkFileNoReset{top_vis_hsv_v_normalized_kurtosisnboxplotOverallImage.tex}{}{}
\checkFileNoReset{top_vis_hsv_v_normalized_skewnessnboxplotOverallImage.tex}{}{}
\checkFileNoReset{top_vis_hsv_v_normalized_stddevnboxplotOverallImage.tex}{}{}
\checkFileNoReset{side_vis_hsv_v_normalized_histogram_bin_01_0_12nboxplotOverallImage.tex}{}{}
\checkFileNoReset{side_vis_hsv_v_normalized_histogram_bin_02_12_25nboxplotOverallImage.tex}{}{}
\checkFileNoReset{side_vis_hsv_v_normalized_histogram_bin_03_25_38nboxplotOverallImage.tex}{}{}
\checkFileNoReset{side_vis_hsv_v_normalized_histogram_bin_04_38_51nboxplotOverallImage.tex}{}{}
\checkFileNoReset{side_vis_hsv_v_normalized_histogram_bin_05_51_63nboxplotOverallImage.tex}{}{}
\checkFileNoReset{side_vis_hsv_v_normalized_histogram_bin_06_63_76nboxplotOverallImage.tex}{}{}
\checkFileNoReset{top_vis_hsv_v_normalized_histogram_bin_01_0_12nboxplotOverallImage.tex}{}{}
\checkFileNoReset{top_vis_hsv_v_normalized_histogram_bin_02_12_25nboxplotOverallImage.tex}{}{}
\checkFileNoReset{top_vis_hsv_v_normalized_histogram_bin_03_25_38nboxplotOverallImage.tex}{}{}
\checkFileNoReset{top_vis_hsv_v_normalized_histogram_bin_04_38_51nboxplotOverallImage.tex}{}{}
\checkFileNoReset{top_vis_hsv_v_normalized_histogram_bin_05_51_63nboxplotOverallImage.tex}{}{}
\checkFileNoReset{top_vis_hsv_v_normalized_histogram_bin_06_63_76nboxplotOverallImage.tex}{}{}


\checkFileNoReset{side_vis_hsv_h_normalized_histogram_bin_stackedOverallImage.tex}{}{}
\checkFileNoReset{side_vis_hsv_h_histogram_bin_stackedOverallImage.tex}{}{}
\checkFileNoReset{top_vis_hsv_h_normalized_histogram_bin_stackedOverallImage.tex}{}{}
\checkFileNoReset{top_vis_hsv_h_histogram_bin_stackedOverallImage.tex}{}{}

\checkFileNoReset{side_vis_hsv_s_normalized_histogram_bin_stackedOverallImage.tex}{}{}
\checkFileNoReset{side_vis_hsv_s_histogram_bin_stackedOverallImage.tex}{}{}
\checkFileNoReset{top_vis_hsv_s_normalized_histogram_bin_stackedOverallImage.tex}{}{}
\checkFileNoReset{top_vis_hsv_s_histogram_bin_stackedOverallImage.tex}{}{}

\checkFileNoReset{side_vis_hsv_v_normalized_histogram_bin_stackedOverallImage.tex}{}{}
\checkFileNoReset{side_vis_hsv_v_histogram_bin_stackedOverallImage.tex}{}{}
\checkFileNoReset{top_vis_hsv_v_normalized_histogram_bin_stackedOverallImage.tex}{}{}
\checkFileNoReset{top_vis_hsv_v_histogram_bin_stackedOverallImage.tex}{}{}

\ifthenelse{\boolean{isFile1}}{

	\resetClear
	\clearpage
	\section{Visible light color analysis}
	In this section the plant pixels color shade values are investigated in
	detail. The color components color hue, color saturation or brightness are
	analyzed independently.
	The following statistical analysis is performed for each color component of
	the plant pixel values: average, standard deviation, skewness and kurtosis.
	
	\resetBoolean
	\checkFileNoReset{side_vis_hsv_h_averagenboxplotOverallImage.tex}{}{}
	\checkFileNoReset{side_vis_hsv_h_histogram_bin_stackedOverallImage.tex}{}{}
	\checkFileNoReset{side_vis_hsv_h_kurtosisnboxplotOverallImage.tex}{}{}
	\checkFileNoReset{side_vis_hsv_h_skewnessnboxplotOverallImage.tex}{}{}
	\checkFileNoReset{side_vis_hsv_h_stddevnboxplotOverallImage.tex}{}{}
	\checkFileNoReset{side_vis_hsv_h_histogram_bin_01_0_12nboxplotOverallImage.tex}{}{}
	\checkFileNoReset{side_vis_hsv_h_histogram_bin_02_12_25nboxplotOverallImage.tex}{}{}
	\checkFileNoReset{side_vis_hsv_h_histogram_bin_03_25_38nboxplotOverallImage.tex}{}{}
	\checkFileNoReset{side_vis_hsv_h_histogram_bin_04_38_51nboxplotOverallImage.tex}{}{}
	\checkFileNoReset{side_vis_hsv_h_histogram_bin_05_51_63nboxplotOverallImage.tex}{}{}
	\checkFileNoReset{side_vis_hsv_h_histogram_bin_06_63_76nboxplotOverallImage.tex}{}{}
	\checkFileNoReset{side_vis_hsv_h_normalized_averagenboxplotOverallImage.tex}{}{}
	\checkFileNoReset{side_vis_hsv_h_normalized_histogram_bin_stackedOverallImage.tex}{}{}
	\checkFileNoReset{side_vis_hsv_h_normalized_kurtosisnboxplotOverallImage.tex}{}{}
	\checkFileNoReset{side_vis_hsv_h_normalized_skewnessnboxplotOverallImage.tex}{}{}
	\checkFileNoReset{side_vis_hsv_h_normalized_stddevnboxplotOverallImage.tex}{}{}
	\checkFileNoReset{side_vis_hsv_h_normalized_histogram_bin_01_0_12nboxplotOverallImage.tex}{}{}
	\checkFileNoReset{side_vis_hsv_h_normalized_histogram_bin_02_12_25nboxplotOverallImage.tex}{}{}
	\checkFileNoReset{side_vis_hsv_h_normalized_histogram_bin_03_25_38nboxplotOverallImage.tex}{}{}
	\checkFileNoReset{side_vis_hsv_h_normalized_histogram_bin_04_38_51nboxplotOverallImage.tex}{}{}
	\checkFileNoReset{side_vis_hsv_h_normalized_histogram_bin_05_51_63nboxplotOverallImage.tex}{}{}
	\checkFileNoReset{side_vis_hsv_h_normalized_histogram_bin_06_63_76nboxplotOverallImage.tex}{}{}
	\checkFileNoReset{side_vis_hsv_s_averagenboxplotOverallImage.tex}{}{}
	\checkFileNoReset{side_vis_hsv_s_histogram_bin_stackedOverallImage.tex}{}{}
	\checkFileNoReset{side_vis_hsv_s_kurtosisnboxplotOverallImage.tex}{}{}
	\checkFileNoReset{side_vis_hsv_s_skewnessnboxplotOverallImage.tex}{}{}
	\checkFileNoReset{side_vis_hsv_s_stddevnboxplotOverallImage.tex}{}{}
	\checkFileNoReset{side_vis_hsv_s_histogram_bin_01_0_12nboxplotOverallImage.tex}{}{}
	\checkFileNoReset{side_vis_hsv_s_histogram_bin_02_12_25nboxplotOverallImage.tex}{}{}
	\checkFileNoReset{side_vis_hsv_s_histogram_bin_03_25_38nboxplotOverallImage.tex}{}{}
	\checkFileNoReset{side_vis_hsv_s_histogram_bin_04_38_51nboxplotOverallImage.tex}{}{}
	\checkFileNoReset{side_vis_hsv_s_histogram_bin_05_51_63nboxplotOverallImage.tex}{}{}
	\checkFileNoReset{side_vis_hsv_s_histogram_bin_06_63_76nboxplotOverallImage.tex}{}{}
	\checkFileNoReset{side_vis_hsv_s_normalized_averagenboxplotOverallImage.tex}{}{}
	\checkFileNoReset{side_vis_hsv_s_normalized_histogram_bin_stackedOverallImage.tex}{}{}
	\checkFileNoReset{side_vis_hsv_s_normalized_kurtosisnboxplotOverallImage.tex}{}{}
	\checkFileNoReset{side_vis_hsv_s_normalized_skewnessnboxplotOverallImage.tex}{}{}
	\checkFileNoReset{side_vis_hsv_s_normalized_stddevnboxplotOverallImage.tex}{}{}
	\checkFileNoReset{side_vis_hsv_s_normalized_histogram_bin_01_0_12nboxplotOverallImage.tex}{}{}
	\checkFileNoReset{side_vis_hsv_s_normalized_histogram_bin_02_12_25nboxplotOverallImage.tex}{}{}
	\checkFileNoReset{side_vis_hsv_s_normalized_histogram_bin_03_25_38nboxplotOverallImage.tex}{}{}
	\checkFileNoReset{side_vis_hsv_s_normalized_histogram_bin_04_38_51nboxplotOverallImage.tex}{}{}
	\checkFileNoReset{side_vis_hsv_s_normalized_histogram_bin_05_51_63nboxplotOverallImage.tex}{}{}
	\checkFileNoReset{side_vis_hsv_s_normalized_histogram_bin_06_63_76nboxplotOverallImage.tex}{}{}
	\checkFileNoReset{side_vis_hsv_v_averagenboxplotOverallImage.tex}{}{}
	\checkFileNoReset{side_vis_hsv_v_histogram_bin_stackedOverallImage.tex}{}{}
	\checkFileNoReset{side_vis_hsv_v_kurtosisnboxplotOverallImage.tex}{}{}
	\checkFileNoReset{side_vis_hsv_v_skewnessnboxplotOverallImage.tex}{}{}
	\checkFileNoReset{side_vis_hsv_v_stddevnboxplotOverallImage.tex}{}{}
	\checkFileNoReset{side_vis_hsv_v_histogram_bin_01_0_12nboxplotOverallImage.tex}{}{}
	\checkFileNoReset{side_vis_hsv_v_histogram_bin_02_12_25nboxplotOverallImage.tex}{}{}
	\checkFileNoReset{side_vis_hsv_v_histogram_bin_03_25_38nboxplotOverallImage.tex}{}{}
	\checkFileNoReset{side_vis_hsv_v_histogram_bin_04_38_51nboxplotOverallImage.tex}{}{}
	\checkFileNoReset{side_vis_hsv_v_histogram_bin_05_51_63nboxplotOverallImage.tex}{}{}
	\checkFileNoReset{side_vis_hsv_v_histogram_bin_06_63_76nboxplotOverallImage.tex}{}{}
	\checkFileNoReset{side_vis_hsv_v_normalized_averagenboxplotOverallImage.tex}{}{}
	\checkFileNoReset{side_vis_hsv_v_normalized_histogram_bin_stackedOverallImage.tex}{}{}
	\checkFileNoReset{side_vis_hsv_v_normalized_kurtosisnboxplotOverallImage.tex}{}{}
	\checkFileNoReset{side_vis_hsv_v_normalized_skewnessnboxplotOverallImage.tex}{}{}
	\checkFileNoReset{side_vis_hsv_v_normalized_stddevnboxplotOverallImage.tex}{}{}
	\checkFileNoReset{side_vis_hsv_v_normalized_histogram_bin_01_0_12nboxplotOverallImage.tex}{}{}
	\checkFileNoReset{side_vis_hsv_v_normalized_histogram_bin_02_12_25nboxplotOverallImage.tex}{}{}
	\checkFileNoReset{side_vis_hsv_v_normalized_histogram_bin_03_25_38nboxplotOverallImage.tex}{}{}
	\checkFileNoReset{side_vis_hsv_v_normalized_histogram_bin_04_38_51nboxplotOverallImage.tex}{}{}
	\checkFileNoReset{side_vis_hsv_v_normalized_histogram_bin_05_51_63nboxplotOverallImage.tex}{}{}
	\checkFileNoReset{side_vis_hsv_v_normalized_histogram_bin_06_63_76nboxplotOverallImage.tex}{}{}
	\checkFileNoReset{side_vis_hsv_h_normalized_histogram_bin_stackedOverallImage.tex}{}{}
	\checkFileNoReset{side_vis_hsv_h_histogram_bin_stackedOverallImage.tex}{}{}
	\checkFileNoReset{side_vis_hsv_s_normalized_histogram_bin_stackedOverallImage.tex}{}{}
	\checkFileNoReset{side_vis_hsv_s_histogram_bin_stackedOverallImage.tex}{}{}
	\checkFileNoReset{side_vis_hsv_v_normalized_histogram_bin_stackedOverallImage.tex}{}{}
	\checkFileNoReset{side_vis_hsv_v_histogram_bin_stackedOverallImage.tex}{}{}
	\ifthenelse{\boolean{isFile1}}{
		
% 		\resetClearSub
% 		\ownClearPage
  		\subsection{Side}
	
		\resetBoolean
		\checkFileNoReset{side_vis_hsv_h_averagenboxplotOverallImage.tex}{}{}
		\checkFileNoReset{side_vis_hsv_h_histogram_bin_stackedOverallImage.tex}{}{}
		\checkFileNoReset{side_vis_hsv_h_kurtosisnboxplotOverallImage.tex}{}{}
		\checkFileNoReset{side_vis_hsv_h_skewnessnboxplotOverallImage.tex}{}{}
		\checkFileNoReset{side_vis_hsv_h_stddevnboxplotOverallImage.tex}{}{}
		\checkFileNoReset{side_vis_hsv_h_histogram_bin_01_0_12nboxplotOverallImage.tex}{}{}
		\checkFileNoReset{side_vis_hsv_h_histogram_bin_02_12_25nboxplotOverallImage.tex}{}{}
		\checkFileNoReset{side_vis_hsv_h_histogram_bin_03_25_38nboxplotOverallImage.tex}{}{}
		\checkFileNoReset{side_vis_hsv_h_histogram_bin_04_38_51nboxplotOverallImage.tex}{}{}
		\checkFileNoReset{side_vis_hsv_h_histogram_bin_05_51_63nboxplotOverallImage.tex}{}{}
		\checkFileNoReset{side_vis_hsv_h_histogram_bin_06_63_76nboxplotOverallImage.tex}{}{}
		\checkFileNoReset{side_vis_hsv_h_normalized_averagenboxplotOverallImage.tex}{}{}
		\checkFileNoReset{side_vis_hsv_h_normalized_histogram_bin_stackedOverallImage.tex}{}{}
		\checkFileNoReset{side_vis_hsv_h_normalized_kurtosisnboxplotOverallImage.tex}{}{}
		\checkFileNoReset{side_vis_hsv_h_normalized_skewnessnboxplotOverallImage.tex}{}{}
		\checkFileNoReset{side_vis_hsv_h_normalized_stddevnboxplotOverallImage.tex}{}{}
		\checkFileNoReset{side_vis_hsv_h_normalized_histogram_bin_01_0_12nboxplotOverallImage.tex}{}{}
		\checkFileNoReset{side_vis_hsv_h_normalized_histogram_bin_02_12_25nboxplotOverallImage.tex}{}{}
		\checkFileNoReset{side_vis_hsv_h_normalized_histogram_bin_03_25_38nboxplotOverallImage.tex}{}{}
		\checkFileNoReset{side_vis_hsv_h_normalized_histogram_bin_04_38_51nboxplotOverallImage.tex}{}{}
		\checkFileNoReset{side_vis_hsv_h_normalized_histogram_bin_05_51_63nboxplotOverallImage.tex}{}{}
		\checkFileNoReset{side_vis_hsv_h_normalized_histogram_bin_06_63_76nboxplotOverallImage.tex}{}{}
		\checkFileNoReset{side_vis_hsv_h_normalized_histogram_bin_stackedOverallImage.tex}{}{}
		\checkFileNoReset{side_vis_hsv_h_histogram_bin_stackedOverallImage.tex}{}{}
		\ifthenelse{\boolean{isFile1}}{
			
			\resetClearSub
			\ownClearPage
	  		\subsubsection{Color shade}
	
			\resetBoolean
			\checkFileNoReset{side_vis_hsv_h_averagenboxplotOverallImage.tex}{}{}
			\checkFileNoReset{}{side_vis_hsv_h_normalized_averagenboxplotOverallImage.tex}{}
			\ifthenelse{\boolean{isFile2}}{
	
				\ownClearPageSub
				\paragraph{Average hue (zoom corrected)}~
% 				\begin{itemize}
% 				\item Area which is enclosed of the convex hull (zoom corrected)
% 				\item Unit: px
% 				\end{itemize}
				\loadTex{side_vis_hsv_h_normalized_averagenboxplotOverallImage}
			}{
				\ifthenelse{\boolean{isFile1}}{
					\ownClearPageSub
					\paragraph{Average hue}~
% 					\begin{itemize}
% 					\item Area which is enclosed of the convex hull
% 					\item Unit: px
% 					\end{itemize}
					\loadTex{side_vis_hsv_h_averagenboxplotOverallImage}
				}{}
			}
	
			\resetBoolean
			\checkFileNoReset{side_vis_hsv_h_stddevnboxplotOverallImage.tex}{}{}
			\checkFileNoReset{}{side_vis_hsv_h_normalized_stddevnboxplotOverallImage.tex}{}
			\ifthenelse{\boolean{isFile2}}{
	
				\ownClearPageSub
				\paragraph{Standard devition (zoom corrected)}~
				\loadTex{side_vis_hsv_h_normalized_stddevnboxplotOverallImage}
			}{
				\ifthenelse{\boolean{isFile1}}{
					\ownClearPageSub
					\paragraph{Standard devition}~
					\loadTex{side_vis_hsv_h_stddevnboxplotOverallImage}
				}{}
			}
	
			\resetBoolean
			\checkFileNoReset{side_vis_hsv_h_skewnessnboxplotOverallImage.tex}{}{}
			\checkFileNoReset{}{side_vis_hsv_h_normalized_skewnessnboxplotOverallImage.tex}{}
			\ifthenelse{\boolean{isFile2}}{
	
				\ownClearPageSub
				\paragraph{Skewness (zoom corrected)}~
				\newline
				The skewness can be described as following:  

				“In probability theory and statistics, skewness is a measure of the
				asymmetry of the probability distributionof a real-valued random variable.
				The skewness value can be positive or negative, or even undefined.
				Qualitatively, a negative skew indicates that the tail on the left side of
				the probability density function islonger than the right side and the bulk
				of the values (possibly including the median) lie to the right of the mean.
				A positive skew indicates that the tail on the right side is longer than the
				left side and the bulk of the values lie to the left of the mean. A zero
				value indicates that the values are relatively evenly distributed on both
				sides of the mean, typically but not necessarily implying a symmetric
				distribution.”
				
				\begin{tiny}
			 	Wikipedia contributors, "Skewness",
			 	Wikipedia, The Free Encyclopedia,
				\url{http://en.wikipedia.org/w/index.php?title=Skewness&oldid=499258725}
				(accessed June 27, 2012).
				\end{tiny}
				
				\loadTex{side_vis_hsv_h_normalized_skewnessnboxplotOverallImage}
			}{
				\ifthenelse{\boolean{isFile1}}{
					\ownClearPageSub
					\paragraph{Skewness}~
					\newline
					The skewness can be described as following:  

					“In probability theory and statistics, skewness is a measure of the
					asymmetry of the probability distributionof a real-valued random variable.
					The skewness value can be positive or negative, or even undefined.
					Qualitatively, a negative skew indicates that the tail on the left side of
					the probability density function islonger than the right side and the bulk
					of the values (possibly including the median) lie to the right of the mean.
					A positive skew indicates that the tail on the right side is longer than the
					left side and the bulk of the values lie to the left of the mean. A zero
					value indicates that the values are relatively evenly distributed on both
					sides of the mean, typically but not necessarily implying a symmetric
					distribution.”
					
					\begin{tiny}
				 	Wikipedia contributors, "Skewness",
				 	Wikipedia, The Free Encyclopedia,
					\url{http://en.wikipedia.org/w/index.php?title=Skewness&oldid=499258725}
					(accessed June 27, 2012).
					\end{tiny}
					
					\loadTex{side_vis_hsv_h_skewnessnboxplotOverallImage}
				}{}
			}
			
			\resetBoolean
			\checkFileNoReset{side_vis_hsv_h_kurtosisnboxplotOverallImage.tex}{}{}
			\checkFileNoReset{}{side_vis_hsv_h_normalized_kurtosisnboxplotOverallImage.tex}{}
			\ifthenelse{\boolean{isFile2}}{
	
				\ownClearPageSub
				\paragraph{Skewness (zoom corrected)}~
				\newline
				The term kurtosis is described as following:

				“In probability theory and statistics, kurtosis (from the Greek word κυρτός,
				kyrtos or kurtos, meaning bulging) is any measure of the "peakedness" of the
				probability distribution of a real-valued random variable. In a similar
				way to the concept of skewness, kurtosis is a descriptor of the shape of a
				probability distribution and, just as for skewness, there are different ways
				of quantifying it for a theoretical distribution and corresponding ways of
				estimating it from a sample from a population.

				One common measure of kurtosis, originating with Karl Pearson, is based on a
				scaled version of the fourth moment of the data or population, but it has
				been argued that this measure really measures heavy tails, and not
				peakedness. For this measure, higher kurtosis means more of the variance
				is the result of infrequent extreme deviations, as opposed to frequent
				modestly sized deviations.”

				\begin{tiny}
			 	Wikipedia contributors, "Kurtosis",
			 	Wikipedia, The Free Encyclopedia,
				\url{http://en.wikipedia.org/w/index.php?title=Kurtosis&oldid=496203029}
				(accessed June 27, 2012).
				\end{tiny}
				
				\loadTex{side_vis_hsv_h_normalized_kurtosisnboxplotOverallImage}
			}{
				\ifthenelse{\boolean{isFile1}}{
					\ownClearPageSub
					\paragraph{Skewness}~
					\newline
					The term kurtosis is described as following:

					“In probability theory and statistics, kurtosis (from the Greek word κυρτός,
					kyrtos or kurtos, meaning bulging) is any measure of the "peakedness" of the
					probability distribution of a real-valued random variable. In a similar
					way to the concept of skewness, kurtosis is a descriptor of the shape of a
					probability distribution and, just as for skewness, there are different ways
					of quantifying it for a theoretical distribution and corresponding ways of
					estimating it from a sample from a population.
	
					One common measure of kurtosis, originating with Karl Pearson, is based on a
					scaled version of the fourth moment of the data or population, but it has
					been argued that this measure really measures heavy tails, and not
					peakedness. For this measure, higher kurtosis means more of the variance
					is the result of infrequent extreme deviations, as opposed to frequent
					modestly sized deviations.”
	
					\begin{tiny}
				 	Wikipedia contributors, "Kurtosis",
				 	Wikipedia, The Free Encyclopedia,
					\url{http://en.wikipedia.org/w/index.php?title=Kurtosis&oldid=496203029}
					(accessed June 27, 2012).
					\end{tiny}
					
					\loadTex{side_vis_hsv_h_kurtosisnboxplotOverallImage}
				}{}
			}
			
			\resetBoolean
			\checkFileNoReset{side_vis_hsv_h_histogram_bin_01_0_12nboxplotOverallImage.tex}{}{}
			\checkFileNoReset{}{side_vis_hsv_h_normalized_histogram_bin_01_0_12nboxplotOverallImage.tex}{}
			\ifthenelse{\boolean{isFile2}}{
	
				\ownClearPageSub
				\paragraph{Hue bin 1 (01-12) (zoom corrected)}~
				\loadTex{side_vis_hsv_h_normalized_histogram_bin_01_0_12nboxplotOverallImage}
			}{
				\ifthenelse{\boolean{isFile1}}{
					\ownClearPageSub
					\paragraph{Hue bin 1 (01-12)}~
					\loadTex{side_vis_hsv_h_histogram_bin_01_0_12nboxplotOverallImage}
				}{}
			}
			
			\resetBoolean
			\checkFileNoReset{side_vis_hsv_h_histogram_bin_02_12_25nboxplotOverallImage.tex}{}{}
			\checkFileNoReset{}{side_vis_hsv_h_normalized_histogram_bin_02_12_25nboxplotOverallImage.tex}{}
			\ifthenelse{\boolean{isFile2}}{
	
				\ownClearPageSub
				\paragraph{Hue bin 2 (12-25) (zoom corrected)}~
				\loadTex{side_vis_hsv_h_normalized_histogram_bin_02_12_25nboxplotOverallImage}
			}{
				\ifthenelse{\boolean{isFile1}}{
					\ownClearPageSub
					\paragraph{Hue bin 2 (12-25) }~
					\loadTex{side_vis_hsv_h_histogram_bin_02_12_25nboxplotOverallImage}
				}{}
			}
			
			\resetBoolean
			\checkFileNoReset{side_vis_hsv_h_histogram_bin_03_25_38nboxplotOverallImage.tex}{}{}
			\checkFileNoReset{}{side_vis_hsv_h_normalized_histogram_bin_03_25_38nboxplotOverallImage.tex}{}
			\ifthenelse{\boolean{isFile2}}{
	
				\ownClearPageSub
				\paragraph{Hue bin 3 (25-38) (zoom corrected)}~
				\loadTex{side_vis_hsv_h_normalized_histogram_bin_03_25_38nboxplotOverallImage}
			}{
				\ifthenelse{\boolean{isFile1}}{
					\ownClearPageSub
					\paragraph{Hue bin 3 (25-38) }~
					\loadTex{side_vis_hsv_h_histogram_bin_03_25_38nboxplotOverallImage}
				}{}
			}
			
			\resetBoolean
			\checkFileNoReset{side_vis_hsv_h_histogram_bin_04_38_51nboxplotOverallImage.tex}{}{}
			\checkFileNoReset{}{side_vis_hsv_h_normalized_histogram_bin_04_38_51nboxplotOverallImage.tex}{}
			\ifthenelse{\boolean{isFile2}}{
	
				\ownClearPageSub
				\paragraph{Hue bin 4 (38-51) (zoom corrected)}~
				\loadTex{side_vis_hsv_h_normalized_histogram_bin_04_38_51nboxplotOverallImage}
			}{
				\ifthenelse{\boolean{isFile1}}{
					\ownClearPageSub
					\paragraph{Hue bin 4 (38-51) }~
					\loadTex{side_vis_hsv_h_histogram_bin_04_38_51nboxplotOverallImage}
				}{}
			}
			
			\resetBoolean
			\checkFileNoReset{side_vis_hsv_h_histogram_bin_05_51_63nboxplotOverallImage.tex}{}{}
			\checkFileNoReset{}{side_vis_hsv_h_normalized_histogram_bin_05_51_63nboxplotOverallImage.tex}{}
			\ifthenelse{\boolean{isFile2}}{
	
				\ownClearPageSub
				\paragraph{Hue bin 5 (51-63) (zoom corrected)}~
				\loadTex{side_vis_hsv_h_normalized_histogram_bin_05_51_63nboxplotOverallImage}
			}{
				\ifthenelse{\boolean{isFile1}}{
					\ownClearPageSub
					\paragraph{Hue bin 5 (51-63) }~
					\loadTex{side_vis_hsv_h_histogram_bin_05_51_63nboxplotOverallImage}
				}{}
			}
			
			\resetBoolean
			\checkFileNoReset{side_vis_hsv_h_histogram_bin_06_63_76nboxplotOverallImage.tex}{}{}
			\checkFileNoReset{}{side_vis_hsv_h_normalized_histogram_bin_06_63_76nboxplotOverallImage.tex}{}
			\ifthenelse{\boolean{isFile2}}{
	
				\ownClearPageSub
				\paragraph{Hue bin 6 (63-76) (zoom corrected)}~
				\loadTex{side_vis_hsv_h_normalized_histogram_bin_06_63_76nboxplotOverallImage}
			}{
				\ifthenelse{\boolean{isFile1}}{
					\ownClearPageSub
					\paragraph{Hue bin 6 (63-76) }~
					\loadTex{side_vis_hsv_h_histogram_bin_06_63_76nboxplotOverallImage}
				}{}
			}
		
			\resetBoolean
			\checkFileNoReset{side_vis_hsv_h_histogram_bin_stackedOverallImage.tex}{}{}
			\checkFileNoReset{}{side_vis_hsv_h_normalized_histogram_bin_stackedOverallImage.tex}{}
			\ifthenelse{\boolean{isFile2}}{
	
				\ownClearPageSub
				\paragraph{Color histogram (zoom corrected)}~
				\loadTex{side_vis_hsv_h_normalized_histogram_bin_stackedOverallImage}
			}{
				\ifthenelse{\boolean{isFile1}}{
					\ownClearPageSub
					\paragraph{Color histogram}~
					\loadTex{side_vis_hsv_h_histogram_bin_stackedOverallImage}
				}{}
			}
		}

		\resetBoolean
		\checkFileNoReset{side_vis_hsv_s_averagenboxplotOverallImage.tex}{}{}
		\checkFileNoReset{side_vis_hsv_s_histogram_bin_stackedOverallImage.tex}{}{}
		\checkFileNoReset{side_vis_hsv_s_kurtosisnboxplotOverallImage.tex}{}{}
		\checkFileNoReset{side_vis_hsv_s_skewnessnboxplotOverallImage.tex}{}{}
		\checkFileNoReset{side_vis_hsv_s_stddevnboxplotOverallImage.tex}{}{}
		\checkFileNoReset{side_vis_hsv_s_histogram_bin_01_0_12nboxplotOverallImage.tex}{}{}
		\checkFileNoReset{side_vis_hsv_s_histogram_bin_02_12_25nboxplotOverallImage.tex}{}{}
		\checkFileNoReset{side_vis_hsv_s_histogram_bin_03_25_38nboxplotOverallImage.tex}{}{}
		\checkFileNoReset{side_vis_hsv_s_histogram_bin_04_38_51nboxplotOverallImage.tex}{}{}
		\checkFileNoReset{side_vis_hsv_s_histogram_bin_05_51_63nboxplotOverallImage.tex}{}{}
		\checkFileNoReset{side_vis_hsv_s_histogram_bin_06_63_76nboxplotOverallImage.tex}{}{}
		\checkFileNoReset{side_vis_hsv_s_normalized_averagenboxplotOverallImage.tex}{}{}
		\checkFileNoReset{side_vis_hsv_s_normalized_histogram_bin_stackedOverallImage.tex}{}{}
		\checkFileNoReset{side_vis_hsv_s_normalized_kurtosisnboxplotOverallImage.tex}{}{}
		\checkFileNoReset{side_vis_hsv_s_normalized_skewnessnboxplotOverallImage.tex}{}{}
		\checkFileNoReset{side_vis_hsv_s_normalized_stddevnboxplotOverallImage.tex}{}{}
		\checkFileNoReset{side_vis_hsv_s_normalized_histogram_bin_01_0_12nboxplotOverallImage.tex}{}{}
		\checkFileNoReset{side_vis_hsv_s_normalized_histogram_bin_02_12_25nboxplotOverallImage.tex}{}{}
		\checkFileNoReset{side_vis_hsv_s_normalized_histogram_bin_03_25_38nboxplotOverallImage.tex}{}{}
		\checkFileNoReset{side_vis_hsv_s_normalized_histogram_bin_04_38_51nboxplotOverallImage.tex}{}{}
		\checkFileNoReset{side_vis_hsv_s_normalized_histogram_bin_05_51_63nboxplotOverallImage.tex}{}{}
		\checkFileNoReset{side_vis_hsv_s_normalized_histogram_bin_06_63_76nboxplotOverallImage.tex}{}{}
		\checkFileNoReset{side_vis_hsv_s_normalized_histogram_bin_stackedOverallImage.tex}{}{}
		\checkFileNoReset{side_vis_hsv_s_histogram_bin_stackedOverallImage.tex}{}{}
		\ifthenelse{\boolean{isFile1}}{
			
			\resetClearSub
			\ownClearPage
	  		\subsubsection{Sataturation}
	
			\resetBoolean
			\checkFileNoReset{side_vis_hsv_s_averagenboxplotOverallImage.tex}{}{}
			\checkFileNoReset{}{side_vis_hsv_s_normalized_averagenboxplotOverallImage.tex}{}
			\ifthenelse{\boolean{isFile2}}{
	
				\ownClearPageSub
				\paragraph{Average sataturation (zoom corrected)}~
% 				\begin{itemize}
% 				\item Area which is enclosed of the convex hull (zoom corrected)
% 				\item Unit: px
% 				\end{itemize}
				\loadTex{side_vis_hsv_s_normalized_averagenboxplotOverallImage}
			}{
				\ifthenelse{\boolean{isFile1}}{
					\ownClearPageSub
					\paragraph{Average sataturation}~
% 					\begin{itemize}
% 					\item Area which is enclosed of the convex hull
% 					\item Unit: px
% 					\end{itemize}
					\loadTex{side_vis_hsv_s_averagenboxplotOverallImage}
				}{}
			}
	
			\resetBoolean
			\checkFileNoReset{side_vis_hsv_s_stddevnboxplotOverallImage.tex}{}{}
			\checkFileNoReset{}{side_vis_hsv_s_normalized_stddevnboxplotOverallImage.tex}{}
			\ifthenelse{\boolean{isFile2}}{
	
				\ownClearPageSub
				\paragraph{Standard devition (zoom corrected)}~
				\loadTex{side_vis_hsv_s_normalized_stddevnboxplotOverallImage}
			}{
				\ifthenelse{\boolean{isFile1}}{
					\ownClearPageSub
					\paragraph{Standard devition}~
					\loadTex{side_vis_hsv_s_stddevnboxplotOverallImage}
				}{}
			}
	
			\resetBoolean
			\checkFileNoReset{side_vis_hsv_s_skewnessnboxplotOverallImage.tex}{}{}
			\checkFileNoReset{}{side_vis_hsv_s_normalized_skewnessnboxplotOverallImage.tex}{}
			\ifthenelse{\boolean{isFile2}}{
	
				\ownClearPageSub
				\paragraph{Skewness (zoom corrected)}~
				\newline
				The skewness can be described as following:  

				“In probability theory and statistics, skewness is a measure of the
				asymmetry of the probability distributionof a real-valued random variable.
				The skewness value can be positive or negative, or even undefined.
				Qualitatively, a negative skew indicates that the tail on the left side of
				the probability density function islonger than the right side and the bulk
				of the values (possibly including the median) lie to the right of the mean.
				A positive skew indicates that the tail on the right side is longer than the
				left side and the bulk of the values lie to the left of the mean. A zero
				value indicates that the values are relatively evenly distributed on both
				sides of the mean, typically but not necessarily implying a symmetric
				distribution.”
				
				\begin{tiny}
			 	Wikipedia contributors, "Skewness",
			 	Wikipedia, The Free Encyclopedia,
				\url{http://en.wikipedia.org/w/index.php?title=Skewness&oldid=499258725}
				(accessed June 27, 2012).
				\end{tiny}
				
				\loadTex{side_vis_hsv_s_normalized_skewnessnboxplotOverallImage}
			}{
				\ifthenelse{\boolean{isFile1}}{
					\ownClearPageSub
					\paragraph{Skewness}~
					\newline
					The skewness can be described as following:  

					“In probability theory and statistics, skewness is a measure of the
					asymmetry of the probability distributionof a real-valued random variable.
					The skewness value can be positive or negative, or even undefined.
					Qualitatively, a negative skew indicates that the tail on the left side of
					the probability density function islonger than the right side and the bulk
					of the values (possibly including the median) lie to the right of the mean.
					A positive skew indicates that the tail on the right side is longer than the
					left side and the bulk of the values lie to the left of the mean. A zero
					value indicates that the values are relatively evenly distributed on both
					sides of the mean, typically but not necessarily implying a symmetric
					distribution.”
					
					\begin{tiny}
				 	Wikipedia contributors, "Skewness",
				 	Wikipedia, The Free Encyclopedia,
					\url{http://en.wikipedia.org/w/index.php?title=Skewness&oldid=499258725}
					(accessed June 27, 2012).
					\end{tiny}
					
					\loadTex{side_vis_hsv_s_skewnessnboxplotOverallImage}
				}{}
			}
			
			\resetBoolean
			\checkFileNoReset{side_vis_hsv_s_kurtosisnboxplotOverallImage.tex}{}{}
			\checkFileNoReset{}{side_vis_hsv_s_normalized_kurtosisnboxplotOverallImage.tex}{}
			\ifthenelse{\boolean{isFile2}}{
	
				\ownClearPageSub
				\paragraph{Skewness (zoom corrected)}~
				\newline
				The term kurtosis is described as following:

				“In probability theory and statistics, kurtosis (from the Greek word κυρτός,
				kyrtos or kurtos, meaning bulging) is any measure of the "peakedness" of the
				probability distribution of a real-valued random variable. In a similar
				way to the concept of skewness, kurtosis is a descriptor of the shape of a
				probability distribution and, just as for skewness, there are different ways
				of quantifying it for a theoretical distribution and corresponding ways of
				estimating it from a sample from a population.

				One common measure of kurtosis, originating with Karl Pearson, is based on a
				scaled version of the fourth moment of the data or population, but it has
				been argued that this measure really measures heavy tails, and not
				peakedness. For this measure, higher kurtosis means more of the variance
				is the result of infrequent extreme deviations, as opposed to frequent
				modestly sized deviations.”

				\begin{tiny}
			 	Wikipedia contributors, "Kurtosis",
			 	Wikipedia, The Free Encyclopedia,
				\url{http://en.wikipedia.org/w/index.php?title=Kurtosis&oldid=496203029}
				(accessed June 27, 2012).
				\end{tiny}
				
				\loadTex{side_vis_hsv_s_normalized_kurtosisnboxplotOverallImage}
			}{
				\ifthenelse{\boolean{isFile1}}{
					\ownClearPageSub
					\paragraph{Skewness}~
					\newline
					The term kurtosis is described as following:

					“In probability theory and statistics, kurtosis (from the Greek word κυρτός,
					kyrtos or kurtos, meaning bulging) is any measure of the "peakedness" of the
					probability distribution of a real-valued random variable. In a similar
					way to the concept of skewness, kurtosis is a descriptor of the shape of a
					probability distribution and, just as for skewness, there are different ways
					of quantifying it for a theoretical distribution and corresponding ways of
					estimating it from a sample from a population.
	
					One common measure of kurtosis, originating with Karl Pearson, is based on a
					scaled version of the fourth moment of the data or population, but it has
					been argued that this measure really measures heavy tails, and not
					peakedness. For this measure, higher kurtosis means more of the variance
					is the result of infrequent extreme deviations, as opposed to frequent
					modestly sized deviations.”
	
					\begin{tiny}
				 	Wikipedia contributors, "Kurtosis",
				 	Wikipedia, The Free Encyclopedia,
					\url{http://en.wikipedia.org/w/index.php?title=Kurtosis&oldid=496203029}
					(accessed June 27, 2012).
					\end{tiny}
					
					\loadTex{side_vis_hsv_s_kurtosisnboxplotOverallImage}
				}{}
			}
			
			\resetBoolean
			\checkFileNoReset{side_vis_hsv_s_histogram_bin_01_0_12nboxplotOverallImage.tex}{}{}
			\checkFileNoReset{}{side_vis_hsv_s_normalized_histogram_bin_01_0_12nboxplotOverallImage.tex}{}
			\ifthenelse{\boolean{isFile2}}{
	
				\ownClearPageSub
				\paragraph{Sataturation bin 1 (01-12) (zoom corrected)}~
				\loadTex{side_vis_hsv_s_normalized_histogram_bin_01_0_12nboxplotOverallImage}
			}{
				\ifthenelse{\boolean{isFile1}}{
					\ownClearPageSub
					\paragraph{Sataturation bin 1 (01-12) }~
					\loadTex{side_vis_hsv_s_histogram_bin_01_0_12nboxplotOverallImage}
				}{}
			}
			
			\resetBoolean
			\checkFileNoReset{side_vis_hsv_s_histogram_bin_02_12_25nboxplotOverallImage.tex}{}{}
			\checkFileNoReset{}{side_vis_hsv_s_normalized_histogram_bin_02_12_25nboxplotOverallImage.tex}{}
			\ifthenelse{\boolean{isFile2}}{
	
				\ownClearPageSub
				\paragraph{Sataturation bin 2 (12-25) (zoom corrected)}~
				\loadTex{side_vis_hsv_s_normalized_histogram_bin_02_12_25nboxplotOverallImage}
			}{
				\ifthenelse{\boolean{isFile1}}{
					\ownClearPageSub
					\paragraph{Sataturation bin 2 (12-25) }~
					\loadTex{side_vis_hsv_s_histogram_bin_02_12_25nboxplotOverallImage}
				}{}
			}
			
			\resetBoolean
			\checkFileNoReset{side_vis_hsv_s_histogram_bin_03_25_38nboxplotOverallImage.tex}{}{}
			\checkFileNoReset{}{side_vis_hsv_s_normalized_histogram_bin_03_25_38nboxplotOverallImage.tex}{}
			\ifthenelse{\boolean{isFile2}}{
	
				\ownClearPageSub
				\paragraph{Sataturation bin 3 (25-38) (zoom corrected)}~
				\loadTex{side_vis_hsv_s_normalized_histogram_bin_03_25_38nboxplotOverallImage}
			}{
				\ifthenelse{\boolean{isFile1}}{
					\ownClearPageSub
					\paragraph{Sataturation bin 3 (25-38) }~
					\loadTex{side_vis_hsv_s_histogram_bin_03_25_38nboxplotOverallImage}
				}{}
			}
			
			\resetBoolean
			\checkFileNoReset{side_vis_hsv_s_histogram_bin_04_38_51nboxplotOverallImage.tex}{}{}
			\checkFileNoReset{}{side_vis_hsv_s_normalized_histogram_bin_04_38_51nboxplotOverallImage.tex}{}
			\ifthenelse{\boolean{isFile2}}{
	
				\ownClearPageSub
				\paragraph{Sataturation bin 4 (38-51) (zoom corrected)}~
				\loadTex{side_vis_hsv_s_normalized_histogram_bin_04_38_51nboxplotOverallImage}
			}{
				\ifthenelse{\boolean{isFile1}}{
					\ownClearPageSub
					\paragraph{Sataturation bin 4 (38-51) }~
					\loadTex{side_vis_hsv_s_histogram_bin_04_38_51nboxplotOverallImage}
				}{}
			}
			
			\resetBoolean
			\checkFileNoReset{side_vis_hsv_s_histogram_bin_05_51_63nboxplotOverallImage.tex}{}{}
			\checkFileNoReset{}{side_vis_hsv_s_normalized_histogram_bin_05_51_63nboxplotOverallImage.tex}{}
			\ifthenelse{\boolean{isFile2}}{
	
				\ownClearPageSub
				\paragraph{Sataturation bin 5 (51-63) (zoom corrected)}~
				\loadTex{side_vis_hsv_s_normalized_histogram_bin_05_51_63nboxplotOverallImage}
			}{
				\ifthenelse{\boolean{isFile1}}{
					\ownClearPageSub
					\paragraph{Sataturation bin 5 (51-63) }~
					\loadTex{side_vis_hsv_s_histogram_bin_05_51_63nboxplotOverallImage}
				}{}
			}
			
			\resetBoolean
			\checkFileNoReset{side_vis_hsv_s_histogram_bin_06_63_76nboxplotOverallImage.tex}{}{}
			\checkFileNoReset{}{side_vis_hsv_s_normalized_histogram_bin_06_63_76nboxplotOverallImage.tex}{}
			\ifthenelse{\boolean{isFile2}}{
	
				\ownClearPageSub
				\paragraph{Sataturation bin 6 (63-76) (zoom corrected)}~
				\loadTex{side_vis_hsv_s_normalized_histogram_bin_06_63_76nboxplotOverallImage}
			}{
				\ifthenelse{\boolean{isFile1}}{
					\ownClearPageSub
					\paragraph{Sataturation bin 6 (63-76) }~
					\loadTex{side_vis_hsv_s_histogram_bin_06_63_76nboxplotOverallImage}
				}{}
			}
		
			\resetBoolean
			\checkFileNoReset{side_vis_hsv_s_histogram_bin_stackedOverallImage.tex}{}{}
			\checkFileNoReset{}{side_vis_hsv_s_normalized_histogram_bin_stackedOverallImage.tex}{}
			\ifthenelse{\boolean{isFile2}}{
	
				\ownClearPageSub
				\paragraph{Sataturation histogram (zoom corrected)}~
				\loadTex{side_vis_hsv_s_normalized_histogram_bin_stackedOverallImage}
			}{
				\ifthenelse{\boolean{isFile1}}{
					\ownClearPageSub
					\paragraph{Sataturation histogram}~
					\loadTex{side_vis_hsv_s_histogram_bin_stackedOverallImage}
				}{}
			}
	  			
		}

		\resetBoolean
		\checkFileNoReset{side_vis_hsv_v_averagenboxplotOverallImage.tex}{}{}
		\checkFileNoReset{side_vis_hsv_v_histogram_bin_stackedOverallImage.tex}{}{}
		\checkFileNoReset{side_vis_hsv_v_kurtosisnboxplotOverallImage.tex}{}{}
		\checkFileNoReset{side_vis_hsv_v_skewnessnboxplotOverallImage.tex}{}{}
		\checkFileNoReset{side_vis_hsv_v_stddevnboxplotOverallImage.tex}{}{}
		\checkFileNoReset{side_vis_hsv_v_histogram_bin_01_0_12nboxplotOverallImage.tex}{}{}
		\checkFileNoReset{side_vis_hsv_v_histogram_bin_02_12_25nboxplotOverallImage.tex}{}{}
		\checkFileNoReset{side_vis_hsv_v_histogram_bin_03_25_38nboxplotOverallImage.tex}{}{}
		\checkFileNoReset{side_vis_hsv_v_histogram_bin_04_38_51nboxplotOverallImage.tex}{}{}
		\checkFileNoReset{side_vis_hsv_v_histogram_bin_05_51_63nboxplotOverallImage.tex}{}{}
		\checkFileNoReset{side_vis_hsv_v_histogram_bin_06_63_76nboxplotOverallImage.tex}{}{}
		\checkFileNoReset{side_vis_hsv_v_normalized_averagenboxplotOverallImage.tex}{}{}
		\checkFileNoReset{side_vis_hsv_v_normalized_histogram_bin_stackedOverallImage.tex}{}{}
		\checkFileNoReset{side_vis_hsv_v_normalized_kurtosisnboxplotOverallImage.tex}{}{}
		\checkFileNoReset{side_vis_hsv_v_normalized_skewnessnboxplotOverallImage.tex}{}{}
		\checkFileNoReset{side_vis_hsv_v_normalized_stddevnboxplotOverallImage.tex}{}{}
		\checkFileNoReset{side_vis_hsv_v_normalized_histogram_bin_01_0_12nboxplotOverallImage.tex}{}{}
		\checkFileNoReset{side_vis_hsv_v_normalized_histogram_bin_02_12_25nboxplotOverallImage.tex}{}{}
		\checkFileNoReset{side_vis_hsv_v_normalized_histogram_bin_03_25_38nboxplotOverallImage.tex}{}{}
		\checkFileNoReset{side_vis_hsv_v_normalized_histogram_bin_04_38_51nboxplotOverallImage.tex}{}{}
		\checkFileNoReset{side_vis_hsv_v_normalized_histogram_bin_05_51_63nboxplotOverallImage.tex}{}{}
		\checkFileNoReset{side_vis_hsv_v_normalized_histogram_bin_06_63_76nboxplotOverallImage.tex}{}{}
		\checkFileNoReset{side_vis_hsv_v_normalized_histogram_bin_stackedOverallImage.tex}{}{}
		\checkFileNoReset{side_vis_hsv_v_histogram_bin_stackedOverallImage.tex}{}{}
		\ifthenelse{\boolean{isFile1}}{
			
			\resetClearSub
			\ownClearPage
	  		\subsubsection{Brigthness}
	
			\resetBoolean
			\checkFileNoReset{side_vis_hsv_v_averagenboxplotOverallImage.tex}{}{}
			\checkFileNoReset{}{side_vis_hsv_v_normalized_averagenboxplotOverallImage.tex}{}
			\ifthenelse{\boolean{isFile2}}{
	
				\ownClearPageSub
				\paragraph{Average brigthness (zoom corrected)}~
% 				\begin{itemize}
% 				\item Area which is enclosed of the convex hull (zoom corrected)
% 				\item Unit: px
% 				\end{itemize}
				\loadTex{side_vis_hsv_v_normalized_averagenboxplotOverallImage}
			}{
				\ifthenelse{\boolean{isFile1}}{
					\ownClearPageSub
					\paragraph{Average brigthness}~
% 					\begin{itemize}
% 					\item Area which is enclosed of the convex hull
% 					\item Unit: px
% 					\end{itemize}
					\loadTex{side_vis_hsv_v_averagenboxplotOverallImage}
				}{}
			}
	
			\resetBoolean
			\checkFileNoReset{side_vis_hsv_v_stddevnboxplotOverallImage.tex}{}{}
			\checkFileNoReset{}{side_vis_hsv_v_normalized_stddevnboxplotOverallImage.tex}{}
			\ifthenelse{\boolean{isFile2}}{
	
				\ownClearPageSub
				\paragraph{Standard devition (zoom corrected)}~
				\loadTex{side_vis_hsv_v_normalized_stddevnboxplotOverallImage}
			}{
				\ifthenelse{\boolean{isFile1}}{
					\ownClearPageSub
					\paragraph{Standard devition}~
					\loadTex{side_vis_hsv_v_stddevnboxplotOverallImage}
				}{}
			}
	
			\resetBoolean
			\checkFileNoReset{side_vis_hsv_v_skewnessnboxplotOverallImage.tex}{}{}
			\checkFileNoReset{}{side_vis_hsv_v_normalized_skewnessnboxplotOverallImage.tex}{}
			\ifthenelse{\boolean{isFile2}}{
	
				\ownClearPageSub
				\paragraph{Skewness (zoom corrected)}~
				\newline
				The skewness can be described as following:  

				“In probability theory and statistics, skewness is a measure of the
				asymmetry of the probability distributionof a real-valued random variable.
				The skewness value can be positive or negative, or even undefined.
				Qualitatively, a negative skew indicates that the tail on the left side of
				the probability density function islonger than the right side and the bulk
				of the values (possibly including the median) lie to the right of the mean.
				A positive skew indicates that the tail on the right side is longer than the
				left side and the bulk of the values lie to the left of the mean. A zero
				value indicates that the values are relatively evenly distributed on both
				sides of the mean, typically but not necessarily implying a symmetric
				distribution.”
				
				\begin{tiny}
			 	Wikipedia contributors, "Skewness",
			 	Wikipedia, The Free Encyclopedia,
				\url{http://en.wikipedia.org/w/index.php?title=Skewness&oldid=499258725}
				(accessed June 27, 2012).
				\end{tiny}
				
				\loadTex{side_vis_hsv_v_normalized_skewnessnboxplotOverallImage}
			}{
				\ifthenelse{\boolean{isFile1}}{
					\ownClearPageSub
					\paragraph{Skewness}~
					\newline
					The skewness can be described as following:  

					“In probability theory and statistics, skewness is a measure of the
					asymmetry of the probability distributionof a real-valued random variable.
					The skewness value can be positive or negative, or even undefined.
					Qualitatively, a negative skew indicates that the tail on the left side of
					the probability density function islonger than the right side and the bulk
					of the values (possibly including the median) lie to the right of the mean.
					A positive skew indicates that the tail on the right side is longer than the
					left side and the bulk of the values lie to the left of the mean. A zero
					value indicates that the values are relatively evenly distributed on both
					sides of the mean, typically but not necessarily implying a symmetric
					distribution.”
					
					\begin{tiny}
				 	Wikipedia contributors, "Skewness",
				 	Wikipedia, The Free Encyclopedia,
					\url{http://en.wikipedia.org/w/index.php?title=Skewness&oldid=499258725}
					(accessed June 27, 2012).
					\end{tiny}
					
					\loadTex{side_vis_hsv_v_skewnessnboxplotOverallImage}
				}{}
			}
			
			\resetBoolean
			\checkFileNoReset{side_vis_hsv_v_kurtosisnboxplotOverallImage.tex}{}{}
			\checkFileNoReset{}{side_vis_hsv_v_normalized_kurtosisnboxplotOverallImage.tex}{}
			\ifthenelse{\boolean{isFile2}}{
	
				\ownClearPageSub
				\paragraph{Skewness (zoom corrected)}~
				\newline
				The term kurtosis is described as following:

				“In probability theory and statistics, kurtosis (from the Greek word κυρτός,
				kyrtos or kurtos, meaning bulging) is any measure of the "peakedness" of the
				probability distribution of a real-valued random variable. In a similar
				way to the concept of skewness, kurtosis is a descriptor of the shape of a
				probability distribution and, just as for skewness, there are different ways
				of quantifying it for a theoretical distribution and corresponding ways of
				estimating it from a sample from a population.

				One common measure of kurtosis, originating with Karl Pearson, is based on a
				scaled version of the fourth moment of the data or population, but it has
				been argued that this measure really measures heavy tails, and not
				peakedness. For this measure, higher kurtosis means more of the variance
				is the result of infrequent extreme deviations, as opposed to frequent
				modestly sized deviations.”

				\begin{tiny}
			 	Wikipedia contributors, "Kurtosis",
			 	Wikipedia, The Free Encyclopedia,
				\url{http://en.wikipedia.org/w/index.php?title=Kurtosis&oldid=496203029}
				(accessed June 27, 2012).
				\end{tiny}
				
				\loadTex{side_vis_hsv_v_normalized_kurtosisnboxplotOverallImage}
			}{
				\ifthenelse{\boolean{isFile1}}{
					\ownClearPageSub
					\paragraph{Skewness}~
					\newline
					The term kurtosis is described as following:

					“In probability theory and statistics, kurtosis (from the Greek word κυρτός,
					kyrtos or kurtos, meaning bulging) is any measure of the "peakedness" of the
					probability distribution of a real-valued random variable. In a similar
					way to the concept of skewness, kurtosis is a descriptor of the shape of a
					probability distribution and, just as for skewness, there are different ways
					of quantifying it for a theoretical distribution and corresponding ways of
					estimating it from a sample from a population.
	
					One common measure of kurtosis, originating with Karl Pearson, is based on a
					scaled version of the fourth moment of the data or population, but it has
					been argued that this measure really measures heavy tails, and not
					peakedness. For this measure, higher kurtosis means more of the variance
					is the result of infrequent extreme deviations, as opposed to frequent
					modestly sized deviations.”
	
					\begin{tiny}
				 	Wikipedia contributors, "Kurtosis",
				 	Wikipedia, The Free Encyclopedia,
					\url{http://en.wikipedia.org/w/index.php?title=Kurtosis&oldid=496203029}
					(accessed June 27, 2012).
					\end{tiny}
					
					\loadTex{side_vis_hsv_v_kurtosisnboxplotOverallImage}
				}{}
			}
			
			\resetBoolean
			\checkFileNoReset{side_vis_hsv_v_histogram_bin_01_0_12nboxplotOverallImage.tex}{}{}
			\checkFileNoReset{}{side_vis_hsv_v_normalized_histogram_bin_01_0_12nboxplotOverallImage.tex}{}
			\ifthenelse{\boolean{isFile2}}{
	
				\ownClearPageSub
				\paragraph{Brigthness bin 1 (01-12) (zoom corrected)}~
				\loadTex{side_vis_hsv_v_normalized_histogram_bin_01_0_12nboxplotOverallImage}
			}{
				\ifthenelse{\boolean{isFile1}}{
					\ownClearPageSub
					\paragraph{Brigthness bin 1 (01-12) }~
					\loadTex{side_vis_hsv_v_histogram_bin_01_0_12nboxplotOverallImage}
				}{}
			}
			
			\resetBoolean
			\checkFileNoReset{side_vis_hsv_v_histogram_bin_02_12_25nboxplotOverallImage.tex}{}{}
			\checkFileNoReset{}{side_vis_hsv_v_normalized_histogram_bin_02_12_25nboxplotOverallImage.tex}{}
			\ifthenelse{\boolean{isFile2}}{
	
				\ownClearPageSub
				\paragraph{Brigthness bin 2 (12-25) (zoom corrected)}~
				\loadTex{side_vis_hsv_v_normalized_histogram_bin_02_12_25nboxplotOverallImage}
			}{
				\ifthenelse{\boolean{isFile1}}{
					\ownClearPageSub
					\paragraph{Brigthness bin 2 (12-25) }~
					\loadTex{side_vis_hsv_v_histogram_bin_02_12_25nboxplotOverallImage}
				}{}
			}
			
			\resetBoolean
			\checkFileNoReset{side_vis_hsv_v_histogram_bin_03_25_38nboxplotOverallImage.tex}{}{}
			\checkFileNoReset{}{side_vis_hsv_v_normalized_histogram_bin_03_25_38nboxplotOverallImage.tex}{}
			\ifthenelse{\boolean{isFile2}}{
	
				\ownClearPageSub
				\paragraph{Brigthness bin 3 (25-38) (zoom corrected)}~
				\loadTex{side_vis_hsv_v_normalized_histogram_bin_03_25_38nboxplotOverallImage}
			}{
				\ifthenelse{\boolean{isFile1}}{
					\ownClearPageSub
					\paragraph{Brigthness bin 3 (25-38) }~
					\loadTex{side_vis_hsv_v_histogram_bin_03_25_38nboxplotOverallImage}
				}{}
			}
			
			\resetBoolean
			\checkFileNoReset{side_vis_hsv_v_histogram_bin_04_38_51nboxplotOverallImage.tex}{}{}
			\checkFileNoReset{}{side_vis_hsv_v_normalized_histogram_bin_04_38_51nboxplotOverallImage.tex}{}
			\ifthenelse{\boolean{isFile2}}{
	
				\ownClearPageSub
				\paragraph{Brigthness bin 4 (38-51) (zoom corrected)}~
				\loadTex{side_vis_hsv_v_normalized_histogram_bin_04_38_51nboxplotOverallImage}
			}{
				\ifthenelse{\boolean{isFile1}}{
					\ownClearPageSub
					\paragraph{Brigthness bin 4 (38-51) }~
					\loadTex{side_vis_hsv_v_histogram_bin_04_38_51nboxplotOverallImage}
				}{}
			}
			
			\resetBoolean
			\checkFileNoReset{side_vis_hsv_v_histogram_bin_05_51_63nboxplotOverallImage.tex}{}{}
			\checkFileNoReset{}{side_vis_hsv_v_normalized_histogram_bin_05_51_63nboxplotOverallImage.tex}{}
			\ifthenelse{\boolean{isFile2}}{
	
				\ownClearPageSub
				\paragraph{Brigthness bin 5 (51-63) (zoom corrected)}~
				\loadTex{side_vis_hsv_v_normalized_histogram_bin_05_51_63nboxplotOverallImage}
			}{
				\ifthenelse{\boolean{isFile1}}{
					\ownClearPageSub
					\paragraph{Brigthness bin 5 (51-63) }~
					\loadTex{side_vis_hsv_v_histogram_bin_05_51_63nboxplotOverallImage}
				}{}
			}
			
			\resetBoolean
			\checkFileNoReset{side_vis_hsv_v_histogram_bin_06_63_76nboxplotOverallImage.tex}{}{}
			\checkFileNoReset{}{side_vis_hsv_v_normalized_histogram_bin_06_63_76nboxplotOverallImage.tex}{}
			\ifthenelse{\boolean{isFile2}}{
	
				\ownClearPageSub
				\paragraph{Brigthness bin 6 (63-76) (zoom corrected)}~
				\loadTex{side_vis_hsv_v_normalized_histogram_bin_06_63_76nboxplotOverallImage}
			}{
				\ifthenelse{\boolean{isFile1}}{
					\ownClearPageSub
					\paragraph{Brigthness bin 6 (63-76) }~
					\loadTex{side_vis_hsv_v_histogram_bin_06_63_76nboxplotOverallImage}
				}{}
			}
		
			\resetBoolean
			\checkFileNoReset{side_vis_hsv_v_histogram_bin_stackedOverallImage.tex}{}{}
			\checkFileNoReset{}{side_vis_hsv_v_normalized_histogram_bin_stackedOverallImage.tex}{}
			\ifthenelse{\boolean{isFile2}}{
	
				\ownClearPageSub
				\paragraph{Brigthness histogram (zoom corrected)}~
				\loadTex{side_vis_hsv_v_normalized_histogram_bin_stackedOverallImage}
			}{
				\ifthenelse{\boolean{isFile1}}{
					\ownClearPageSub
					\paragraph{Brigthness histogram}~
					\loadTex{side_vis_hsv_v_histogram_bin_stackedOverallImage}
				}{}
			}		
	  		
		}
	}
		
	\resetBoolean
	\checkFileNoReset{top_vis_hsv_h_averagenboxplotOverallImage.tex}{}{}
	\checkFileNoReset{top_vis_hsv_h_histogram_bin_stackedOverallImage.tex}{}{}
	\checkFileNoReset{top_vis_hsv_h_kurtosisnboxplotOverallImage.tex}{}{}
	\checkFileNoReset{top_vis_hsv_h_skewnessnboxplotOverallImage.tex}{}{}
	\checkFileNoReset{top_vis_hsv_h_stddevnboxplotOverallImage.tex}{}{}
	\checkFileNoReset{top_vis_hsv_h_histogram_bin_01_0_12nboxplotOverallImage.tex}{}{}
	\checkFileNoReset{top_vis_hsv_h_histogram_bin_02_12_25nboxplotOverallImage.tex}{}{}
	\checkFileNoReset{top_vis_hsv_h_histogram_bin_03_25_38nboxplotOverallImage.tex}{}{}
	\checkFileNoReset{top_vis_hsv_h_histogram_bin_04_38_51nboxplotOverallImage.tex}{}{}
	\checkFileNoReset{top_vis_hsv_h_histogram_bin_05_51_63nboxplotOverallImage.tex}{}{}
	\checkFileNoReset{top_vis_hsv_h_histogram_bin_06_63_76nboxplotOverallImage.tex}{}{}
	\checkFileNoReset{top_vis_hsv_h_normalized_averagenboxplotOverallImage.tex}{}{}
	\checkFileNoReset{top_vis_hsv_h_normalized_histogram_bin_stackedOverallImage.tex}{}{}
	\checkFileNoReset{top_vis_hsv_h_normalized_kurtosisnboxplotOverallImage.tex}{}{}
	\checkFileNoReset{top_vis_hsv_h_normalized_skewnessnboxplotOverallImage.tex}{}{}
	\checkFileNoReset{top_vis_hsv_h_normalized_stddevnboxplotOverallImage.tex}{}{}
	\checkFileNoReset{top_vis_hsv_h_normalized_histogram_bin_01_0_12nboxplotOverallImage.tex}{}{}
	\checkFileNoReset{top_vis_hsv_h_normalized_histogram_bin_02_12_25nboxplotOverallImage.tex}{}{}
	\checkFileNoReset{top_vis_hsv_h_normalized_histogram_bin_03_25_38nboxplotOverallImage.tex}{}{}
	\checkFileNoReset{top_vis_hsv_h_normalized_histogram_bin_04_38_51nboxplotOverallImage.tex}{}{}
	\checkFileNoReset{top_vis_hsv_h_normalized_histogram_bin_05_51_63nboxplotOverallImage.tex}{}{}
	\checkFileNoReset{top_vis_hsv_h_normalized_histogram_bin_06_63_76nboxplotOverallImage.tex}{}{}
	\checkFileNoReset{top_vis_hsv_s_averagenboxplotOverallImage.tex}{}{}
	\checkFileNoReset{top_vis_hsv_s_histogram_bin_stackedOverallImage.tex}{}{}
	\checkFileNoReset{top_vis_hsv_s_kurtosisnboxplotOverallImage.tex}{}{}
	\checkFileNoReset{top_vis_hsv_s_skewnessnboxplotOverallImage.tex}{}{}
	\checkFileNoReset{top_vis_hsv_s_stddevnboxplotOverallImage.tex}{}{}
	\checkFileNoReset{top_vis_hsv_s_histogram_bin_01_0_12nboxplotOverallImage.tex}{}{}
	\checkFileNoReset{top_vis_hsv_s_histogram_bin_02_12_25nboxplotOverallImage.tex}{}{}
	\checkFileNoReset{top_vis_hsv_s_histogram_bin_03_25_38nboxplotOverallImage.tex}{}{}
	\checkFileNoReset{top_vis_hsv_s_histogram_bin_04_38_51nboxplotOverallImage.tex}{}{}
	\checkFileNoReset{top_vis_hsv_s_histogram_bin_05_51_63nboxplotOverallImage.tex}{}{}
	\checkFileNoReset{top_vis_hsv_s_histogram_bin_06_63_76nboxplotOverallImage.tex}{}{}
	\checkFileNoReset{top_vis_hsv_s_normalized_averagenboxplotOverallImage.tex}{}{}
	\checkFileNoReset{top_vis_hsv_s_normalized_histogram_bin_stackedOverallImage.tex}{}{}
	\checkFileNoReset{top_vis_hsv_s_normalized_kurtosisnboxplotOverallImage.tex}{}{}
	\checkFileNoReset{top_vis_hsv_s_normalized_skewnessnboxplotOverallImage.tex}{}{}
	\checkFileNoReset{top_vis_hsv_s_normalized_stddevnboxplotOverallImage.tex}{}{}
	\checkFileNoReset{top_vis_hsv_s_normalized_histogram_bin_01_0_12nboxplotOverallImage.tex}{}{}
	\checkFileNoReset{top_vis_hsv_s_normalized_histogram_bin_02_12_25nboxplotOverallImage.tex}{}{}
	\checkFileNoReset{top_vis_hsv_s_normalized_histogram_bin_03_25_38nboxplotOverallImage.tex}{}{}
	\checkFileNoReset{top_vis_hsv_s_normalized_histogram_bin_04_38_51nboxplotOverallImage.tex}{}{}
	\checkFileNoReset{top_vis_hsv_s_normalized_histogram_bin_05_51_63nboxplotOverallImage.tex}{}{}
	\checkFileNoReset{top_vis_hsv_s_normalized_histogram_bin_06_63_76nboxplotOverallImage.tex}{}{}
	\checkFileNoReset{top_vis_hsv_v_averagenboxplotOverallImage.tex}{}{}
	\checkFileNoReset{top_vis_hsv_v_histogram_bin_stackedOverallImage.tex}{}{}
	\checkFileNoReset{top_vis_hsv_v_kurtosisnboxplotOverallImage.tex}{}{}
	\checkFileNoReset{top_vis_hsv_v_skewnessnboxplotOverallImage.tex}{}{}
	\checkFileNoReset{top_vis_hsv_v_stddevnboxplotOverallImage.tex}{}{}
	\checkFileNoReset{top_vis_hsv_v_histogram_bin_01_0_12nboxplotOverallImage.tex}{}{}
	\checkFileNoReset{top_vis_hsv_v_histogram_bin_02_12_25nboxplotOverallImage.tex}{}{}
	\checkFileNoReset{top_vis_hsv_v_histogram_bin_03_25_38nboxplotOverallImage.tex}{}{}
	\checkFileNoReset{top_vis_hsv_v_histogram_bin_04_38_51nboxplotOverallImage.tex}{}{}
	\checkFileNoReset{top_vis_hsv_v_histogram_bin_05_51_63nboxplotOverallImage.tex}{}{}
	\checkFileNoReset{top_vis_hsv_v_histogram_bin_06_63_76nboxplotOverallImage.tex}{}{}
	\checkFileNoReset{top_vis_hsv_v_normalized_averagenboxplotOverallImage.tex}{}{}
	\checkFileNoReset{top_vis_hsv_v_normalized_histogram_bin_stackedOverallImage.tex}{}{}
	\checkFileNoReset{top_vis_hsv_v_normalized_kurtosisnboxplotOverallImage.tex}{}{}
	\checkFileNoReset{top_vis_hsv_v_normalized_skewnessnboxplotOverallImage.tex}{}{}
	\checkFileNoReset{top_vis_hsv_v_normalized_stddevnboxplotOverallImage.tex}{}{}
	\checkFileNoReset{top_vis_hsv_v_normalized_histogram_bin_01_0_12nboxplotOverallImage.tex}{}{}
	\checkFileNoReset{top_vis_hsv_v_normalized_histogram_bin_02_12_25nboxplotOverallImage.tex}{}{}
	\checkFileNoReset{top_vis_hsv_v_normalized_histogram_bin_03_25_38nboxplotOverallImage.tex}{}{}
	\checkFileNoReset{top_vis_hsv_v_normalized_histogram_bin_04_38_51nboxplotOverallImage.tex}{}{}
	\checkFileNoReset{top_vis_hsv_v_normalized_histogram_bin_05_51_63nboxplotOverallImage.tex}{}{}
	\checkFileNoReset{top_vis_hsv_v_normalized_histogram_bin_06_63_76nboxplotOverallImage.tex}{}{}
	\checkFileNoReset{top_vis_hsv_h_normalized_histogram_bin_stackedOverallImage.tex}{}{}
	\checkFileNoReset{top_vis_hsv_h_histogram_bin_stackedOverallImage.tex}{}{}
	\checkFileNoReset{top_vis_hsv_s_normalized_histogram_bin_stackedOverallImage.tex}{}{}
	\checkFileNoReset{top_vis_hsv_s_histogram_bin_stackedOverallImage.tex}{}{}
	\checkFileNoReset{top_vis_hsv_v_normalized_histogram_bin_stackedOverallImage.tex}{}{}
	\checkFileNoReset{top_vis_hsv_v_histogram_bin_stackedOverallImage.tex}{}{}
	\ifthenelse{\boolean{isFile1}}{
		
		\resetClearSub
		\ownClearPage
  		\subsection{Top}
	
		\resetBoolean
		\checkFileNoReset{top_vis_hsv_h_averagenboxplotOverallImage.tex}{}{}
		\checkFileNoReset{top_vis_hsv_h_histogram_bin_stackedOverallImage.tex}{}{}
		\checkFileNoReset{top_vis_hsv_h_kurtosisnboxplotOverallImage.tex}{}{}
		\checkFileNoReset{top_vis_hsv_h_skewnessnboxplotOverallImage.tex}{}{}
		\checkFileNoReset{top_vis_hsv_h_stddevnboxplotOverallImage.tex}{}{}
		\checkFileNoReset{top_vis_hsv_h_histogram_bin_01_0_12nboxplotOverallImage.tex}{}{}
		\checkFileNoReset{top_vis_hsv_h_histogram_bin_02_12_25nboxplotOverallImage.tex}{}{}
		\checkFileNoReset{top_vis_hsv_h_histogram_bin_03_25_38nboxplotOverallImage.tex}{}{}
		\checkFileNoReset{top_vis_hsv_h_histogram_bin_04_38_51nboxplotOverallImage.tex}{}{}
		\checkFileNoReset{top_vis_hsv_h_histogram_bin_05_51_63nboxplotOverallImage.tex}{}{}
		\checkFileNoReset{top_vis_hsv_h_histogram_bin_06_63_76nboxplotOverallImage.tex}{}{}
		\checkFileNoReset{top_vis_hsv_h_normalized_averagenboxplotOverallImage.tex}{}{}
		\checkFileNoReset{top_vis_hsv_h_normalized_histogram_bin_stackedOverallImage.tex}{}{}
		\checkFileNoReset{top_vis_hsv_h_normalized_kurtosisnboxplotOverallImage.tex}{}{}
		\checkFileNoReset{top_vis_hsv_h_normalized_skewnessnboxplotOverallImage.tex}{}{}
		\checkFileNoReset{top_vis_hsv_h_normalized_stddevnboxplotOverallImage.tex}{}{}
		\checkFileNoReset{top_vis_hsv_h_normalized_histogram_bin_01_0_12nboxplotOverallImage.tex}{}{}
		\checkFileNoReset{top_vis_hsv_h_normalized_histogram_bin_02_12_25nboxplotOverallImage.tex}{}{}
		\checkFileNoReset{top_vis_hsv_h_normalized_histogram_bin_03_25_38nboxplotOverallImage.tex}{}{}
		\checkFileNoReset{top_vis_hsv_h_normalized_histogram_bin_04_38_51nboxplotOverallImage.tex}{}{}
		\checkFileNoReset{top_vis_hsv_h_normalized_histogram_bin_05_51_63nboxplotOverallImage.tex}{}{}
		\checkFileNoReset{top_vis_hsv_h_normalized_histogram_bin_06_63_76nboxplotOverallImage.tex}{}{}
		\checkFileNoReset{top_vis_hsv_h_normalized_histogram_bin_stackedOverallImage.tex}{}{}
		\checkFileNoReset{top_vis_hsv_h_histogram_bin_stackedOverallImage.tex}{}{}
		\ifthenelse{\boolean{isFile1}}{
			
% 			\resetClearSub
% 			\ownClearPage
	  		\subsubsection{Color shade}
	
			\resetBoolean
			\checkFileNoReset{top_vis_hsv_h_averagenboxplotOverallImage.tex}{}{}
			\checkFileNoReset{}{top_vis_hsv_h_normalized_averagenboxplotOverallImage.tex}{}
			\ifthenelse{\boolean{isFile2}}{
	
				\ownClearPageSub
				\paragraph{Average hue (zoom corrected)}~
% 				\begin{itemize}
% 				\item Area which is enclosed of the convex hull (zoom corrected)
% 				\item Unit: px
% 				\end{itemize}
				\loadTex{top_vis_hsv_h_normalized_averagenboxplotOverallImage}
			}{
				\ifthenelse{\boolean{isFile1}}{
					\ownClearPageSub
					\paragraph{Average hue}~
% 					\begin{itemize}
% 					\item Area which is enclosed of the convex hull
% 					\item Unit: px
% 					\end{itemize}
					\loadTex{top_vis_hsv_h_averagenboxplotOverallImage}
				}{}
			}
	
			\resetBoolean
			\checkFileNoReset{top_vis_hsv_h_stddevnboxplotOverallImage.tex}{}{}
			\checkFileNoReset{}{top_vis_hsv_h_normalized_stddevnboxplotOverallImage.tex}{}
			\ifthenelse{\boolean{isFile2}}{
	
				\ownClearPageSub
				\paragraph{Standard devition (zoom corrected)}~
				\loadTex{top_vis_hsv_h_normalized_stddevnboxplotOverallImage}
			}{
				\ifthenelse{\boolean{isFile1}}{
					\ownClearPageSub
					\paragraph{Standard devition}~
					\loadTex{top_vis_hsv_h_stddevnboxplotOverallImage}
				}{}
			}
	
			\resetBoolean
			\checkFileNoReset{top_vis_hsv_h_skewnessnboxplotOverallImage.tex}{}{}
			\checkFileNoReset{}{top_vis_hsv_h_normalized_skewnessnboxplotOverallImage.tex}{}
			\ifthenelse{\boolean{isFile2}}{
	
				\ownClearPageSub
				\paragraph{Skewness (zoom corrected)}~
				\newline
				The skewness can be described as following:  

				“In probability theory and statistics, skewness is a measure of the
				asymmetry of the probability distributionof a real-valued random variable.
				The skewness value can be positive or negative, or even undefined.
				Qualitatively, a negative skew indicates that the tail on the left side of
				the probability density function islonger than the right side and the bulk
				of the values (possibly including the median) lie to the right of the mean.
				A positive skew indicates that the tail on the right side is longer than the
				left side and the bulk of the values lie to the left of the mean. A zero
				value indicates that the values are relatively evenly distributed on both
				sides of the mean, typically but not necessarily implying a symmetric
				distribution.”
				
				\begin{tiny}
			 	Wikipedia contributors, "Skewness",
			 	Wikipedia, The Free Encyclopedia,
				\url{http://en.wikipedia.org/w/index.php?title=Skewness&oldid=499258725}
				(accessed June 27, 2012).
				\end{tiny}
				
				\loadTex{top_vis_hsv_h_normalized_skewnessnboxplotOverallImage}
			}{
				\ifthenelse{\boolean{isFile1}}{
					\ownClearPageSub
					\paragraph{Skewness}~
					\newline
					The skewness can be described as following:  

					“In probability theory and statistics, skewness is a measure of the
					asymmetry of the probability distributionof a real-valued random variable.
					The skewness value can be positive or negative, or even undefined.
					Qualitatively, a negative skew indicates that the tail on the left side of
					the probability density function islonger than the right side and the bulk
					of the values (possibly including the median) lie to the right of the mean.
					A positive skew indicates that the tail on the right side is longer than the
					left side and the bulk of the values lie to the left of the mean. A zero
					value indicates that the values are relatively evenly distributed on both
					sides of the mean, typically but not necessarily implying a symmetric
					distribution.”
					
					\begin{tiny}
				 	Wikipedia contributors, "Skewness",
				 	Wikipedia, The Free Encyclopedia,
					\url{http://en.wikipedia.org/w/index.php?title=Skewness&oldid=499258725}
					(accessed June 27, 2012).
					\end{tiny}
					
					\loadTex{top_vis_hsv_h_skewnessnboxplotOverallImage}
				}{}
			}
			
			\resetBoolean
			\checkFileNoReset{top_vis_hsv_h_kurtosisnboxplotOverallImage.tex}{}{}
			\checkFileNoReset{}{top_vis_hsv_h_normalized_kurtosisnboxplotOverallImage.tex}{}
			\ifthenelse{\boolean{isFile2}}{
	
				\ownClearPageSub
				\paragraph{Skewness (zoom corrected)}~
				\newline
				The term kurtosis is described as following:

				“In probability theory and statistics, kurtosis (from the Greek word κυρτός,
				kyrtos or kurtos, meaning bulging) is any measure of the "peakedness" of the
				probability distribution of a real-valued random variable. In a similar
				way to the concept of skewness, kurtosis is a descriptor of the shape of a
				probability distribution and, just as for skewness, there are different ways
				of quantifying it for a theoretical distribution and corresponding ways of
				estimating it from a sample from a population.

				One common measure of kurtosis, originating with Karl Pearson, is based on a
				scaled version of the fourth moment of the data or population, but it has
				been argued that this measure really measures heavy tails, and not
				peakedness. For this measure, higher kurtosis means more of the variance
				is the result of infrequent extreme deviations, as opposed to frequent
				modestly sized deviations.”

				\begin{tiny}
			 	Wikipedia contributors, "Kurtosis",
			 	Wikipedia, The Free Encyclopedia,
				\url{http://en.wikipedia.org/w/index.php?title=Kurtosis&oldid=496203029}
				(accessed June 27, 2012).
				\end{tiny}
				
				\loadTex{top_vis_hsv_h_normalized_kurtosisnboxplotOverallImage}
			}{
				\ifthenelse{\boolean{isFile1}}{
					\ownClearPageSub
					\paragraph{Skewness}~
					\newline
					The term kurtosis is described as following:

					“In probability theory and statistics, kurtosis (from the Greek word κυρτός,
					kyrtos or kurtos, meaning bulging) is any measure of the "peakedness" of the
					probability distribution of a real-valued random variable. In a similar
					way to the concept of skewness, kurtosis is a descriptor of the shape of a
					probability distribution and, just as for skewness, there are different ways
					of quantifying it for a theoretical distribution and corresponding ways of
					estimating it from a sample from a population.
	
					One common measure of kurtosis, originating with Karl Pearson, is based on a
					scaled version of the fourth moment of the data or population, but it has
					been argued that this measure really measures heavy tails, and not
					peakedness. For this measure, higher kurtosis means more of the variance
					is the result of infrequent extreme deviations, as opposed to frequent
					modestly sized deviations.”
	
					\begin{tiny}
				 	Wikipedia contributors, "Kurtosis",
				 	Wikipedia, The Free Encyclopedia,
					\url{http://en.wikipedia.org/w/index.php?title=Kurtosis&oldid=496203029}
					(accessed June 27, 2012).
					\end{tiny}
					
					\loadTex{top_vis_hsv_h_kurtosisnboxplotOverallImage}
				}{}
			}
			
			\resetBoolean
			\checkFileNoReset{top_vis_hsv_h_histogram_bin_01_0_12nboxplotOverallImage.tex}{}{}
			\checkFileNoReset{}{top_vis_hsv_h_normalized_histogram_bin_01_0_12nboxplotOverallImage.tex}{}
			\ifthenelse{\boolean{isFile2}}{
	
				\ownClearPageSub
				\paragraph{Hue bin 1 (01-12) (zoom corrected)}~
				\loadTex{top_vis_hsv_h_normalized_histogram_bin_01_0_12nboxplotOverallImage}
			}{
				\ifthenelse{\boolean{isFile1}}{
					\ownClearPageSub
					\paragraph{Hue bin 1 (01-12) }~
					\loadTex{top_vis_hsv_h_histogram_bin_01_0_12nboxplotOverallImage}
				}{}
			}
			
			\resetBoolean
			\checkFileNoReset{top_vis_hsv_h_histogram_bin_02_12_25nboxplotOverallImage.tex}{}{}
			\checkFileNoReset{}{top_vis_hsv_h_normalized_histogram_bin_02_12_25nboxplotOverallImage.tex}{}
			\ifthenelse{\boolean{isFile2}}{
	
				\ownClearPageSub
				\paragraph{Hue bin 2 (12-25) (zoom corrected)}~
				\loadTex{top_vis_hsv_h_normalized_histogram_bin_02_12_25nboxplotOverallImage}
			}{
				\ifthenelse{\boolean{isFile1}}{
					\ownClearPageSub
					\paragraph{Hue bin 2 (12-25) }~
					\loadTex{top_vis_hsv_h_histogram_bin_02_12_25nboxplotOverallImage}
				}{}
			}
			
			\resetBoolean
			\checkFileNoReset{top_vis_hsv_h_histogram_bin_03_25_38nboxplotOverallImage.tex}{}{}
			\checkFileNoReset{}{top_vis_hsv_h_normalized_histogram_bin_03_25_38nboxplotOverallImage.tex}{}
			\ifthenelse{\boolean{isFile2}}{
	
				\ownClearPageSub
				\paragraph{Hue bin 3 (25-38) (zoom corrected)}~
				\loadTex{top_vis_hsv_h_normalized_histogram_bin_03_25_38nboxplotOverallImage}
			}{
				\ifthenelse{\boolean{isFile1}}{
					\ownClearPageSub
					\paragraph{Hue bin 3 (25-38) }~
					\loadTex{top_vis_hsv_h_histogram_bin_03_25_38nboxplotOverallImage}
				}{}
			}
			
			\resetBoolean
			\checkFileNoReset{top_vis_hsv_h_histogram_bin_04_38_51nboxplotOverallImage.tex}{}{}
			\checkFileNoReset{}{top_vis_hsv_h_normalized_histogram_bin_04_38_51nboxplotOverallImage.tex}{}
			\ifthenelse{\boolean{isFile2}}{
	
				\ownClearPageSub
				\paragraph{Hue bin 4 (38-51) (zoom corrected)}~
				\loadTex{top_vis_hsv_h_normalized_histogram_bin_04_38_51nboxplotOverallImage}
			}{
				\ifthenelse{\boolean{isFile1}}{
					\ownClearPageSub
					\paragraph{Hue bin 4 (38-51) }~
					\loadTex{top_vis_hsv_h_histogram_bin_04_38_51nboxplotOverallImage}
				}{}
			}
			
			\resetBoolean
			\checkFileNoReset{top_vis_hsv_h_histogram_bin_05_51_63nboxplotOverallImage.tex}{}{}
			\checkFileNoReset{}{top_vis_hsv_h_normalized_histogram_bin_05_51_63nboxplotOverallImage.tex}{}
			\ifthenelse{\boolean{isFile2}}{
	
				\ownClearPageSub
				\paragraph{Hue bin 5 (51-63) (zoom corrected)}~
				\loadTex{top_vis_hsv_h_normalized_histogram_bin_05_51_63nboxplotOverallImage}
			}{
				\ifthenelse{\boolean{isFile1}}{
					\ownClearPageSub
					\paragraph{Hue bin 5 (51-63) }~
					\loadTex{top_vis_hsv_h_histogram_bin_05_51_63nboxplotOverallImage}
				}{}
			}
			
			\resetBoolean
			\checkFileNoReset{top_vis_hsv_h_histogram_bin_06_63_76nboxplotOverallImage.tex}{}{}
			\checkFileNoReset{}{top_vis_hsv_h_normalized_histogram_bin_06_63_76nboxplotOverallImage.tex}{}
			\ifthenelse{\boolean{isFile2}}{
	
				\ownClearPageSub
				\paragraph{Hue bin 6 (63-76) (zoom corrected)}~
				\loadTex{top_vis_hsv_h_normalized_histogram_bin_06_63_76nboxplotOverallImage}
			}{
				\ifthenelse{\boolean{isFile1}}{
					\ownClearPageSub
					\paragraph{Hue bin 6 (63-76) }~
					\loadTex{top_vis_hsv_h_histogram_bin_06_63_76nboxplotOverallImage}
				}{}
			}
		
			\resetBoolean
			\checkFileNoReset{top_vis_hsv_h_histogram_bin_stackedOverallImage.tex}{}{}
			\checkFileNoReset{}{top_vis_hsv_h_normalized_histogram_bin_stackedOverallImage.tex}{}
			\ifthenelse{\boolean{isFile2}}{
	
				\ownClearPageSub
				\paragraph{Color histogram (zoom corrected)}~
				\loadTex{top_vis_hsv_h_normalized_histogram_bin_stackedOverallImage}
			}{
				\ifthenelse{\boolean{isFile1}}{
					\ownClearPageSub
					\paragraph{Color histogram}~
					\loadTex{top_vis_hsv_h_histogram_bin_stackedOverallImage}
				}{}
			}
		
		

	
		}

		\resetBoolean
		\checkFileNoReset{top_vis_hsv_s_averagenboxplotOverallImage.tex}{}{}
		\checkFileNoReset{top_vis_hsv_s_histogram_bin_stackedOverallImage.tex}{}{}
		\checkFileNoReset{top_vis_hsv_s_kurtosisnboxplotOverallImage.tex}{}{}
		\checkFileNoReset{top_vis_hsv_s_skewnessnboxplotOverallImage.tex}{}{}
		\checkFileNoReset{top_vis_hsv_s_stddevnboxplotOverallImage.tex}{}{}
		\checkFileNoReset{top_vis_hsv_s_histogram_bin_01_0_12nboxplotOverallImage.tex}{}{}
		\checkFileNoReset{top_vis_hsv_s_histogram_bin_02_12_25nboxplotOverallImage.tex}{}{}
		\checkFileNoReset{top_vis_hsv_s_histogram_bin_03_25_38nboxplotOverallImage.tex}{}{}
		\checkFileNoReset{top_vis_hsv_s_histogram_bin_04_38_51nboxplotOverallImage.tex}{}{}
		\checkFileNoReset{top_vis_hsv_s_histogram_bin_05_51_63nboxplotOverallImage.tex}{}{}
		\checkFileNoReset{top_vis_hsv_s_histogram_bin_06_63_76nboxplotOverallImage.tex}{}{}
		\checkFileNoReset{top_vis_hsv_s_normalized_averagenboxplotOverallImage.tex}{}{}
		\checkFileNoReset{top_vis_hsv_s_normalized_histogram_bin_stackedOverallImage.tex}{}{}
		\checkFileNoReset{top_vis_hsv_s_normalized_kurtosisnboxplotOverallImage.tex}{}{}
		\checkFileNoReset{top_vis_hsv_s_normalized_skewnessnboxplotOverallImage.tex}{}{}
		\checkFileNoReset{top_vis_hsv_s_normalized_stddevnboxplotOverallImage.tex}{}{}
		\checkFileNoReset{top_vis_hsv_s_normalized_histogram_bin_01_0_12nboxplotOverallImage.tex}{}{}
		\checkFileNoReset{top_vis_hsv_s_normalized_histogram_bin_02_12_25nboxplotOverallImage.tex}{}{}
		\checkFileNoReset{top_vis_hsv_s_normalized_histogram_bin_03_25_38nboxplotOverallImage.tex}{}{}
		\checkFileNoReset{top_vis_hsv_s_normalized_histogram_bin_04_38_51nboxplotOverallImage.tex}{}{}
		\checkFileNoReset{top_vis_hsv_s_normalized_histogram_bin_05_51_63nboxplotOverallImage.tex}{}{}
		\checkFileNoReset{top_vis_hsv_s_normalized_histogram_bin_06_63_76nboxplotOverallImage.tex}{}{}
		\checkFileNoReset{top_vis_hsv_s_normalized_histogram_bin_stackedOverallImage.tex}{}{}
		\checkFileNoReset{top_vis_hsv_s_histogram_bin_stackedOverallImage.tex}{}{}
		\ifthenelse{\boolean{isFile1}}{
			
			\resetClearSub
			\ownClearPage
	  		\subsubsection{Sataturation}
	
			\resetBoolean
			\checkFileNoReset{top_vis_hsv_s_averagenboxplotOverallImage.tex}{}{}
			\checkFileNoReset{}{top_vis_hsv_s_normalized_averagenboxplotOverallImage.tex}{}
			\ifthenelse{\boolean{isFile2}}{
	
				\ownClearPageSub
				\paragraph{Average sataturation (zoom corrected)}~
% 				\begin{itemize}
% 				\item Area which is enclosed of the convex hull (zoom corrected)
% 				\item Unit: px
% 				\end{itemize}
				\loadTex{top_vis_hsv_s_normalized_averagenboxplotOverallImage}
			}{
				\ifthenelse{\boolean{isFile1}}{
					\ownClearPageSub
					\paragraph{Average sataturation}~
% 					\begin{itemize}
% 					\item Area which is enclosed of the convex hull
% 					\item Unit: px
% 					\end{itemize}
					\loadTex{top_vis_hsv_s_averagenboxplotOverallImage}
				}{}
			}
	
			\resetBoolean
			\checkFileNoReset{top_vis_hsv_s_stddevnboxplotOverallImage.tex}{}{}
			\checkFileNoReset{}{top_vis_hsv_s_normalized_stddevnboxplotOverallImage.tex}{}
			\ifthenelse{\boolean{isFile2}}{
	
				\ownClearPageSub
				\paragraph{Standard devition (zoom corrected)}~
				\loadTex{top_vis_hsv_s_normalized_stddevnboxplotOverallImage}
			}{
				\ifthenelse{\boolean{isFile1}}{
					\ownClearPageSub
					\paragraph{Standard devition}~
					\loadTex{top_vis_hsv_s_stddevnboxplotOverallImage}
				}{}
			}
	
			\resetBoolean
			\checkFileNoReset{top_vis_hsv_s_skewnessnboxplotOverallImage.tex}{}{}
			\checkFileNoReset{}{top_vis_hsv_s_normalized_skewnessnboxplotOverallImage.tex}{}
			\ifthenelse{\boolean{isFile2}}{
	
				\ownClearPageSub
				\paragraph{Skewness (zoom corrected)}~
				\newline
				The skewness can be described as following:  

				“In probability theory and statistics, skewness is a measure of the
				asymmetry of the probability distributionof a real-valued random variable.
				The skewness value can be positive or negative, or even undefined.
				Qualitatively, a negative skew indicates that the tail on the left side of
				the probability density function islonger than the right side and the bulk
				of the values (possibly including the median) lie to the right of the mean.
				A positive skew indicates that the tail on the right side is longer than the
				left side and the bulk of the values lie to the left of the mean. A zero
				value indicates that the values are relatively evenly distributed on both
				sides of the mean, typically but not necessarily implying a symmetric
				distribution.”
				
				\begin{tiny}
			 	Wikipedia contributors, "Skewness",
			 	Wikipedia, The Free Encyclopedia,
				\url{http://en.wikipedia.org/w/index.php?title=Skewness&oldid=499258725}
				(accessed June 27, 2012).
				\end{tiny}
				
				\loadTex{top_vis_hsv_s_normalized_skewnessnboxplotOverallImage}
			}{
				\ifthenelse{\boolean{isFile1}}{
					\ownClearPageSub
					\paragraph{Skewness}~
					\newline
					The skewness can be described as following:  

					“In probability theory and statistics, skewness is a measure of the
					asymmetry of the probability distributionof a real-valued random variable.
					The skewness value can be positive or negative, or even undefined.
					Qualitatively, a negative skew indicates that the tail on the left side of
					the probability density function islonger than the right side and the bulk
					of the values (possibly including the median) lie to the right of the mean.
					A positive skew indicates that the tail on the right side is longer than the
					left side and the bulk of the values lie to the left of the mean. A zero
					value indicates that the values are relatively evenly distributed on both
					sides of the mean, typically but not necessarily implying a symmetric
					distribution.”
					
					\begin{tiny}
				 	Wikipedia contributors, "Skewness",
				 	Wikipedia, The Free Encyclopedia,
					\url{http://en.wikipedia.org/w/index.php?title=Skewness&oldid=499258725}
					(accessed June 27, 2012).
					\end{tiny}
					
					\loadTex{top_vis_hsv_s_skewnessnboxplotOverallImage}
				}{}
			}
			
			\resetBoolean
			\checkFileNoReset{top_vis_hsv_s_kurtosisnboxplotOverallImage.tex}{}{}
			\checkFileNoReset{}{top_vis_hsv_s_normalized_kurtosisnboxplotOverallImage.tex}{}
			\ifthenelse{\boolean{isFile2}}{
	
				\ownClearPageSub
				\paragraph{Skewness (zoom corrected)}~
				\newline
				The term kurtosis is described as following:

				“In probability theory and statistics, kurtosis (from the Greek word κυρτός,
				kyrtos or kurtos, meaning bulging) is any measure of the "peakedness" of the
				probability distribution of a real-valued random variable. In a similar
				way to the concept of skewness, kurtosis is a descriptor of the shape of a
				probability distribution and, just as for skewness, there are different ways
				of quantifying it for a theoretical distribution and corresponding ways of
				estimating it from a sample from a population.

				One common measure of kurtosis, originating with Karl Pearson, is based on a
				scaled version of the fourth moment of the data or population, but it has
				been argued that this measure really measures heavy tails, and not
				peakedness. For this measure, higher kurtosis means more of the variance
				is the result of infrequent extreme deviations, as opposed to frequent
				modestly sized deviations.”

				\begin{tiny}
			 	Wikipedia contributors, "Kurtosis",
			 	Wikipedia, The Free Encyclopedia,
				\url{http://en.wikipedia.org/w/index.php?title=Kurtosis&oldid=496203029}
				(accessed June 27, 2012).
				\end{tiny}
				
				\loadTex{top_vis_hsv_s_normalized_kurtosisnboxplotOverallImage}
			}{
				\ifthenelse{\boolean{isFile1}}{
					\ownClearPageSub
					\paragraph{Skewness}~
					\newline
					The term kurtosis is described as following:

					“In probability theory and statistics, kurtosis (from the Greek word κυρτός,
					kyrtos or kurtos, meaning bulging) is any measure of the "peakedness" of the
					probability distribution of a real-valued random variable. In a similar
					way to the concept of skewness, kurtosis is a descriptor of the shape of a
					probability distribution and, just as for skewness, there are different ways
					of quantifying it for a theoretical distribution and corresponding ways of
					estimating it from a sample from a population.
	
					One common measure of kurtosis, originating with Karl Pearson, is based on a
					scaled version of the fourth moment of the data or population, but it has
					been argued that this measure really measures heavy tails, and not
					peakedness. For this measure, higher kurtosis means more of the variance
					is the result of infrequent extreme deviations, as opposed to frequent
					modestly sized deviations.”
	
					\begin{tiny}
				 	Wikipedia contributors, "Kurtosis",
				 	Wikipedia, The Free Encyclopedia,
					\url{http://en.wikipedia.org/w/index.php?title=Kurtosis&oldid=496203029}
					(accessed June 27, 2012).
					\end{tiny}
					
					\loadTex{top_vis_hsv_s_kurtosisnboxplotOverallImage}
				}{}
			}
			
			\resetBoolean
			\checkFileNoReset{top_vis_hsv_s_histogram_bin_01_0_12nboxplotOverallImage.tex}{}{}
			\checkFileNoReset{}{top_vis_hsv_s_normalized_histogram_bin_01_0_12nboxplotOverallImage.tex}{}
			\ifthenelse{\boolean{isFile2}}{
	
				\ownClearPageSub
				\paragraph{Sataturation bin 1 (01-12) (zoom corrected)}~
				\loadTex{top_vis_hsv_s_normalized_histogram_bin_01_0_12nboxplotOverallImage}
			}{
				\ifthenelse{\boolean{isFile1}}{
					\ownClearPageSub
					\paragraph{Sataturation bin 1 (01-12) }~
					\loadTex{top_vis_hsv_s_histogram_bin_01_0_12nboxplotOverallImage}
				}{}
			}
			
			\resetBoolean
			\checkFileNoReset{top_vis_hsv_s_histogram_bin_02_12_25nboxplotOverallImage.tex}{}{}
			\checkFileNoReset{}{top_vis_hsv_s_normalized_histogram_bin_02_12_25nboxplotOverallImage.tex}{}
			\ifthenelse{\boolean{isFile2}}{
	
				\ownClearPageSub
				\paragraph{Sataturation bin 2 (12-25) (zoom corrected)}~
				\loadTex{top_vis_hsv_s_normalized_histogram_bin_02_12_25nboxplotOverallImage}
			}{
				\ifthenelse{\boolean{isFile1}}{
					\ownClearPageSub
					\paragraph{Sataturation bin 2 (12-25) }~
					\loadTex{top_vis_hsv_s_histogram_bin_02_12_25nboxplotOverallImage}
				}{}
			}
			
			\resetBoolean
			\checkFileNoReset{top_vis_hsv_s_histogram_bin_03_25_38nboxplotOverallImage.tex}{}{}
			\checkFileNoReset{}{top_vis_hsv_s_normalized_histogram_bin_03_25_38nboxplotOverallImage.tex}{}
			\ifthenelse{\boolean{isFile2}}{
	
				\ownClearPageSub
				\paragraph{Sataturation bin 3 (25-38) (zoom corrected)}~
				\loadTex{top_vis_hsv_s_normalized_histogram_bin_03_25_38nboxplotOverallImage}
			}{
				\ifthenelse{\boolean{isFile1}}{
					\ownClearPageSub
					\paragraph{Sataturation bin 3 (25-38) }~
					\loadTex{top_vis_hsv_s_histogram_bin_03_25_38nboxplotOverallImage}
				}{}
			}
			
			\resetBoolean
			\checkFileNoReset{top_vis_hsv_s_histogram_bin_04_38_51nboxplotOverallImage.tex}{}{}
			\checkFileNoReset{}{top_vis_hsv_s_normalized_histogram_bin_04_38_51nboxplotOverallImage.tex}{}
			\ifthenelse{\boolean{isFile2}}{
	
				\ownClearPageSub
				\paragraph{Sataturation bin 4 (38-51) (zoom corrected)}~
				\loadTex{top_vis_hsv_s_normalized_histogram_bin_04_38_51nboxplotOverallImage}
			}{
				\ifthenelse{\boolean{isFile1}}{
					\ownClearPageSub
					\paragraph{Sataturation bin 4 (38-51) }~
					\loadTex{top_vis_hsv_s_histogram_bin_04_38_51nboxplotOverallImage}
				}{}
			}
			
			\resetBoolean
			\checkFileNoReset{top_vis_hsv_s_histogram_bin_05_51_63nboxplotOverallImage.tex}{}{}
			\checkFileNoReset{}{top_vis_hsv_s_normalized_histogram_bin_05_51_63nboxplotOverallImage.tex}{}
			\ifthenelse{\boolean{isFile2}}{
	
				\ownClearPageSub
				\paragraph{Sataturation bin 5 (51-63) (zoom corrected)}~
				\loadTex{top_vis_hsv_s_normalized_histogram_bin_05_51_63nboxplotOverallImage}
			}{
				\ifthenelse{\boolean{isFile1}}{
					\ownClearPageSub
					\paragraph{Sataturation bin 5 (51-63) }~
					\loadTex{top_vis_hsv_s_histogram_bin_05_51_63nboxplotOverallImage}
				}{}
			}
			
			\resetBoolean
			\checkFileNoReset{top_vis_hsv_s_histogram_bin_06_63_76nboxplotOverallImage.tex}{}{}
			\checkFileNoReset{}{top_vis_hsv_s_normalized_histogram_bin_06_63_76nboxplotOverallImage.tex}{}
			\ifthenelse{\boolean{isFile2}}{
	
				\ownClearPageSub
				\paragraph{Sataturation bin 6 (63-76) (zoom corrected)}~
				\loadTex{top_vis_hsv_s_normalized_histogram_bin_06_63_76nboxplotOverallImage}
			}{
				\ifthenelse{\boolean{isFile1}}{
					\ownClearPageSub
					\paragraph{Sataturation bin 6 (63-76) }~
					\loadTex{top_vis_hsv_s_histogram_bin_06_63_76nboxplotOverallImage}
				}{}
			}
		
			\resetBoolean
			\checkFileNoReset{top_vis_hsv_s_histogram_bin_stackedOverallImage.tex}{}{}
			\checkFileNoReset{}{top_vis_hsv_s_normalized_histogram_bin_stackedOverallImage.tex}{}
			\ifthenelse{\boolean{isFile2}}{
	
				\ownClearPageSub
				\paragraph{Sataturation histogram (zoom corrected)}~
				\loadTex{top_vis_hsv_s_normalized_histogram_bin_stackedOverallImage}
			}{
				\ifthenelse{\boolean{isFile1}}{
					\ownClearPageSub
					\paragraph{Sataturation histogram}~
					\loadTex{top_vis_hsv_s_histogram_bin_stackedOverallImage}
				}{}
			}
	  			
		}

		\resetBoolean
		\checkFileNoReset{top_vis_hsv_v_averagenboxplotOverallImage.tex}{}{}
		\checkFileNoReset{top_vis_hsv_v_histogram_bin_stackedOverallImage.tex}{}{}
		\checkFileNoReset{top_vis_hsv_v_kurtosisnboxplotOverallImage.tex}{}{}
		\checkFileNoReset{top_vis_hsv_v_skewnessnboxplotOverallImage.tex}{}{}
		\checkFileNoReset{top_vis_hsv_v_stddevnboxplotOverallImage.tex}{}{}
		\checkFileNoReset{top_vis_hsv_v_histogram_bin_01_0_12nboxplotOverallImage.tex}{}{}
		\checkFileNoReset{top_vis_hsv_v_histogram_bin_02_12_25nboxplotOverallImage.tex}{}{}
		\checkFileNoReset{top_vis_hsv_v_histogram_bin_03_25_38nboxplotOverallImage.tex}{}{}
		\checkFileNoReset{top_vis_hsv_v_histogram_bin_04_38_51nboxplotOverallImage.tex}{}{}
		\checkFileNoReset{top_vis_hsv_v_histogram_bin_05_51_63nboxplotOverallImage.tex}{}{}
		\checkFileNoReset{top_vis_hsv_v_histogram_bin_06_63_76nboxplotOverallImage.tex}{}{}
		\checkFileNoReset{top_vis_hsv_v_normalized_averagenboxplotOverallImage.tex}{}{}
		\checkFileNoReset{top_vis_hsv_v_normalized_histogram_bin_stackedOverallImage.tex}{}{}
		\checkFileNoReset{top_vis_hsv_v_normalized_kurtosisnboxplotOverallImage.tex}{}{}
		\checkFileNoReset{top_vis_hsv_v_normalized_skewnessnboxplotOverallImage.tex}{}{}
		\checkFileNoReset{top_vis_hsv_v_normalized_stddevnboxplotOverallImage.tex}{}{}
		\checkFileNoReset{top_vis_hsv_v_normalized_histogram_bin_01_0_12nboxplotOverallImage.tex}{}{}
		\checkFileNoReset{top_vis_hsv_v_normalized_histogram_bin_02_12_25nboxplotOverallImage.tex}{}{}
		\checkFileNoReset{top_vis_hsv_v_normalized_histogram_bin_03_25_38nboxplotOverallImage.tex}{}{}
		\checkFileNoReset{top_vis_hsv_v_normalized_histogram_bin_04_38_51nboxplotOverallImage.tex}{}{}
		\checkFileNoReset{top_vis_hsv_v_normalized_histogram_bin_05_51_63nboxplotOverallImage.tex}{}{}
		\checkFileNoReset{top_vis_hsv_v_normalized_histogram_bin_06_63_76nboxplotOverallImage.tex}{}{}
		\checkFileNoReset{top_vis_hsv_v_normalized_histogram_bin_stackedOverallImage.tex}{}{}
		\checkFileNoReset{top_vis_hsv_v_histogram_bin_stackedOverallImage.tex}{}{}
		\ifthenelse{\boolean{isFile1}}{
			
			\resetClearSub
			\ownClearPage
	  		\subsubsection{Brigthness}
	
			\resetBoolean
			\checkFileNoReset{top_vis_hsv_v_averagenboxplotOverallImage.tex}{}{}
			\checkFileNoReset{}{top_vis_hsv_v_normalized_averagenboxplotOverallImage.tex}{}
			\ifthenelse{\boolean{isFile2}}{
	
				\ownClearPageSub
				\paragraph{Average brigthness (zoom corrected)}~
% 				\begin{itemize}
% 				\item Area which is enclosed of the convex hull (zoom corrected)
% 				\item Unit: px
% 				\end{itemize}
				\loadTex{top_vis_hsv_v_normalized_averagenboxplotOverallImage}
			}{
				\ifthenelse{\boolean{isFile1}}{
					\ownClearPageSub
					\paragraph{Average brigthness}~
% 					\begin{itemize}
% 					\item Area which is enclosed of the convex hull
% 					\item Unit: px
% 					\end{itemize}
					\loadTex{top_vis_hsv_v_averagenboxplotOverallImage}
				}{}
			}
	
			\resetBoolean
			\checkFileNoReset{top_vis_hsv_v_stddevnboxplotOverallImage.tex}{}{}
			\checkFileNoReset{}{top_vis_hsv_v_normalized_stddevnboxplotOverallImage.tex}{}
			\ifthenelse{\boolean{isFile2}}{
	
				\ownClearPageSub
				\paragraph{Standard devition (zoom corrected)}~
				\loadTex{top_vis_hsv_v_normalized_stddevnboxplotOverallImage}
			}{
				\ifthenelse{\boolean{isFile1}}{
					\ownClearPageSub
					\paragraph{Standard devition}~
					\loadTex{top_vis_hsv_v_stddevnboxplotOverallImage}
				}{}
			}
	
			\resetBoolean
			\checkFileNoReset{top_vis_hsv_v_skewnessnboxplotOverallImage.tex}{}{}
			\checkFileNoReset{}{top_vis_hsv_v_normalized_skewnessnboxplotOverallImage.tex}{}
			\ifthenelse{\boolean{isFile2}}{
	
				\ownClearPageSub
				\paragraph{Skewness (zoom corrected)}~
				\newline
				The skewness can be described as following:  

				“In probability theory and statistics, skewness is a measure of the
				asymmetry of the probability distributionof a real-valued random variable.
				The skewness value can be positive or negative, or even undefined.
				Qualitatively, a negative skew indicates that the tail on the left side of
				the probability density function islonger than the right side and the bulk
				of the values (possibly including the median) lie to the right of the mean.
				A positive skew indicates that the tail on the right side is longer than the
				left side and the bulk of the values lie to the left of the mean. A zero
				value indicates that the values are relatively evenly distributed on both
				sides of the mean, typically but not necessarily implying a symmetric
				distribution.”
				
				\begin{tiny}
			 	Wikipedia contributors, "Skewness",
			 	Wikipedia, The Free Encyclopedia,
				\url{http://en.wikipedia.org/w/index.php?title=Skewness&oldid=499258725}
				(accessed June 27, 2012).
				\end{tiny}
				
				\loadTex{top_vis_hsv_v_normalized_skewnessnboxplotOverallImage}
			}{
				\ifthenelse{\boolean{isFile1}}{
					\ownClearPageSub
					\paragraph{Skewness}~
					\newline
					The skewness can be described as following:  

					“In probability theory and statistics, skewness is a measure of the
					asymmetry of the probability distributionof a real-valued random variable.
					The skewness value can be positive or negative, or even undefined.
					Qualitatively, a negative skew indicates that the tail on the left side of
					the probability density function islonger than the right side and the bulk
					of the values (possibly including the median) lie to the right of the mean.
					A positive skew indicates that the tail on the right side is longer than the
					left side and the bulk of the values lie to the left of the mean. A zero
					value indicates that the values are relatively evenly distributed on both
					sides of the mean, typically but not necessarily implying a symmetric
					distribution.”
					
					\begin{tiny}
				 	Wikipedia contributors, "Skewness",
				 	Wikipedia, The Free Encyclopedia,
					\url{http://en.wikipedia.org/w/index.php?title=Skewness&oldid=499258725}
					(accessed June 27, 2012).
					\end{tiny}
					
					\loadTex{top_vis_hsv_v_skewnessnboxplotOverallImage}
				}{}
			}
			
			\resetBoolean
			\checkFileNoReset{top_vis_hsv_v_kurtosisnboxplotOverallImage.tex}{}{}
			\checkFileNoReset{}{top_vis_hsv_v_normalized_kurtosisnboxplotOverallImage.tex}{}
			\ifthenelse{\boolean{isFile2}}{
	
				\ownClearPageSub
				\paragraph{Skewness (zoom corrected)}~
				\newline
				The term kurtosis is described as following:

				“In probability theory and statistics, kurtosis (from the Greek word κυρτός,
				kyrtos or kurtos, meaning bulging) is any measure of the "peakedness" of the
				probability distribution of a real-valued random variable. In a similar
				way to the concept of skewness, kurtosis is a descriptor of the shape of a
				probability distribution and, just as for skewness, there are different ways
				of quantifying it for a theoretical distribution and corresponding ways of
				estimating it from a sample from a population.

				One common measure of kurtosis, originating with Karl Pearson, is based on a
				scaled version of the fourth moment of the data or population, but it has
				been argued that this measure really measures heavy tails, and not
				peakedness. For this measure, higher kurtosis means more of the variance
				is the result of infrequent extreme deviations, as opposed to frequent
				modestly sized deviations.”

				\begin{tiny}
			 	Wikipedia contributors, "Kurtosis",
			 	Wikipedia, The Free Encyclopedia,
				\url{http://en.wikipedia.org/w/index.php?title=Kurtosis&oldid=496203029}
				(accessed June 27, 2012).
				\end{tiny}
				
				\loadTex{top_vis_hsv_v_normalized_kurtosisnboxplotOverallImage}
			}{
				\ifthenelse{\boolean{isFile1}}{
					\ownClearPageSub
					\paragraph{Skewness}~
					\newline
					The term kurtosis is described as following:

					“In probability theory and statistics, kurtosis (from the Greek word κυρτός,
					kyrtos or kurtos, meaning bulging) is any measure of the "peakedness" of the
					probability distribution of a real-valued random variable. In a similar
					way to the concept of skewness, kurtosis is a descriptor of the shape of a
					probability distribution and, just as for skewness, there are different ways
					of quantifying it for a theoretical distribution and corresponding ways of
					estimating it from a sample from a population.
	
					One common measure of kurtosis, originating with Karl Pearson, is based on a
					scaled version of the fourth moment of the data or population, but it has
					been argued that this measure really measures heavy tails, and not
					peakedness. For this measure, higher kurtosis means more of the variance
					is the result of infrequent extreme deviations, as opposed to frequent
					modestly sized deviations.”
	
					\begin{tiny}
				 	Wikipedia contributors, "Kurtosis",
				 	Wikipedia, The Free Encyclopedia,
					\url{http://en.wikipedia.org/w/index.php?title=Kurtosis&oldid=496203029}
					(accessed June 27, 2012).
					\end{tiny}
					
					\loadTex{top_vis_hsv_v_kurtosisnboxplotOverallImage}
				}{}
			}
			
			\resetBoolean
			\checkFileNoReset{top_vis_hsv_v_histogram_bin_01_0_12nboxplotOverallImage.tex}{}{}
			\checkFileNoReset{}{top_vis_hsv_v_normalized_histogram_bin_01_0_12nboxplotOverallImage.tex}{}
			\ifthenelse{\boolean{isFile2}}{
	
				\ownClearPageSub
				\paragraph{Brigthness bin 1 (01-12) (zoom corrected)}~
				\loadTex{top_vis_hsv_v_normalized_histogram_bin_01_0_12nboxplotOverallImage}
			}{
				\ifthenelse{\boolean{isFile1}}{
					\ownClearPageSub
					\paragraph{Brigthness bin 1 (01-12) }~
					\loadTex{top_vis_hsv_v_histogram_bin_01_0_12nboxplotOverallImage}
				}{}
			}
			
			\resetBoolean
			\checkFileNoReset{top_vis_hsv_v_histogram_bin_02_12_25nboxplotOverallImage.tex}{}{}
			\checkFileNoReset{}{top_vis_hsv_v_normalized_histogram_bin_02_12_25nboxplotOverallImage.tex}{}
			\ifthenelse{\boolean{isFile2}}{
	
				\ownClearPageSub
				\paragraph{Brigthness bin 2 (12-25) (zoom corrected)}~
				\loadTex{top_vis_hsv_v_normalized_histogram_bin_02_12_25nboxplotOverallImage}
			}{
				\ifthenelse{\boolean{isFile1}}{
					\ownClearPageSub
					\paragraph{Brigthness bin 2 (12-25) }~
					\loadTex{top_vis_hsv_v_histogram_bin_02_12_25nboxplotOverallImage}
				}{}
			}
			
			\resetBoolean
			\checkFileNoReset{top_vis_hsv_v_histogram_bin_03_25_38nboxplotOverallImage.tex}{}{}
			\checkFileNoReset{}{top_vis_hsv_v_normalized_histogram_bin_03_25_38nboxplotOverallImage.tex}{}
			\ifthenelse{\boolean{isFile2}}{
	
				\ownClearPageSub
				\paragraph{Brigthness bin 3 (25-38) (zoom corrected)}~
				\loadTex{top_vis_hsv_v_normalized_histogram_bin_03_25_38nboxplotOverallImage}
			}{
				\ifthenelse{\boolean{isFile1}}{
					\ownClearPageSub
					\paragraph{Brigthness bin 3 (25-38) }~
					\loadTex{top_vis_hsv_v_histogram_bin_03_25_38nboxplotOverallImage}
				}{}
			}
			
			\resetBoolean
			\checkFileNoReset{top_vis_hsv_v_histogram_bin_04_38_51nboxplotOverallImage.tex}{}{}
			\checkFileNoReset{}{top_vis_hsv_v_normalized_histogram_bin_04_38_51nboxplotOverallImage.tex}{}
			\ifthenelse{\boolean{isFile2}}{
	
				\ownClearPageSub
				\paragraph{Brigthness bin 4 (38-51) (zoom corrected)}~
				\loadTex{top_vis_hsv_v_normalized_histogram_bin_04_38_51nboxplotOverallImage}
			}{
				\ifthenelse{\boolean{isFile1}}{
					\ownClearPageSub
					\paragraph{Brigthness bin 4 (38-51) }~
					\loadTex{top_vis_hsv_v_histogram_bin_04_38_51nboxplotOverallImage}
				}{}
			}
			
			\resetBoolean
			\checkFileNoReset{top_vis_hsv_v_histogram_bin_05_51_63nboxplotOverallImage.tex}{}{}
			\checkFileNoReset{}{top_vis_hsv_v_normalized_histogram_bin_05_51_63nboxplotOverallImage.tex}{}
			\ifthenelse{\boolean{isFile2}}{
	
				\ownClearPageSub
				\paragraph{Brigthness bin 5 (51-63) (zoom corrected)}~
				\loadTex{top_vis_hsv_v_normalized_histogram_bin_05_51_63nboxplotOverallImage}
			}{
				\ifthenelse{\boolean{isFile1}}{
					\ownClearPageSub
					\paragraph{Brigthness bin 5 (51-63) }~
					\loadTex{top_vis_hsv_v_histogram_bin_05_51_63nboxplotOverallImage}
				}{}
			}
			
			\resetBoolean
			\checkFileNoReset{top_vis_hsv_v_histogram_bin_06_63_76nboxplotOverallImage.tex}{}{}
			\checkFileNoReset{}{top_vis_hsv_v_normalized_histogram_bin_06_63_76nboxplotOverallImage.tex}{}
			\ifthenelse{\boolean{isFile2}}{
	
				\ownClearPageSub
				\paragraph{Brigthness bin 6 (63-76) (zoom corrected)}~
				\loadTex{top_vis_hsv_v_normalized_histogram_bin_06_63_76nboxplotOverallImage}
			}{
				\ifthenelse{\boolean{isFile1}}{
					\ownClearPageSub
					\paragraph{Brigthness bin 6 (63-76) }~
					\loadTex{top_vis_hsv_v_histogram_bin_06_63_76nboxplotOverallImage}
				}{}
			}
		
			\resetBoolean
			\checkFileNoReset{top_vis_hsv_v_histogram_bin_stackedOverallImage.tex}{}{}
			\checkFileNoReset{}{top_vis_hsv_v_normalized_histogram_bin_stackedOverallImage.tex}{}
			\ifthenelse{\boolean{isFile2}}{
	
				\ownClearPageSub
				\paragraph{Brigthness histogram (zoom corrected)}~
				\loadTex{top_vis_hsv_v_normalized_histogram_bin_stackedOverallImage}
			}{
				\ifthenelse{\boolean{isFile1}}{
					\ownClearPageSub
					\paragraph{Brigthness histogram}~
					\loadTex{top_vis_hsv_v_histogram_bin_stackedOverallImage}
				}{}
			}			
		}
	}			
}

	
	
	
	
	
	
	
	
	
	
	
% 	\resetBoolean
% 	\checkFileNoReset{HSV_Farbtonskala.png}{}{}
% 	\ifthenelse{\boolean{isFile1}}{
% 		\loadImage{HSV_Farbtonskala.png}
% 	}{}
% 	
% 	
% 	\checkFile{side_vis_hue_averagenboxplotOverallImage.tex}{top_vis_hue_averagenboxplotOverallImage.tex}{}
% 	\ifthenelse{\boolean{isFile1} \or \boolean{isFile2}}{
% 	
% 	%%side.vis.hue.average
% 		\ownClearPage
% 		\subsection{Average hue}
% 		
% 		\ifthenelse{\boolean{isFile1}}{
% 			\loadTex{side_vis_hue_averagenboxplotOverallImage}
% 		}{}	
% 		
% 	%%top.vis.hue.average
% 		\ifthenelse{\boolean{isFile2}}{		
% 			\loadTex{top_vis_hue_averagenboxplotOverallImage}
% 		}{}
% 	}{}
% 	
% 	
% 	\resetBoolean
% 	\checkFileNoReset{side_vis_hue_histogram_bin_stackedOverallImage.tex}{}{}
% 	\checkFileNoReset{}{side_vis_normalized_histogram_bin_stackedOverallImage.tex}{}
% 	\ifthenelse{\boolean{isFile2}}{
% 				
% 		\ownClearPage
% 		\subsection{Color histogram side view (zoom corrected)}
% 		%Visible light color histogram (zoom corrected) for side view (hue):
% 		\loadTex{side_vis_normalized_histogram_bin_stackedOverallImage}	
% 	}{
% 		\ifthenelse{\boolean{isFile1}}{
% 		
% 			\ownClearPage
% 			\subsection{Color histogram side view}
% 			%Visible light color histogram for side view (hue):
% 			\loadTex{side_vis_hue_histogram_bin_stackedOverallImage}
% 		}{}	
% 	}
% 	
% 	\resetBoolean
% 	\checkFileNoReset{top_vis_hue_histogram_bin_stackedOverallImage.tex}{}{}
% 	\checkFileNoReset{}{top_vis_normalized_histogram_bin_stackedOverallImage.tex}{}
% 	\ifthenelse{\boolean{isFile2}}{
% 				
% 		\ownClearPage
% 		\subsection{Color histogram top view (zoom corrected)}
% 		%Visible light color histogram (zoom corrected) for side view (hue):
% 		\loadTex{top_vis_normalized_histogram_bin_stackedOverallImage}	
% 	}{
% 		\ifthenelse{\boolean{isFile1}}{
% 		
% 			\ownClearPage
% 			\subsection{Color histogram top view}
% 			%Visible light color histogram for side view (hue):
% 			\loadTex{top_vis_hue_histogram_bin_stackedOverallImage}
% 		}{}	
% 	}		
% }{}


\resetBoolean
\checkFileNoReset{side_fluo_histogram_bin_stackedOverallImage.tex}{}{}
\checkFileNoReset{top_fluo_histogram_bin_stackedOverallImage.tex}{}{}
\checkFileNoReset{side_fluo_normalized_histogram_bin_stackedOverallImage.tex}{}{}
\checkFileNoReset{top_fluo_normalized_histogram_bin_stackedOverallImage.tex}{}{}
\checkFileNoReset{side_fluo_intensity_average__relative_nboxplotOverallImage.tex}{}{}
\checkFileNoReset{top_fluo_intensity_average__relative___pix_nboxplotOverallImage.tex}{}{}
\ifthenelse{\boolean{isFile1}}{

	\resetClear
	\clearpage
	\section{Fluorescence activity}
	\begin{itemize}
	\item Average and histogram of observed fluorescence colors (Fluo)
	\end{itemize}

	\checkFile{side_fluo_intensity_average__relative_nboxplotOverallImage.tex}{top_fluo_intensity_average__relative___pix_nboxplotOverallImage.tex}{}
	\ifthenelse{\boolean{isFile1} \or \boolean{isFile2}}{
		
		\ownClearPage
		\subsection{Average fluorescence activity intensity}

		\ifthenelse{\boolean{isFile1}}{
			\loadTex{side_fluo_intensity_average__relative_nboxplotOverallImage}
		}{}
		\ifthenelse{\boolean{isFile2}}{
			\loadTex{top_fluo_intensity_average__relative___pix_nboxplotOverallImage}
			}{}
	}{}

% 	\resetBoolean
% 	\checkFileNoReset{side_fluo_histogram_bin_stackedOverallImage.tex}{}{}
% 	\checkFileNoReset{top_fluo_histogram_bin_stackedOverallImage.tex}{}{}
% 	\checkFileNoReset{}{side_fluo_normalized_histogram_bin_stackedOverallImage.tex}{}
% 	\checkFileNoReset{}{top_fluo_normalized_histogram_bin_stackedOverallImage.tex}{}
% 	\ifthenelse{\boolean{isFile2}}{
% 
% 		\ownClearPage
% 		\subsection{Fluorescence spectra (zoom corrected)}
% 		\begin{itemize}
% 		\item Histogram of observed fluorescence colors
% 		\end{itemize}
% 		\checkFile{side_fluo_normalized_histogram_bin_stackedOverallImage.tex}{top_fluo_normalized_histogram_bin_stackedOverallImage.tex}{}
% 		\ifthenelse{\boolean{isFile1}}{
% 			Fluorescence activity distribution (zoom corrected) for side view:
% 			\loadTex{side_fluo_normalized_histogram_bin_stackedOverallImage}
% 		}{}
% 		
% 		\ifthenelse{\boolean{isFile2}}{
% 			Fluorescence activity distribution (not zoom corrected) for top view:
% 			\loadTex{side_fluo_normalized_histogram_bin_stackedOverallImage}
% 		}{}
% 	
% 	}{
% 		\ifthenelse{\boolean{isFile1}}{
% 		
% 			\ownClearPage
% 			\subsection{Fluorescence spectra}
% 			\begin{itemize}
% 			\item Histogram of observed fluorescence colors
% 			\end{itemize}
% 			\checkFile{side_fluo_histogram_bin_stackedOverallImage.tex}{top_fluo_histogram_bin_stackedOverallImage.tex}{}
% 			\ifthenelse{\boolean{isFile1}}{
% 			
% 				Fluorescence activity distribution for side view:
% 				\loadTex{side_fluo_histogram_bin_stackedOverallImage}
% 			}{}
% 			\ifthenelse{\boolean{isFile2}}{
% 			
% 				Fluorescence activity distribution for top view:
% 				\loadTex{top_fluo_histogram_bin_stackedOverallImage}
% 			}{}
% 		}{}
% 	}

	\resetBoolean
	\checkFileNoReset{side_fluo_histogram_bin_stackedOverallImage.tex}{}{}
	\checkFileNoReset{}{side_fluo_normalized_histogram_bin_stackedOverallImage.tex}{}
	\ifthenelse{\boolean{isFile2}}{

		\ownClearPage
		\subsection{Fluorescence spectra side view (zoom corrected)}
		%Fluorescence activity distribution (zoom corrected) for side view:
		\loadTex{side_fluo_normalized_histogram_bin_stackedOverallImage}
	}{
		\ifthenelse{\boolean{isFile1}}{
		
			\ownClearPage
			\subsection{Fluorescence spectra side view}
			%Fluorescence activity distribution for side view:
			\loadTex{side_fluo_histogram_bin_stackedOverallImage}
		}
	}
	
	\resetBoolean
	\checkFileNoReset{top_fluo_histogram_bin_stackedOverallImage.tex}{}{}
	\checkFileNoReset{}{top_fluo_normalized_histogram_bin_stackedOverallImage.tex}{}
	\ifthenelse{\boolean{isFile2}}{

		\ownClearPage
		\subsection{Fluorescence spectra top view (zoom corrected)}
		%Fluorescence activity distribution (zoom corrected) for side view:
		\loadTex{top_fluo_normalized_histogram_bin_stackedOverallImage}
	}{
		\ifthenelse{\boolean{isFile1}}{
		
			\ownClearPage
			\subsection{Fluorescence spectra top view}
			%Fluorescence activity distribution for side view:
			\loadTex{top_fluo_histogram_bin_stackedOverallImage}
		}
	}	
}{}

%\checkFile{side_fluo_histogram_ratio_bin_1_0_25_side_fluo_histogram_ratio_bin_2_2stackedOverallImage.tex}{top_fluo_histogram_ratio_bin_1_0_25_top_fluo_histogram_ratio_bin_2_25_stackedOverallImage.tex}
%\ifthenelse{\boolean{isFile1} \or \boolean{isFile2}}{
%	\clearpage
%	\subsection{Fluo ratio intensity (to do: check!!!!!)}
%	\begin{itemize}
%	\end{itemize}
%	\ifthenelse{\boolean{isFile1}}{
%		\loadTex{side_fluo_histogram_ratio_bin_1_0_25_side_fluo_histogram_ratio_bin_2_2stackedOverallImage}
%	}{}
%	\ifthenelse{\boolean{isFile2}}{	
%		\loadTex{top_fluo_histogram_ratio_bin_1_0_25_top_fluo_histogram_ratio_bin_2_25_stackedOverallImage}
%	}{}
%}{}
%
%
%\checkFile{side_fluo_normalized_histogram_ratio_bin_1_0_25_side_fluo_normalized_hstackedOverallImage.tex}{}
%\ifthenelse{\boolean{isFile1}}{
%	\clearpage
%	\subsection{Fluo ratio intensity (zoom corrected) (to do: check!!!!!)}
%	\begin{itemize}
%	\item xxx
%	\item The darker the higher acitivity
%	\item Column name: xxx
%	\item Unit: %
%	\end{itemize}
%	\loadTex{side_fluo_normalized_histogram_ratio_bin_1_0_25_side_fluo_normalized_hstackedOverallImage}
%}{}

\resetBoolean
\checkFileNoReset{side_nir_histogram_bin_stackedOverallImage.tex}{}{}
\checkFileNoReset{top_nir_histogram_bin_stackedOverallImage.tex}{}{}
\checkFileNoReset{side_nir_normalized_histogram_bin_stackedOverallImage.tex}{}{}
\checkFileNoReset{top_nir_normalized_histogram_bin_stackedOverallImage.tex}{}{}
\checkFileNoReset{side_nir_intensity_average__relative_nboxplotOverallImage.tex}{}{}
\checkFileNoReset{top_nir_intensity_average__relative___pix_nboxplotOverallImage.tex}{}{}
\checkFileNoReset{side_nir_skeleton_intensity_average__relative_nboxplotOverallImage.tex}{}{}
\checkFileNoReset{top_nir_skeleton_intensity_average__relative_nboxplotOverallImage.tex}{}{}

\ifthenelse{\boolean{isFile1}}{

	\resetClear
	\clearpage
	\section{Near-infrared intensity}
	\begin{itemize}
	\item Average intensity of near infrared (NIR)
	\item Represents the water content of the plant
	\end{itemize}
	
	\resetBoolean
	\checkFileNoReset{side_nir_intensity_average__relative_nboxplotOverallImage.tex}{}{}
	\checkFileNoReset{top_nir_intensity_average__relative___pix_nboxplotOverallImage.tex}{}{}
	\checkFileNoReset{side_nir_skeleton_intensity_average__relative_nboxplotOverallImage.tex}{}{}
	\checkFileNoReset{top_nir_skeleton_intensity_averagenboxplotOverallImage.tex}{}{}

	\ifthenelse{\boolean{isFile1}}{
% 	\checkFile{side_nir_intensity_average__relative_nboxplotOverallImage.tex}{top_nir_intensity_average__relative___pix_nboxplotOverallImage.tex}{}
% 	\ifthenelse{\boolean{isFile1} \or \boolean{isFile2}}{
		\ownClearPage
		\subsection{Average near-infrared intensity}
			
		\checkFile{side_nir_intensity_average__relative_nboxplotOverallImage.tex}{}{}
		\ifthenelse{\boolean{isFile1}}{
			Average NIR intensity for side view:
			\loadTex{side_nir_intensity_average__relative_nboxplotOverallImage}
		}{}
		
		\checkFile{side_nir_skeleton_intensity_average__relative_nboxplotOverallImage.tex}{}{}
		\ifthenelse{\boolean{isFile1}}{
			Average NIR intensity of skeleton for side view:
			\loadTex{side_nir_skeleton_intensity_average__relative_nboxplotOverallImage}
		}{}
		
		\checkFile{top_nir_intensity_average__relative___pix_nboxplotOverallImage.tex}{}{}
		\ifthenelse{\boolean{isFile1}}{
			Average NIR intensity for top view:
			\loadTex{top_nir_intensity_average__relative___pix_nboxplotOverallImage}
		}{}
		
		\checkFile{top_nir_skeleton_intensity_averagenboxplotOverallImage.tex}{}{}
		\ifthenelse{\boolean{isFile1}}{
			Average NIR intensity of skeleton for top view:
			\loadTex{top_nir_skeleton_intensity_averagenboxplotOverallImage}
		}{}
	}{}
	
	
	\resetBoolean
	\checkFileNoReset{side_nir_histogram_bin_stackedOverallImage.tex}{}{}
	\checkFileNoReset{}{side_nir_normalized_histogram_bin_stackedOverallImage.tex}{}
	\ifthenelse{\boolean{isFile2}}{

			\ownClearPage
			\subsection{Intensity histogram side view (zoom corrected)}
			%NIR intensity distribution for side view (zoom corrected):
			\loadTex{side_nir_normalized_histogram_bin_stackedOverallImage}
	}{
		\ifthenelse{\boolean{isFile1}}{
		
			\ownClearPage
			\subsection{Intensity histogram side view}
			%NIR intensity distribution for side view:
			\loadTex{side_nir_histogram_bin_stackedOverallImage}
		}{}
	}
	
	\resetBoolean
	\checkFileNoReset{top_nir_histogram_bin_stackedOverallImage.tex}{}{}
	\checkFileNoReset{}{top_nir_normalized_histogram_bin_stackedOverallImage.tex}{}
	\ifthenelse{\boolean{isFile2}}{

			\ownClearPage
			\subsection{Intensity histogram top view (zoom corrected)}
			\begin{itemize}
			\item Average intensity of near infrared (NIR)
			\item Represents the water content of the plant
			\end{itemize}
			%NIR intensity distribution for side view (zoom corrected):
			\loadTex{top_nir_normalized_histogram_bin_stackedOverallImage}
	}{
		\ifthenelse{\boolean{isFile1}}{
		
			\ownClearPage
			\subsection{Intensity histogram top view}
			\begin{itemize}
			\item Average intensity of near infrared (NIR)
			\item Represents the water content of the plant
			\end{itemize}	
			%NIR intensity distribution for side view:
			\loadTex{top_nir_histogram_bin_stackedOverallImage}
		}{}
	}
}{}

% \resetBoolean
% 	\checkFileNoReset{side_nir_histogram_bin_stackedOverallImage.tex}{}{}
% 	\checkFileNoReset{top_nir_histogram_bin_stackedOverallImage.tex}{}{}
% 	\checkFileNoReset{}{side_nir_normalized_histogram_bin_stackedOverallImage.tex}{}
% 	\checkFileNoReset{}{top_nir_histogram_bin_stackedOverallImage.tex}{}
% 	\ifthenelse{\boolean{isFile2}}{
% 
% 			\ownClearPage
% 			\subsection{Intensity histogram (zoom corrected)}
% 			\begin{itemize}
% 			\item Average intensity of near infrared (NIR)
% 			\item Represents the water content of the plant
% 			\end{itemize}
% 			
% 			\checkFile{side_nir_normalized_histogram_bin_stackedOverallImage.tex}{top_nir_histogram_bin_stackedOverallImage.tex}{}
% 			\ifthenelse{\boolean{isFile1}}{
% 				NIR intensity distribution for side view (zoom corrected):
% 				\loadTex{side_nir_normalized_histogram_bin_stackedOverallImage}
% 			}{}
% 			\ifthenelse{\boolean{isFile2}}{
% 				NIR intensity distribution for top view (not zoom corrected):
% 				\loadTex{top_nir_histogram_bin_stackedOverallImage}
% 			}{}
% 	
% 	}{
% 		\ifthenelse{\boolean{isFile1}}{
% 		
% 			\ownClearPage
% 			\subsection{Intensity histogram}
% 			\begin{itemize}
% 			\item Average intensity of near infrared (NIR)
% 			\item Represents the water content of the plant
% 			\end{itemize}
% 			\checkFile{side_nir_histogram_bin_stackedOverallImage.tex}{top_nir_histogram_bin_stackedOverallImage.tex}{}
% 			\ifthenelse{\boolean{isFile1}}{
% 			
% 				NIR intensity distribution for side view:
% 				\loadTex{side_nir_histogram_bin_stackedOverallImage}
% 			}{}
% 			\ifthenelse{\boolean{isFile2}}{
% 			
% 				NIR intensity distribution for top view:
% 				\loadTex{top_nir_histogram_bin_stackedOverallImage}
% 			}{}
% 		}{}
% 	}

\resetBoolean
\checkFileNoReset{side_ir_histogram_bin_stackedOverallImage.tex}{}{}
\checkFileNoReset{top_ir_histogram_bin_stackedOverallImage.tex}{}{}
\checkFileNoReset{top_ir_intensity_averagenboxplotOverallImage.tex}{}{}
\checkFileNoReset{side_ir_intensity_averagenboxplotOverallImage.tex}{}{}

\ifthenelse{\boolean{isFile1}}{

	\resetClear
	\clearpage
	\section{Infrared intensity}
	\begin{itemize}
	\item Average intensity of infrared (IR)
	\item Represents the relative temperature of the plant in comparison to the background.
	The higher the value, the colder the leafs. High values may be the result of high transpiration. For temperature measurements, the upper 50 percent of
	the background temperature values are used. If there is a blue rubber mat, the result of this calculation scheme is, that the warmer temperature of the
	rubber mat is used for comparison, not the possibly colder soil (it may be colder because of transpiration).
	\end{itemize}
	
	\checkFile{side_ir_intensity_averagenboxplotOverallImage.tex}{top_ir_intensity_averagenboxplotOverallImage.tex}{}
	\ifthenelse{\boolean{isFile1} \or \boolean{isFile2}}{
		\ownClearPage
		\subsection{Average infrared intensity}
		
		\ifthenelse{\boolean{isFile1}}{	
			Average IR intensity for side view:
			\loadTex{side_ir_intensity_averagenboxplotOverallImage}
		}{}
		\ifthenelse{\boolean{isFile2}}{
			Average IR intensity for top view:
			\loadTex{top_ir_intensity_averagenboxplotOverallImage}
		}{}
	}{}
	
	
	\resetBoolean
	\checkFileNoReset{side_ir_histogram_bin_stackedOverallImage.tex}{}{}
	\ifthenelse{\boolean{isFile1}}{
		
		\ownClearPage
		\subsection{Intensity histogram side view}
		\loadTex{side_ir_histogram_bin_stackedOverallImage}
	}{}
	
	\resetBoolean
	\checkFileNoReset{top_ir_histogram_bin_stackedOverallImage.tex}{}{}
	\ifthenelse{\boolean{isFile1}}{
		
		\ownClearPage
		\subsection{Intensity histogram top view}
		\loadTex{top_ir_histogram_bin_stackedOverallImage}
	}{}
}{}


\resetBoolean
\checkFileNoReset{side_leaf_count_median__leafs_nboxplotOverallImage.tex}{}{}
\checkFileNoReset{side_leaf_length_sum_norm_max__mm_nboxplotOverallImage.tex}{}{}
\checkFileNoReset{side_bloom_count__tassel_nboxplotOverallImage.tex}{}{}
\ifthenelse{\boolean{isFile1}}{
	
	\resetClear
	\clearpage
	\section{Plant structures}

	% http://upload.wikimedia.org/wikipedia/commons/9/98/Maize_plant_diagram.svg
	
	\checkFile{side_leaf_count_median__leafs_nboxplotOverallImage.tex}{}{}
	\ifthenelse{\boolean{isFile1}}{
		\ownClearPage
		\subsection{Number of leafs}
		\begin{itemize}
		\item Number of leafs-tips
		\item Colum name: side.leaf.count.median
		\end{itemize}
		\loadTex{side_leaf_count_median__leafs_nboxplotOverallImage}
	}{}
	
	
	\checkFile{side_leaf_length_sum_norm_max__mm_nboxplotOverallImage.tex}{}{}
	\ifthenelse{\boolean{isFile1}}{
		\ownClearPage
		\subsection{Leaf lengths}
		\begin{itemize}
		\item Length of all leafs plus stem
		\item Column name: side.leaf.length.sum.norm.max
		\item Unit: mm
		\item Hint: the yellow line in the image
		\end{itemize}
		\loadTex{side_leaf_length_sum_norm_max__mm_nboxplotOverallImage}
	}{}
	
	
	\checkFile{side_bloom_count__tassel_nboxplotOverallImage.tex}{}{}
	\ifthenelse{\boolean{isFile1}}{
		\ownClearPage	
		\subsection{Flower detection}
		\begin{itemize}
		\item Number of tassel florets
		\item Colum name: side.bloom.count
		\item Hint: the number of blue retangles in the image
		\end{itemize}
		\loadTex{side_bloom_count__tassel_nboxplotOverallImage}
	}{}

}{}


\resetBoolean
\checkFileNoReset{side_nir_wetness_average__percent_.pdf}{}{}
\checkFileNoReset{top_nir_wetness_average__percent_nboxplotOverallImage.tex}{}{}
\checkFileNoReset{side_nir_wetness_plant_weight_drought_lossnboxplotOverallImage.tex}{}{}
\checkFileNoReset{top_nir_wetness_plant_weight_drought_lossnboxplotOverallImage.tex}{}{}
\ifthenelse{\boolean{isFile1}}{

	\resetClear
	\clearpage
	\section{\noindent Wetness}
	
	\begin{itemize}
	\item Near-infrared analysis
	\end{itemize}
	
	
	\checkFile{side_nir_wetness_average__percent_.pdf}{}{}
	\ifthenelse{\boolean{isFile1}}{
	
		\ownClearPage
		\subsection{Average wetness of side image}
		\begin{itemize}
		\item Average wetness of the plants from NIR side camera
		\item Column name: side.nir.wetness.av
		\item Unit: \%
		\end{itemize}
		\loadImage{side_nir_wetness_average__percent_.pdf}
	}{}
	
	
	\checkFile{top_nir_wetness_average__percent_nboxplotOverallImage.tex}{}{}
	\ifthenelse{\boolean{isFile1}}{
		\ownClearPage
		\subsection{Average wetness of top image}
		\begin{itemize}
		\item Average wetness of the plants from NIR top camera
		\item Column name: top.nir.wetness.av
		\item Unit: \%
		\end{itemize}
		\loadTex{top_nir_wetness_average__percent_nboxplotOverallImage}
	}{}	
		
		
	\checkFile{side_nir_wetness_plant_weight_drought_lossnboxplotOverallImage.tex}{}{}
	\ifthenelse{\boolean{isFile1}}{
		\ownClearPage	
		\subsection{Weighted loss through drought stress - side image}
		\begin{itemize}
		\item Number of foreground pixels from NIR side camera minus the weighted value of the plant
		\item weightOfPlant = fully wet: 1 unit, fully dry: 1/7 unit
		\item Column name: side.nir.wetness.plant\_weight\_drought\_loss
		\end{itemize}
		\loadTex{side_nir_wetness_plant_weight_drought_lossnboxplotOverallImage}
	}{}
	
	
	\checkFile{top_nir_wetness_plant_weight_drought_lossnboxplotOverallImage.tex}{}{}
	\ifthenelse{\boolean{isFile1}}{
		\ownClearPage
		\subsection{Weighted loss through drought stress - top image}
		\begin{itemize}
		\item Number of foreground pixels from NIR top camera minus the weighted value of the plant
		\item weightOfPlant = fully wet: 1 unit, fully dry: 1/7 unit
		\item Column name: top.nir.wetness.plant\_weight\_drought\_loss
		\end{itemize}
		\loadTex{top_nir_wetness_plant_weight_drought_lossnboxplotOverallImage}
	}{}
}{}

\resetBoolean
\checkFileNoReset{side_hull_area__px_nboxplotOverallImage.tex}{}{}
\checkFileNoReset{side_hull_area_norm__mm_2_nboxplotOverallImage.tex}{}{}
\checkFileNoReset{side_hull_pc1nboxplotOverallImage.tex}{}{}
\checkFileNoReset{side_hull_pc1_normnboxplotOverallImage.tex}{}{}
\checkFileNoReset{side_hull_pc2nboxplotOverallImage.tex}{}{}
\checkFileNoReset{side_hull_pc2_normnboxplotOverallImage.tex}{}{}
\checkFileNoReset{side_hull_fillgrade__percent_nboxplotOverallImage.tex}{}{}
\checkFileNoReset{top_hull_area__px_nboxplotOverallImage.tex}{}{}
\checkFileNoReset{top_hull_area_norm__mm_2_nboxplotOverallImage.tex}{}{}
\checkFileNoReset{top_hull_pc1nboxplotOverallImage.tex}{}{}
\checkFileNoReset{top_hull_pc1_normnboxplotOverallImage.tex}{}{}
\checkFileNoReset{top_hull_pc2nboxplotOverallImage.tex}{}{}
\checkFileNoReset{top_hull_pc2_normnboxplotOverallImage.tex}{}{}
\checkFileNoReset{top_hull_fillgrade__percent_nboxplotOverallImage.tex}{}{}

\checkFileNoReset{side_compactness_16__relative_nboxplotOverallImage.tex}{}{}
\checkFileNoReset{top_compactness_16__relative_nboxplotOverallImage.tex}{}{}
\checkFileNoReset{side_hull_circularity__relative_nboxplotOverallImage.tex}{}{}
\checkFileNoReset{top_hull_circularity__relative_nboxplotOverallImage.tex}{}{}
\checkFileNoReset{side_compactness_01__relative_nboxplotOverallImage.tex}{}{}
\checkFileNoReset{top_compactness_01__relative_nboxplotOverallImage.tex}{}{}

\ifthenelse{\boolean{isFile1}}{

	\resetClear
	\clearpage
	\section{Convex hull}

	"In mathematics, the convex hull or convex envelope for a set of points X in a
	real vector space V (for example, usual 2- or 3-dimensional space) is the
	minimal convex set containing X. When the set X is a finite subset of the
	plane, we may imagine stretching a rubber band so that it surrounds the entire
	set X and then releasing it, allowing it to contract; when it becomes taut, it
	encloses the convex hull of X."
	
 	\begin{tiny}
 	Wikipedia contributors, "Convex hull",
 	Wikipedia, The Free Encyclopedia,
	\url{http://en.wikipedia.org/w/index.php?title=Convex_hull&oldid=482955324}
	(accessed March 30, 2012).
	\end{tiny}

	\resetBoolean
	\checkFileNoReset{side_hull_area__px_nboxplotOverallImage.tex}{}{}
	\checkFileNoReset{side_hull_area_norm__mm_2_nboxplotOverallImage.tex}{}{}
	\checkFileNoReset{side_hull_pc1nboxplotOverallImage.tex}{}{}
	\checkFileNoReset{side_hull_pc1_normnboxplotOverallImage.tex}{}{}
	\checkFileNoReset{side_hull_pc2nboxplotOverallImage.tex}{}{}
	\checkFileNoReset{side_hull_pc2_normnboxplotOverallImage.tex}{}{}
	\checkFileNoReset{side_hull_fillgrade__percent_nboxplotOverallImage.tex}{}{}
	\checkFileNoReset{side_compactness_16__relative_nboxplotOverallImage.tex}{}{}
	\checkFileNoReset{side_hull_circularity__relative_nboxplotOverallImage.tex}{}{}
	\checkFileNoReset{side_compactness_01__relative_nboxplotOverallImage.tex}{}{}
	\checkFileNoReset{side_hull_circumcircle_d__px_nboxplotOverallImage.tex}{}{}
	\ifthenelse{\boolean{isFile1}}{
 		
 		\resetClearSub
 		\ownClearPage
 		\subsection{Side}


		\resetBoolean
		\checkFileNoReset{side_hull_area__px_nboxplotOverallImage.tex}{}{}
		\checkFileNoReset{}{side_hull_area_norm__mm_2_nboxplotOverallImage.tex}{}
		\ifthenelse{\boolean{isFile2}}{

			\ownClearPageSub
			\subsubsection{Area (zoom corrected)}
			\begin{itemize}
			\item Area which is enclosed of the convex hull (zoom corrected)
			\item Unit: px
			\end{itemize}
			\loadTex{side_hull_area_norm__mm_2_nboxplotOverallImage}
		}{
			\ifthenelse{\boolean{isFile1}}{
				\ownClearPageSub
				\subsubsection{Area}
				\begin{itemize}
				\item Area which is enclosed of the convex hull
				\item Unit: px
				\end{itemize}
				\loadTex{side_hull_area__px_nboxplotOverallImage}
			}{}
		}
		
		\checkFileNoReset{side_hull_circumcircle_d__px_nboxplotOverallImage.tex}{}{}
		\ifthenelse{\boolean{isFile1}}{
	
			\ownClearPageSub
			\subsubsection{Circumcircle diameter}
			\begin{itemize}
			\item The minimum diameter of a circle surrounding the plant. 
			\item Unit: px
			\end{itemize}
			\loadTex{side_hull_circumcircle_d__px_nboxplotOverallImage}
			
		}


		\resetBoolean
		\checkFileNoReset{side_hull_pc1nboxplotOverallImage.tex}{}{}
		\checkFileNoReset{}{side_hull_pc1_normnboxplotOverallImage.tex}{}
		\ifthenelse{\boolean{isFile2}}{
		
			\ownClearPageSub
			\subsubsection{PC1 - maximum distance (zoom corrected)}
			\begin{itemize}
			\item Maximum extension of the plant (zoom corrected)
			\item Unit: mm
			\end{itemize}
			\loadTex{side_hull_pc1_normnboxplotOverallImage}
			
		}{
			\ifthenelse{\boolean{isFile1}}{
				
				\ownClearPageSub
				\subsubsection{PC1 - maximum distance}
				\begin{itemize}
				\item Maximum extension of the plant
				\item Unit: px
				\end{itemize}
				\loadTex{side_hull_pc1nboxplotOverallImage}
			}{}	
		}

		
		\resetBoolean
		\checkFileNoReset{side_hull_pc2nboxplotOverallImage.tex}{}{}
		\checkFileNoReset{}{side_hull_pc2_normnboxplotOverallImage.tex}{}
		\ifthenelse{\boolean{isFile2}}{
			
			\ownClearPageSub
			\subsubsection{PC2 - Opposite direction maxmium distance (zoom corrected)}
			\begin{itemize}
			\item Opposite direction of the maximum extension of the plant (zoom corrected)
			\item Unit: mm
			\end{itemize}
			\loadTex{side_hull_pc2_normnboxplotOverallImage}

		}{
			\ifthenelse{\boolean{isFile1}}{
				\ownClearPageSub
				\subsubsection{PC2 - Opposite direction maxmium distance}
				\begin{itemize}
				\item Opposite direction of the maximum extension of the plant
				\item Unit: px
				\end{itemize}
				\loadTex{side_hull_pc2nboxplotOverallImage}
			}{}
		}
				

		\checkFile{side_hull_fillgrade__percent_nboxplotOverallImage.tex}{}{}
		\ifthenelse{\boolean{isFile1}}{
			\ownClearPageSub
			\subsubsection{Fillgrade}
			\begin{itemize}
			\item Ratio of the area of the whole convex hull and the area of the plant
			(green pixel)
			\item Unit: none
			\end{itemize}
			\loadTex{side_hull_fillgrade__percent_nboxplotOverallImage}
		}{}
		
		\checkFile{side_compactness_16__relative_nboxplotOverallImage.tex}{side_compactness_01__relative_nboxplotOverallImage.tex}{}
		\ifthenelse{\boolean{isFile1} \or \boolean{isFile2}}{
			\ownClearPageSub
			\subsubsection{Compactness}
			
			"The compactness measure of a shape, sometimes called the shape factor, is a
			numerical quantity representing the degree to which a shape is compact.
			\ldots A common compactness measure is the Isoperimetric quotient, the ratio
			of the area of the shape to the area of a circle (the most compact shape) having the same perimeter."
	
		 	\begin{tiny}
		 	Wikipedia contributors, "Compactness measure of a shape",
		 	Wikipedia, The Free Encyclopedia,
			\url{http://en.wikipedia.org/w/index.php?title=Compactness_measure_of_a_shape&oldid=485591387}
			(accessed June 25, 2012).
			\end{tiny}
			
			\ifthenelse{\boolean{isFile1}} {
				\loadTex{side_compactness_16__relative_nboxplotOverallImage}
			}{}
			\ifthenelse{\boolean{isFile2}} {
				\loadTex{side_compactness_01__relative_nboxplotOverallImage}
			}{}
		
		\checkFile{side_hull_circularity__relative_nboxplotOverallImage.tex}{}{}
		\ifthenelse{\boolean{isFile1}}{
			\ownClearPageSub
			\subsubsection{Circularity}
			"Roundness is the measure of the sharpness of a particle's edges and corners."
	
		 	\begin{tiny}
		 	Wikipedia contributors, "Roundness (object)",
		 	Wikipedia, The Free Encyclopedia,
			\url{http://en.wikipedia.org/w/index.php?title=Roundness_(object)&oldid=487251484}
			(accessed June 25, 2012).
			\end{tiny}
			\loadTex{side_hull_circularity__relative_nboxplotOverallImage}
		}{}
	}{}
}{}


	\resetBoolean
	\checkFileNoReset{top_hull_area__px_nboxplotOverallImage.tex}{}{}
	\checkFileNoReset{top_hull_area_norm__mm_2_nboxplotOverallImage.tex}{}{}
	\checkFileNoReset{top_hull_pc1nboxplotOverallImage.tex}{}{}
	\checkFileNoReset{top_hull_pc1_normnboxplotOverallImage.tex}{}{}
	\checkFileNoReset{top_hull_pc2nboxplotOverallImage.tex}{}{}
	\checkFileNoReset{top_hull_pc2_normnboxplotOverallImage.tex}{}{}
	\checkFileNoReset{top_hull_fillgrade__percent_nboxplotOverallImage.tex}{}{}
	\checkFileNoReset{top_compactness_16__relative_nboxplotOverallImage.tex}{}{}
	\checkFileNoReset{top_hull_circularity__relative_nboxplotOverallImage.tex}{}{}
	\checkFileNoReset{top_compactness_01__relative_nboxplotOverallImage.tex}{}{}
	\checkFileNoReset{top_hull_circumcircle_d__px_nboxplotOverallImage.tex}{}{}
	\ifthenelse{\boolean{isFile1}}{
		\resetClearSub
		\ownClearPage
		\subsection{Top}

		\resetBoolean
		\checkFileNoReset{top_hull_area__px_nboxplotOverallImage.tex}{}{}
		\checkFileNoReset{}{top_hull_area_norm__mm_2_nboxplotOverallImage.tex}{}
		\ifthenelse{\boolean{isFile2}}{
		
			\ownClearPageSub
			\subsubsection{Area (zoom corrected)}
			\begin{itemize}
			\item Area which is enclosed of the convex hull (zoom corrected)
			\item Unit: px
			\end{itemize}
			\loadTex{top_hull_area_norm__mm_2_nboxplotOverallImage}
		}{
			\ifthenelse{\boolean{isFile1}}{
				\ownClearPageSub
				\subsubsection{Area}
				\begin{itemize}
				\item Area which is enclosed of the convex hull
				\item Unit: px
				\end{itemize}
				\loadTex{top_hull_area__px_nboxplotOverallImage}	
			}{}
		}

		\checkFileNoReset{top_hull_circumcircle_d__px_nboxplotOverallImage.tex}{}{}
		\ifthenelse{\boolean{isFile1}}{
	
			\ownClearPageSub
			\subsubsection{Circumcircle diameter}
			\begin{itemize}
			\item The minimum diameter of a circle surrounding the plant. 
			\item Unit: px
			\end{itemize}
			\loadTex{top_hull_circumcircle_d__px_nboxplotOverallImage}
			
		}

		\resetBoolean
		\checkFileNoReset{top_hull_pc1nboxplotOverallImage.tex}{}{}
		\checkFileNoReset{}{top_hull_pc1_normnboxplotOverallImage.tex}{}
		\ifthenelse{\boolean{isFile2}}{
			
			\ownClearPageSub
			\subsubsection{PC1 - maximum distance (zoom corrected)}
			\begin{itemize}
			\item Maximum extension of the plant (zoom corrected)
			\item Unit: mm
			\end{itemize}
			\loadTex{top_hull_pc1_normnboxplotOverallImage}

		}{
			\ifthenelse{\boolean{isFile1}}{
				\ownClearPageSub
				\subsubsection{PC1 - maximum distance}
				\begin{itemize}
				\item Maximum extension of the plant
				\item Unit: px
				\end{itemize}
				\loadTex{top_hull_pc1nboxplotOverallImage}
			}{}
		}

		
		\resetBoolean
		\checkFileNoReset{top_hull_pc2nboxplotOverallImage.tex}{}{}
		\checkFileNoReset{}{top_hull_pc2_normnboxplotOverallImage.tex}{}
		\ifthenelse{\boolean{isFile2}}{
		
			\ownClearPageSub
			\subsubsection{PC2 - Opposite direction maxmium distance (zoom corrected)}
			\begin{itemize}
			\item Opposite direction of the maximum extension of the plant (zoom corrected)
			\item Unit: mm
			\end{itemize}
			\loadTex{top_hull_pc2_normnboxplotOverallImage}

		}{
			\ifthenelse{\boolean{isFile1}}{
				\ownClearPageSub
				\subsubsection{PC2 - Opposite direction maxmium distance}
				\begin{itemize}
				\item Opposite direction of the maximum extension of the plant
				\item Unit: px
				\end{itemize}
				\loadTex{top_hull_pc2nboxplotOverallImage.tex}
			}{}
		}

		\checkFile{top_hull_fillgrade__percent_nboxplotOverallImage.tex}{}{}
		\ifthenelse{\boolean{isFile1}}{
			\ownClearPageSub
			\subsubsection{Fillgrade}
			\begin{itemize}
			\item Ratio of the area of the whole convex hull and the area of the plant
			(``green pixel'')
			\item Unit: none
			\end{itemize}
			\loadTex{top_hull_fillgrade__percent_nboxplotOverallImage}
		}{}
		
		\checkFile{top_compactness_16__relative_nboxplotOverallImage.tex}{top_compactness_01__relative_nboxplotOverallImage.tex}{}
		\ifthenelse{\boolean{isFile1} \or \boolean{isFile2}}{
			\ownClearPageSub
			\subsubsection{Compactness}
			
			"The compactness measure of a shape, sometimes called the shape factor, is a
			numerical quantity representing the degree to which a shape is compact.
			\ldots A common compactness measure is the Isoperimetric quotient, the ratio
			of the area of the shape to the area of a circle (the most compact shape) having the same perimeter."
	
		 	\begin{tiny}
		 	Wikipedia contributors, "Compactness measure of a shape",
		 	Wikipedia, The Free Encyclopedia,
			\url{http://en.wikipedia.org/w/index.php?title=Compactness_measure_of_a_shape&oldid=485591387}
			(accessed June 25, 2012).
			\end{tiny}
			
			\ifthenelse{\boolean{isFile1}} {
				\loadTex{top_compactness_16__relative_nboxplotOverallImage}
			}{}
			\ifthenelse{\boolean{isFile2}} {
				\loadTex{top_compactness_01__relative_nboxplotOverallImage}
			}{}
		}{}
		
		\checkFile{top_hull_circularity__relative_nboxplotOverallImage.tex}{}{}
		\ifthenelse{\boolean{isFile1}}{
			\ownClearPageSub
			\subsubsection{Circularity}
			"Roundness is the measure of the sharpness of a particle's edges and corners."
	
		 	\begin{tiny}
		 	Wikipedia contributors, "Roundness (object)",
		 	Wikipedia, The Free Encyclopedia,
			\url{http://en.wikipedia.org/w/index.php?title=Roundness_(object)&oldid=487251484}
			(accessed June 25, 2012).
			\end{tiny}
			\loadTex{top_hull_circularity__relative_nboxplotOverallImage}
		}{}
		
	}{}
}{}

\checkFile{appendixImage.tex}{}{}
\ifthenelse{\boolean{isFile1}}{

	\resetClear
	\clearpage
	\section{Appendix}
	\loadTex{appendixImage}
}{}


\vfill
\small{Leibniz Institute of Plant Genetics and Crop Plant Research, Group Image Analysis, Corrensstr. 3, 06466 Gatersleben, Germany}
\newline
\small{\textit{URL:} http://ba-13.ipk-gatersleben.de}
\end{document}