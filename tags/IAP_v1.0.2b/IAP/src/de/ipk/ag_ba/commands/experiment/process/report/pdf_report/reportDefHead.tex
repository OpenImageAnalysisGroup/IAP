% 
% Author: Entzian, Klukas (bug-fixing)
%%%%%%%%%%%%%%%%%%%%%%%%%%%%%%%%%%%%%%%%%%%%%%%%%%%%%%%%%%%%%%%%%%%%%%%%%%%%%%%%%%%%%%%%%%%%

%%%%%%%%%%%%% Klassen-Optionen
\documentclass[%
  paper=a4, % Stellt auf A4-Papier ein
  pagesize, % Diese Option reicht die Papiergröße an
            % alle Ausgabeformate weiter
  DIV=calc, % Errechnet einen guten Satzspiegel
%  BCOR=1cm, % Weil bei mir immer 1cm in der Bindung
            % der Klemmmappe verschwindet
  headings=small,% Für etwas kleinere Überschriften
  english,   % englische Rechtschreibung d.\,h. (Silbentrennung)
  oneside
]{scrartcl}  % Eine Klasse für einseitige Texte mit Kapiteln  
            %
\usepackage[T1]{fontenc}
\usepackage[utf8]{inputenc} %ermöglicht die direkte Eingabe von Sonderzeichen
%\usepackage[utf8]{inputenc} %ermöglicht die direkte Eingabe von Sonderzeichen
\usepackage{fixltx2e}
\usepackage[english]{babel} % Anpassung der Überschriften (chapter, chapitre,
% Kapitel …) und die richtige Silbentrennung
\usepackage[
bookmarks=true,
   % Lesezeichen erzeugen
bookmarksopen=false,
   % Lesezeichen nicht ausgeklappt
bookmarksnumbered=true,
   % Anzeige der Kapitelzahlen am Anfang der Namen der Lesezeichen
pdfstartpage=1,
   % Seite, welche automatisch geöffnet werden soll
   % praktisch, wenn z.B. im Inhaltsverzeichnis gestartet
   % werden soll oder eine Seite bearbeitet wird.
baseurl=http://iap.ipk-gatersleben.de/,
   % URL des PDF-Dokuments (oder Hintergrundinformationen)
pdftitle={--experimentname--},
   % Titel des PDF-Dokuments
pdfauthor={A. Entzian, D. Chen, C. Klukas},
   % Autor(Innen) des PDF-Dokuments
%pdfsubject={Kurzbeschreibung als ein Satz},
   % Inhaltsbeschreibung des PDF-Dokuments
pdfkeywords={image analysis, phenotyping, Arabidopsis, barley, maize},
   % Stichwortangabe zum PDF-Dokument
breaklinks=true,
   % ermöglicht einen Umbruch von URLs
colorlinks=true,
   % Einfärbung von Links
linkcolor=black,
   % Linkfarbe: schwarz
anchorcolor=black,
   % Ankerfarbe: schwarz
citecolor=black,
   % Literaturlinks: schwarz
filecolor=black,
   % Links zu lokalen Dateien: schwarz
menucolor=black,
   % Acrobat Menü Einträge: schwarz
pagecolor=black,
   % Links zu anderen Seiten im Text: schwarz
urlcolor=black
   % URL-Farbe: schwarz
]{hyperref}
\usepackage[left=2.5cm,right=2.5cm,top=2.5cm,bottom=2.5cm]{geometry}
\usepackage{datetime}

%%%%%%%%%%%%% Typografisch empfehlenswerte Pakete
\usepackage{% 
  %ellipsis, % Korrigiert den Weißraum um Auslassungspunkte
  ragged2e, % Ermöglicht Flattersatz mit Silbentrennung
 %marginnote,% Für bessere Randnotizen mit \marginnote statt
            % \marginline 
}
% \usepackage[tracking=true]{microtype}% 
%             % Microtype ist einfach super, aber lesen Sie
%             % unbedingt die Anleitung um das Folgende zu
%             % verstehen.
% \DeclareMicrotypeSet*[tracking]{my}% 
%   { font = */*/*/sc/* }% 
% \SetTracking{ encoding = *, shape = sc }{ 45 }% Hier wird festgelegt,
%             % dass alle Passagen in Kapitälchen automatisch leicht
%             % gesperrt werden. Das Paket soul, das ich früher empfohlen
%             % habe ist damit für diese Zwecke nicht mehr nötig.
%             %

\usepackage{%
  lmodern, % A) Latin Modern Fonts sind die Nachfolger von Computer
            % Modern, den LaTeX-Standardfonts
%  hfoldsty % B) Diese Schrift stellt alle Ziffern, außer
            % im Mathemodus, auf Minuskel- oder Mediäval-Ziffern um.
            % Wenn Ihre pdfs unscharf aussehen installieren Sie bitte
            % die cm-super-Fonts (Type1-Fonts).
% charter   % C) Diese Zeile lädt die Charter als Schriftart
}
% \usepackage[osf,sc]{mathpazo}% D) So erreichen Sie Palatino als
            % Schrift mit Minuskel-Ziffern und echten Kapitälchen
            %

%\usepackage{here}
%\usepackage{scrpage2} 
%\usepackage{tocloft}
\usepackage{graphicx}
%\usepackage{amsmath} 
\usepackage{ifthen}
\usepackage{longtable}
% \usepackage[center]{caption}


% —– Start: Angaben zur Formatierung von Überschriften —– %
\usepackage{titletoc}
% \titlecontents{Abschnitt}[Links]{Gesamtformatierung}{Vor dem %
% Label}{Nach dem Label}{Abstandsfüller & Seitenzahl}[Rechts]
%4.5 and 3.0
%7.5 and 3.2
%10.6 and 3.7
\titlecontents{subsection}[4.5em]{}{\contentslabel{3.0em}}{}{\titlerule*[0.6pc]{.}\contentspage}
\titlecontents{subsubsection}[7.5em]{}{\contentslabel{3.2em}}{}{\titlerule*[0.6pc]{.}\contentspage}
\titlecontents{paragraph}[10.6em]{}{\contentslabel{3.7em}}{}{\titlerule*[0.6pc]{.}\contentspage}
\titlecontents{subparagraph}[13.6em]{}{\contentslabel{4.1em}}{}{\titlerule*[0.6pc]{.}\contentspage}
% —– Ende: Angaben zur Formatierung von Überschriften —– %

\usepackage{titlesec}
\titlespacing*{\section}{0pt}{24pt}{4pt} %-1\parskip
\titlespacing*{\subsection}{0pt}{12pt}{4pt}
\titlespacing*{\subsubsection}{0pt}{10pt}{4pt}
\titlespacing*{\paragraph}{0pt}{10pt}{4pt}
\titlespacing*{\subparagraph}{0pt}{10pt}{4pt}

\setcounter{tocdepth}{3} %5
\setcounter{secnumdepth}{5} %5

\setparsizes{\parindent}{10pt}{\parfillskip}
 
\usepackage{etoolbox}
\appto\chapterheadendvskip{\vspace{-1\parskip}}


\makeatletter

\providecommand{\tabularnewline}{\\}


\def\ScaleIfNeededHeight{%
	\ifdim\Gin@nat@height>0.9\textheight
		0.9\textheight
	\else
		\Gin@nat@height
	\fi 
}

\def\ScaleIfNeededWidth{%
	\ifdim\Gin@nat@width > \linewidth
		linewidth
	\else
		\Gin@nat@width
	\fi 
}

\def\ConstForImageWidth{\textwidth}

%\newcommand{\loadImage}[1]{\IfFileExists{#1}{\begin{figure}[H] \begin{center} \includegraphics[height=\ScaleIfNeededHeight,width=\ConstForImageWidth, keepaspectratio]{#1} \end{center} \end{figure}}{\begin{figure}[!htb] \centering \includegraphics[height=\ScaleIfNeededHeight,width=\ConstForImageWidth, keepaspectratio]{noValues.pdf} \end{figure}}}
%\newcommand{\loadTex}[1]{\IfFileExists{#1}{\input{#1}}{\begin{figure}[H] \begin{center} \includegraphics[height=\ScaleIfNeededHeight,width=\ConstForImageWidth, keepaspectratio]{noValues.pdf} \end{center} \end{figure}}}
\newcommand{\loadImage}[1]{\IfFileExists{#1}{\begin{OwnLoadImage} \includegraphics[height=\ScaleIfNeededHeight,width=\ConstForImageWidth, keepaspectratio]{#1} \end{OwnLoadImage}}{\begin{OwnLoadImage} \includegraphics[height=\ScaleIfNeededHeight,width=\ConstForImageWidth, keepaspectratio]{noValues.pdf} \end{OwnLoadImage}}}
\newcommand{\loadImageCap}[2]{\IfFileExists{#1}
	{\begin{OwnLoadImageVSpace}
			\includegraphics[width=\ScaleIfNeededWidth,keepaspectratio]{#1} \caption{#2} \label{fig:#1}
			%[height=\ScaleIfNeededHeight,width=\ScaleIfNeededWidth,keepaspectratio]
	 \end{OwnLoadImageVSpace}
	}{\begin{OwnLoadImage}
		\includegraphics[height=\ScaleIfNeededHeight,width=\ScaleIfNeededWidth,
			keepaspectratio]{noValues.pdf} \caption{no image} 
	  \end{OwnLoadImage}}}

\newcommand{\loadTex}[1]{\IfFileExists{#1}{\input{#1}}{\begin{OwnLoadImage}
\includegraphics[height=\ScaleIfNeededHeight,width=\ConstForImageWidth, keepaspectratio]{noValues.pdf} \end{OwnLoadImage}}}

\newenvironment{OwnLoadImage}
  {\par\noindent\minipage{\textwidth}\centering}
  {\endminipage}

\newenvironment{OwnLoadImageVSpace}
  {\par\noindent\minipage{\textwidth}\vspace{0.3cm}\centering}
  {\vspace{0.3cm}\endminipage}

\newboolean{isFile1}

\newcommand{\resetBoolean}{
	\setboolean{isFile1}{false}
	}

\newcommand{\checkFileNoReset}[1]{
	\IfFileExists{#1}{\setboolean{isFile1}{true}}{}
	}

\newboolean{isClearPage}
\setboolean{isClearPage}{true}

\newboolean{isClearPageSub}
\setboolean{isClearPageSub}{true}

\newcommand{\resetClear}{
	\setboolean{isClearPage}{true}
}

\newcommand{\resetClearSub}{
	\setboolean{isClearPageSub}{true}
}


\newcommand{\ownClearPage}{

	\ifthenelse{\boolean{isClearPage}}{
		\setboolean{isClearPage}{false}
	}{
		\clearpage
	}
}

\newcommand{\ownClearPageSub}{

	\ifthenelse{\boolean{isClearPageSub}}{
		\setboolean{isClearPageSub}{false}
	}{
		\clearpage
	}
}

% \usepackage{array}
% \newcolumntype{C}[1]{>{\begin{center}}p{#1}<{\end{center}}}

% \newboolean{forceClearPage}
% \setboolean{forceClearPage}{true}
% 
% \newboolean{noForceClearPage}
% \setboolean{noForceClearPage}{false}

% \newcommand{\ownClearPage}[1]{
% 	
% 	\ifthenelse{#1}{
% 		\setboolean{isClearPage}{true}
% 		\clearpage	
% 	}{
% 		\ifthenelse{\boolean{isClearPage}}{
% 			\setboolean{isClearPage}{false}
% 		}{
% 			\setboolean{isClearPage}{true}
% 			\clearpage
% 		}
% 	}
% }
	

% \newenvironment{OwnLoadImage}
%   {\par\raggedbottom\null\vfill\noindent\minipage{\textwidth}\centering}
%   {\endminipage\par\vfill\vfill}
  
\renewcommand{\topfraction}{.85} % maximaler Abstand von Seitenoberseite, bis zu welchem Gleitumgebungen noch plaziert werden dürfen
\renewcommand{\bottomfraction}{.7} % maximaler Anteil welchen Gleitumgebungen am unteren Seitenrand einnehmen dürfen
\renewcommand{\textfraction}{.15} % Anteil einer Seite mit Gleitumgebungen, welcher mindestens von Text belegt sein muss -> ansonsten kein Text auf der Seite
\renewcommand{\floatpagefraction}{.66} % minimaler Seitenanteil welcher besetzt sein muss, bevor eine neue Seite für Gleitumgebungen angelegt wird
\setcounter{topnumber}{3} % maximale Anzahl Gleitobjekte am oberen Seitenrand
\setcounter{bottomnumber}{3} % maximale Anzahl Gleitobjekte am unteren Seitenrand
\setcounter{totalnumber}{6} % maximale Anzahl Gleitobjekte pro Seite

\makeatother
