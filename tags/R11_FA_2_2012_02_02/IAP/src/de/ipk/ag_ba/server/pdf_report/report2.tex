%% LyX 2.0.0 created this file.  For more info, see http://www.lyx.org/.
%% Do not edit unless you really know what you are doing.
\documentclass[oneside,english]{amsart}
\usepackage[T1]{fontenc}
\usepackage[latin9]{inputenc}
\usepackage{textcomp}
\usepackage{amsthm}
\usepackage{amstext}
\usepackage{graphicx}
\usepackage{here}

\makeatletter

%%%%%%%%%%%%%%%%%%%%%%%%%%%%%% LyX specific LaTeX commands.
\newcommand{\lyxmathsym}[1]{\ifmmode\begingroup\def\b@ld{bold}
  \text{\ifx\math@version\b@ld\bfseries\fi#1}\endgroup\else#1\fi}

%% Because html converters don't know tabularnewline
\providecommand{\tabularnewline}{\\}
%% A simple dot to overcome graphicx limitations
\newcommand{\lyxdot}{.}

\newcommand{\ImageFormat}{.png}


%%%%%%%%%%%%%%%%%%%%%%%%%%%%%% Textclass specific LaTeX commands.
\numberwithin{equation}{section}
\numberwithin{figure}{section}
\newenvironment{lyxlist}[1]
{\begin{list}{}
{\settowidth{\labelwidth}{#1}
 \setlength{\leftmargin}{\labelwidth}
 \addtolength{\leftmargin}{\labelsep}
 \renewcommand{\makelabel}[1]{##1\hfil}}}
{\end{list}}

\def\ScaleIfNeeded{%
	\ifdim\Gin@nat@height> 0.9\textheight
		0.9\textheight
	\else
		\Gin@nat@height
	\fi
}

\def\ConstForImageWidth{8cm}

\newcommand{\loadImage}[1]{\IfFileExists{#1}{\begin{figure}[H] \begin{center} \includegraphics[height=\ScaleIfNeeded,width=\ConstForImageWidth, keepaspectratio]{#1} \end{center} \end{figure}}{\begin{figure}[!h] \centering \includegraphics[height=\ScaleIfNeeded,width=\ConstForImageWidth, keepaspectratio]{noValues.pdf} \end{figure}}}
\newcommand{\loadTex}[1]{\IfFileExists{#1}{\include{#1}}{\begin{figure}[H] \begin{center} \includegraphics[height=\ScaleIfNeeded,width=\ConstForImageWidth, keepaspectratio]{noValues.pdf} \end{center} \end{figure}}}
\makeatother

\usepackage{babel}
\begin{document}

\title{Analysis of Experiment --experimentname--}


\author{Research Group Image Analysis\\ \\ IPK-Gatersleben\\ \\ C. Klukas \& A. Entzian}


\date{29.09.2011}


\address{Leibniz Institute of Plant Genetics and Crop Plant Research, Group
Image Analysis, Corrensstr. 3, 06466 Gatersleben, Germany}


\email{klukas@ipk-gatersleben.de, entzian@ipk-gatersleben.de}


\urladdr{http://ba-13.ipk-gatersleben.de}
\begin{abstract}
This document is automatically created based on the automated image analysis performed 
inside the integrated analysis platform IAP. Currently the pipelines and data analysis tools are in beta status, 
which means that we are working on makeing the analysis procedures, statistical operations, diagram output and report
generation more stable and suitable to the needs of any user of the automated imaging infrastructure at the IPK.

If you have any hints, comments or suggestions for improvements, please don't hesitate to contact any member
of the group image analysis.
\end{abstract}

\keywords{image analysis, phenotyping, maize }

\maketitle

\clearpage
\tableofcontents

\clearpage

\section{Experiment info}
\subsection{Experiment properties}

\begin{lyxlist}{00.00.0000}
\item [{%
\begin{tabular}{|c|c|c|}
\hline 
Experiment & --experimentname--\tabularnewline
\hline 
\hline 
Start & --StartExp--\tabularnewline
\hline 
End & --EndExp--\tabularnewline
\hline 
Numeric values & --NumExp-- \tabularnewline
\hline 
Images & --ImagesExp-- \tabularnewline
\hline 
Storage & --StorageExp-- \tabularnewline
\hline 
\end{tabular}}]~
\end{lyxlist}

\subsection{Remarks}

\begin{lyxlist}{00.00.0000}
\item [{%
\begin{tabular}{|c|c|}
\hline 
Notes\tabularnewline
\hline 
\hline 
--RemarkExp--\tabularnewline
\hline 
\end{tabular}}]~
\end{lyxlist}

\subsection{Experiment factors}

\begin{lyxlist}{00.00.0000}
\item [{%
\begin{tabular}{|c|c|c|c|c|c|}
\hline 
ID & Genotype & Variety & Treatment & Sequence & xyz\tabularnewline
\hline 
\hline 
--ID-- & --Genotype-- & --Variety-- & --Treatment-- & --Sequence-- & --xyz--\tabularnewline
\hline 
\end{tabular}}]~
\end{lyxlist}

%%\newpage
\clearpage
\section{\noindent Weight and water consumption}

\subsection{Weights (before and after watering)}
\begin{itemize}
\item Weight before and after watering.
\item Colnum name: Weight A (g), Weight B (g)
\item Unit: g
\end{itemize}
\loadImage{Weight_A__g_nboxplot.pdf}
\loadImage{Weight_B__g_nboxplot.pdf}



\subsection{Watering amounts / consumption}
\begin{itemize}
\item Consumption of water based of weight measurments.
\item Colnum name: Watering (weight-diff)
\item Unit: g
\end{itemize}
\loadImage{Water__weight_diff_nboxplot.pdf}


\clearpage
\section{\noindent Biomass}
\begin{itemize}
\item Digital Biomass based on the side.area and top.area
\item Unit: pixel\textthreesuperior{}
\end{itemize}

\subsection{IAP based formula - GRB}
\begin{itemize}
\item Equation: $Biomass_{IAP}=\sqrt{side.area_{average}^{2}*top.area_{average}}$
\item Column name: volume.iap
\end{itemize}
\loadImage{volume_iap__px3_boxplot.pdf}
\loadImage{volume_iap__px3_nboxplot.pdf}



\subsection{IAP based formula - GRB max}
\begin{itemize}
\item Equation: $Biomass_{LemnaTec}=\sqrt{side.area_{max\_image}^{2}*top.area_{max\_image}}$
\item Column name: volume.iap\_max
\end{itemize}
\loadImage{volume_iap_maxboxplot.pdf}
\loadImage{volume_iap_maxnboxplot.pdf}


\subsection{IAP based formula - FLUO}
\begin{itemize}
\item Equation: $Biomass_{IAP}=\sqrt{side.area_{average}^{2}*top.area_{average}}$
\item Column name: volume.fluo.iap
\end{itemize}
\loadImage{volume_fluo_iapboxplot.pdf}
\loadImage{volume_fluo_iapnboxplot.pdf}


\subsection{LemnaTec formula}
\begin{itemize}
\item Equation: $Biomass_{LemnaTec}=\sqrt{side.area_{0^{\circ}}*side.area_{90^{\circ}}*top.area}$
\item Column name: volume.lt
\end{itemize}
\loadImage{volume_lt__px3_boxplot.pdf}
\loadImage{volume_lt__px3_nboxplot.pdf}


\subsection{KeyGene formula}
\begin{itemize}
\item Equation: $Biomass_{KeyGene}=side.area_{0^{\circ}}+side.area_{90^{\circ}}+\log(\frac{top.area}{3})$
\item Column name: digital.biomass.keygene.norm
\end{itemize}

%\loadImage{digital_biomass_keygene_normnboxplot.pdf}


\clearpage
\section{\noindent Water use efficiency}
\begin{itemize}
\item Ratio of plant growth and water use.
\end{itemize}

\subsection{Volume based}
\begin{itemize}
\item The volume based on the IAP formula used for the value "plant growth".
\item Colnum name: volume.iap.wue
\item Unit: .
\end{itemize}

\loadImage{volume_iap_wuenboxplot.pdf}



\clearpage
\section{General growth related plant properties}

\subsection{Plant height}
\begin{itemize}
\item Plant height (mm)
\item Column name: side.height
\item Unit: mm
\end{itemize}
\loadImage{side_height__mm_boxplot.pdf}
\loadImage{side_height__mm_nboxplot.pdf}


\subsection{Normalized plant height}
\begin{itemize}
\item Plant height (mm) (normalized to distance of left and right marker)
\item Column name: side.height.norm
\item Unit: mm
\end{itemize}
\loadImage{side_height_norm__mm_boxplot.pdf}
\loadImage{side_height_norm__mm_nboxplot.pdf}


\subsection{Plant width}
\begin{itemize}
\item Plant width (mm) 
\item Column name: side.width
\item Unit: mm
\end{itemize}
\loadImage{side_width__mm_boxplot.pdf}
\loadImage{side_width__mm_nboxplot.pdf}


\subsection{Normalized plant width}
\begin{itemize}
\item Plant width (mm) (normalized to distance of left and right marker) 
\item Column name: side.width.norm
\item Unit: mm
\end{itemize}
\loadImage{side_width_norm__mm_boxplot.pdf}
\loadImage{side_width_norm__mm_nboxplot.pdf}


\subsection{Projected side area}
\begin{itemize}
\item Number of foreground pixels from side camera
\item Colum name: side.area
\item Unit: pixel\texttwosuperior{}
\end{itemize}
\loadImage{side_area__px_boxplot.pdf}
\loadImage{side_area__px_nboxplot.pdf}


\subsection{Normalized projected side area}
\begin{itemize}
\item Number of foreground pixels from side camera (normalized to distance of left and right marker)
\item Colum name: side.area.norm
\item Unit: pixel\texttwosuperior{}
\end{itemize}
\loadImage{side_area_norm__mm2_boxplot.pdf}
\loadImage{side_area_norm__mm2_nboxplot.pdf}


\subsection{Projected top area}
\begin{itemize}
\item Number of foreground pixels from top camera
\item Colum name: top.area
\item Unit: pixel\texttwosuperior{}
\end{itemize}
\loadImage{top_area__px_boxplot.pdf}
\loadImage{top_area__px_nboxplot.pdf}


\subsection{Normalized projected top area}
\begin{itemize}
\item Number of foreground pixels from top camera (normalized to distance of left and right marker)
\item Colum name: top.area.norm
\item Unit: pixel\texttwosuperior{}
\end{itemize}

\loadImage{top_area_norm__mm2_nboxplot.pdf}


\clearpage
\section{\noindent Relative values}
\begin{itemize}
\item Presented values in relative dependence.
\item Unit: \%
\end{itemize}

\subsection{Relative projected side area}
\begin{itemize}
\item Relative number of foreground pixels from side camera
\item Colnum name: side.area.relative
\end{itemize}

\loadImage{side_area_relativenboxplot.pdf}



\subsection{Relative plant height}
\begin{itemize}
\item Relative plant height (mm) (normalized to distance of left and right marker)
\item Colnum name: side.height.norm.relative
\end{itemize}

\loadImage{side_height_norm_relativenboxplot.pdf}


\subsection{Relative plant width}
\begin{itemize}
\item Relative plant width (mm) (normalized to distance of left and right marker) 
\item Colnum name: side.width.norm.relative
\end{itemize}

\loadImage{side_width_norm_relativenboxplot.pdf}


\subsection{Relative projected top area}
\begin{itemize}
\item Relative number of foreground pixels from top camera
\item Colnum name: top.area.relative
\end{itemize}

\loadImage{top_area_relativenboxplot.pdf}


\subsection{Relative projected side area}
\begin{itemize}
\item Relative number of foreground pixels from side camera
\item Colnum name: side.area.relative
\end{itemize}

\loadImage{side_area_relativenboxplot.pdf}


\subsection{Relative volume (IAP based formular - RGB)}
\begin{itemize}
\item Digital Biomass based on the side.area and top.area
\item Colnum name: volume.iap.relative
\end{itemize}

\loadImage{volume_iap_relativenboxplot.pdf}


\clearpage
\section{Fluorescence intensity analysis}

\subsection{Chlorophyl intensity}
\begin{itemize}
\item Average activity of chlorophyl, based on red reflection.
\item The darker the higher acitivity
\item Column name: side.fluo.intensity.chlorophyl.average
\item Unit: relative intensity/pixel
\end{itemize}

%%\loadImage{side_fluo_intensity_chlorophyl_average__relative_.pdf}
\loadTex{side_fluo_normalized_histogram_bin_1_0_25_side_fluo_normalized_histogrstackedOverallImage}


\subsection{Phenols and other fluorescence intensity}
\begin{itemize}
\item Average activity without chlorohyll (e. g. phenol), based of yellow
reflection.
\item The darker the higher acitivity
\item Column name: side.fluo.intensity.phenol.average
\item Unit: relative intensity/pixel
\end{itemize}

%\loadImage{side_fluo_intensity_phenol_average.pdf}


\subsection{Near-infrared intensity}
\begin{itemize}
\item Average intensity of near infrared (NIR)
\item Represents the water content of the plant
\item Column name: side.nir.intensity.average
\item Unit: relative intensity/pixel
\end{itemize}

%%\loadImage{side_nir_intensity_average__relative_.pdf}
\loadTex{side_nir_normalized_histogram_bin_1_0_25_side_nir_normalized_histogramstackedOverallImage}



\clearpage
\section{Plant structures}

\subsection{Number of leafs}
\begin{itemize}
\item Number of leafs-tips
\item Colum name: side.leaf.count.median
\item Unit: leafs
\item Hint: leaf-tips are shown as red retangles in the image
\end{itemize}

\loadImage{side_leaf_count_median__leafs_nboxplot.pdf}


\subsection{Tassel (for maize plants)}
\begin{itemize}
\item Male flower count
\item Colum name: side.bloom.count
\item Unit: tassel
\item Hint: the number of blue retangles in the image
\end{itemize}

\loadImage{side_bloom_count__tassel_nboxplot.pdf}


\subsection{Leaf lengths}
\begin{itemize}
\item Length of all leafs plus stem
\item Column name: side.leaf.length.sum.norm.max
\item Unit: mm
\item Hint: the yellow line in the image
\end{itemize}

\loadImage{side_leaf_length_sum_norm_max__mm_nboxplot.pdf}


\clearpage
\section{\noindent Wetness}
\begin{itemize}
\item Vales based on the NIR images
\item Unit: .
\end{itemize}

\subsection{Average wetness of side image}
\begin{itemize}
\item Average wetness of the plants from NIR side camera
\item Column name: side.nir.wetness.av
\item Unit: .
\end{itemize}

\loadImage{side_nir_wetness_av.pdf}


\subsection{Average wetness of top image}
\begin{itemize}
\item Average wetness of the plants from NIR top camera
\item Column name: top.nir.wetness.av
\item Unit: .
\end{itemize}

\loadImage{top_nir_wetness_avnboxplot.pdf}


\subsection{Weighted loss through drought stress - side image}
\begin{itemize}
\item Number of foreground pixels from NIR side camera minus the weighted value of the plant
\item weightOfPlant = fully wet: 1 unit, fully dry: 1/7 unit
\item Column name: side.nir.wetness.plant\_weight\_drought\_loss
\item Unit: .
\end{itemize}

\loadImage{side_nir_wetness_plant_weight_drought_lossnboxplot.pdf}



\subsection{Weighted loss through drought stress - top image}
\begin{itemize}
\item Number of foreground pixels from NIR top camera minus the weighted value of the plant
\item weightOfPlant = fully wet: 1 unit, fully dry: 1/7 unit
\item Column name: top.nir.wetness.plant\_weight\_drought\_loss
\item Unit: .
\end{itemize}

\loadImage{top_nir_wetness_plant_weight_drought_lossnboxplot.pdf}


\clearpage
\section{Appendix}

\loadTex{appendixImage}



\end{document}
