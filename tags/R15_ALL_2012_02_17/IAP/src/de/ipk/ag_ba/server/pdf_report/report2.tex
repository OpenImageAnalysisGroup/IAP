%%%%%%%%%%%%% Klassen-Optionen
\documentclass[%
  paper=a4, % Stellt auf A4-Papier ein
  pagesize, % Diese Option reicht die Papiergröße an
            % alle Ausgabeformate weiter
  DIV=calc, % Errechnet einen guten Satzspiegel
%  BCOR=1cm, % Weil bei mir immer 1cm in der Bindung
            % der Klemmmappe verschwindet
  headings=small,% Für etwas kleinere Überschriften
  english,   % englische Rechtschreibung d.\,h. (Silbentrennung)
  oneside
]{scrartcl}  % Eine Klasse für einseitige Texte mit Kapiteln  
            %
% \documentclass[oneside,english]{article}
                  
%%\documentclass[oneside,english]{scrreprt}
\usepackage[T1]{fontenc}
\usepackage[latin9]{inputenc} %ermöglicht die direkte Eingabe von Sonderzeichen
%%\usepackage[utf8]{inputenc} %ermöglicht die direkte Eingabe von Sonderzeichen
\usepackage{fixltx2e}
\usepackage[english]{babel} % Anpassung der Überschriften (chapter, chapitre, Kapitel …) und die richtige Silbentrennung
\usepackage{hyperref}
\usepackage[left=2.5cm,right=2.5cm,top=2.5cm,bottom=2.5cm]{geometry}

%%%%%%%%%%%%% Typografisch empfehlenswerte Pakete
\usepackage{% 
  ellipsis, % Korrigiert den Weißraum um Auslassungspunkte
  ragged2e, % Ermöglicht Flattersatz mit Silbentrennung
 marginnote,% Für bessere Randnotizen mit \marginnote statt
            % \marginline 
}
\usepackage[tracking=true]{microtype}% 
            % Microtype ist einfach super, aber lesen Sie
            % unbedingt die Anleitung um das Folgende zu
            % verstehen.
\DeclareMicrotypeSet*[tracking]{my}% 
  { font = */*/*/sc/* }% 
\SetTracking{ encoding = *, shape = sc }{ 45 }% Hier wird festgelegt,
            % dass alle Passagen in Kapitälchen automatisch leicht
            % gesperrt werden. Das Paket soul, das ich früher empfohlen
            % habe ist damit für diese Zwecke nicht mehr nötig.
            %

\usepackage{%
  lmodern, % A) Latin Modern Fonts sind die Nachfolger von Computer
            % Modern, den LaTeX-Standardfonts
%  hfoldsty % B) Diese Schrift stellt alle Ziffern, außer
            % im Mathemodus, auf Minuskel- oder Mediäval-Ziffern um.
            % Wenn Ihre pdfs unscharf aussehen installieren Sie bitte
            % die cm-super-Fonts (Type1-Fonts).
% charter   % C) Diese Zeile lädt die Charter als Schriftart
}
% \usepackage[osf,sc]{mathpazo}% D) So erreichen Sie Palatino als
            % Schrift mit Minuskel-Ziffern und echten Kapitälchen
            %

%\usepackage{here}
%\usepackage{scrpage2} 
%\usepackage{tocloft}
\usepackage{graphicx}
%\usepackage{amsmath} 
\usepackage{ifthen}

\makeatletter

\providecommand{\tabularnewline}{\\}

% \numberwithin{equation}{section}
% \numberwithin{figure}{section}

\def\ScaleIfNeeded{%
	\ifdim\Gin@nat@height> 0.9\textheight
		0.9\textheight
	\else
		\Gin@nat@height
	\fi 
}


\def\ConstForImageWidth{\textwidth}

%\newcommand{\loadImage}[1]{\IfFileExists{#1}{\begin{figure}[H] \begin{center} \includegraphics[height=\ScaleIfNeeded,width=\ConstForImageWidth, keepaspectratio]{#1} \end{center} \end{figure}}{\begin{figure}[!htb] \centering \includegraphics[height=\ScaleIfNeeded,width=\ConstForImageWidth, keepaspectratio]{noValues.pdf} \end{figure}}}
%\newcommand{\loadTex}[1]{\IfFileExists{#1}{\input{#1}}{\begin{figure}[H] \begin{center} \includegraphics[height=\ScaleIfNeeded,width=\ConstForImageWidth, keepaspectratio]{noValues.pdf} \end{center} \end{figure}}}
\newcommand{\loadImage}[1]{\IfFileExists{#1}{\begin{OwnLoadImage} \includegraphics[height=\ScaleIfNeeded,width=\ConstForImageWidth, keepaspectratio]{#1} \end{OwnLoadImage}}{\begin{OwnLoadImage} \includegraphics[height=\ScaleIfNeeded,width=\ConstForImageWidth, keepaspectratio]{noValues.pdf} \end{OwnLoadImage}}}
\newcommand{\loadTex}[1]{\IfFileExists{#1}{\input{#1}}{\begin{OwnLoadImage} \includegraphics[height=\ScaleIfNeeded,width=\ConstForImageWidth, keepaspectratio]{noValues.pdf} \end{OwnLoadImage}}}

\newenvironment{OwnLoadImage}
  {\par\noindent\minipage{\textwidth}\centering}
  {\endminipage}

\newboolean{isFile1}
\newboolean{isFile2}


\newcommand{\checkFile}[2]{
	\setboolean{isFile1}{false}
	\setboolean{isFile2}{false}
	\IfFileExists{#1}{\setboolean{isFile1}{true}}{}
	\IfFileExists{#2}{\setboolean{isFile2}{true}}{}
	}

% \newenvironment{OwnLoadImage}
%   {\par\raggedbottom\null\vfill\noindent\minipage{\textwidth}\centering}
%   {\endminipage\par\vfill\vfill}
  
\renewcommand{\topfraction}{.85} % maximaler Abstand von Seitenoberseite, bis zu welchem Gleitumgebungen noch plaziert werden dürfen
\renewcommand{\bottomfraction}{.7} % maximaler Anteil welchen Gleitumgebungen am unteren Seitenrand einnehmen dürfen
\renewcommand{\textfraction}{.15} % Anteil einer Seite mit Gleitumgebungen, welcher mindestens von Text belegt sein muss -> ansonsten kein Text auf der Seite
\renewcommand{\floatpagefraction}{.66} % minimaler Seitenanteil welcher besetzt sein muss, bevor eine neue Seite für Gleitumgebungen angelegt wird
\setcounter{topnumber}{3} % maximale Anzahl Gleitobjekte am oberen Seitenrand
\setcounter{bottomnumber}{3} % maximale Anzahl Gleitobjekte am unteren Seitenrand
\setcounter{totalnumber}{6} % maximale Anzahl Gleitobjekte pro Seite

 
\makeatother


\begin{document}


\title{Experiment --experimentname--}
\author{Research Group Image Analysis\tabularnewline \tabularnewline
IPK-Gatersleben\tabularnewline \tabularnewline C. Klukas \& A. Entzian}
\date{\today}

\maketitle
\thispagestyle{empty}  
\begin{abstract}
This document is automatically created based on the automated image analysis performed 
inside the integrated analysis platform IAP. Currently the pipelines and data analysis tools are in beta status, 
which means that we are working on makeing the analysis procedures, statistical operations, diagram output and report
generation more stable and suitable to the needs of any user of the automated imaging infrastructure at the IPK.

If you have any hints, comments or suggestions for improvements, please don't hesitate to contact any member
of the group image analysis.
\end{abstract}
\vfill
\small{\textit{E-mail address:} klukas@ipk-gatersleben.de, entzian@ipk-gatersleben.de}
\newline 
\small{\textit{Key words and phrases:} image analysis, phenotyping, maize }
 

\clearpage
\tableofcontents

\clearpage
\pagestyle{headings}
\section{Experiment info} 
\subsection{Experiment properties}


\begin{tabular}{|c|c|c|}
\hline 
Experiment & --experimentname--\tabularnewline
\hline 
\hline 
Start & --StartExp--\tabularnewline
\hline 
End & --EndExp--\tabularnewline
\hline 
Numeric values & --NumExp-- \tabularnewline
\hline 
Images & --ImagesExp-- \tabularnewline
\hline 
Storage & --StorageExp-- \tabularnewline
\hline 
\end{tabular}


\subsection{Remarks}


\begin{tabular}{|c|c|}
\hline 
Notes\tabularnewline
\hline 
\hline 
--RemarkExp--\tabularnewline
\hline 
\end{tabular}

\subsection{Experiment factors}


\begin{tabular}{|c|c|c|c|c|c|}
\hline 
ID & Genotype & Variety & Treatment & Sequence & xyz\tabularnewline
\hline 
\hline 
--ID-- & --Genotype-- & --Variety-- & --Treatment-- & --Sequence-- & --xyz--\tabularnewline
\hline 
\end{tabular}

\clearpage
\section{\noindent Weight and water consumption}


\checkFile{Weight_A__g_nboxplot.pdf}{Weight_B__g_nboxplot.pdf}
\ifthenelse{\boolean{isFile1} \or \boolean{isFile2}}{
	\subsection{Weights (before and after watering)}
	\begin{itemize}
	\item Weight before and after watering.
	\item Colnum name: Weight A (g), Weight B (g)
	\item Unit: g
	\end{itemize}
	\ifthenelse{\boolean{isFile1}}{
		\loadImage{Weight_A__g_nboxplot.pdf}
	}{}
	\ifthenelse{\boolean{isFile2}}{
		\loadImage{Weight_B__g_nboxplot.pdf}
	}{}
}{}


\checkFile{Water__weight_diff_nboxplot.pdf}{}
\ifthenelse{\boolean{isFile1}}{
	\subsection{Watering amounts / consumption}
	\begin{itemize}
	\item Consumption of water based of weight measurments.
	\item Colnum name: Watering (weight-diff)
	\item Unit: g
	\end{itemize}
	\loadImage{Water__weight_diff_nboxplot.pdf}
}{}


\clearpage
\section{\noindent Digital biomass}
\begin{itemize}
\item Digital Biomass based on the side.area and top.area
\item Unit: $pixel^2$
\end{itemize}

\checkFile{volume_iap__px3_boxplot.pdf}{volume_iap__px3_nboxplot.pdf}
\ifthenelse{\boolean{isFile1} \or \boolean{isFile2}}{
	\subsection{IAP formula}
	\begin{itemize}
	\item Equation: $Biomass_{IAP}=\sqrt{side.area_{average}^{2}*top.area_{average}}$
	\item Column name: volume.iap
	\end{itemize}
	\ifthenelse{\boolean{isFile1}}{
		\loadImage{volume_iap__px3_boxplot.pdf}
	}{}
	\ifthenelse{\boolean{isFile2}}{
		\loadImage{volume_iap__px3_nboxplot.pdf}
	}{}
}{}


\checkFile{volume_iap__px3_boxplot.pdf}{volume_iap__px3_nboxplot.pdf}
\ifthenelse{\boolean{isFile1} \or \boolean{isFile2}}{
	\subsection{LemnaTec formula}
	\begin{itemize}
	\item Equation: $Biomass_{LemnaTec}=\sqrt{side.area_{0^{\circ}}*side.area_{90^{\circ}}*top.area}$
	\item Column name: volume.lt
	\end{itemize}
	\ifthenelse{\boolean{isFile1}}{
		\loadImage{volume_lt__px3_boxplot.pdf}
	}{}
	\ifthenelse{\boolean{isFile2}}{
		\loadImage{volume_lt__px3_nboxplot.pdf}
	}{}
}{}


\checkFile{digital_biomass_keygene_normnboxplot.pdf}{}
\ifthenelse{\boolean{isFile1}}{
	\subsection{KeyGene formula}
	\begin{itemize}
	\item Equation: $Biomass_{KeyGene}=side.area_{0^{\circ}}+side.area_{90^{\circ}}+\log(\frac{top.area}{3})$
	\item Column name: digital.biomass.keygene.norm
	\end{itemize}
	\loadImage{digital_biomass_keygene_normnboxplot.pdf}
}{}


\clearpage
\section{\noindent Water use efficiency}
\begin{itemize}
\item Ratio of plant growth and water use.
\end{itemize}

\checkFile{volume_iap_wuenboxplot.pdf}{}
\ifthenelse{\boolean{isFile1}}{
	\subsection{Based on digital biomass}
	\begin{itemize}
	\item Growth per day divided by water usage per day.
	\item Colnum name: volume.iap.wue
	\item Unit: .
	\end{itemize}
	\loadImage{volume_iap_wuenboxplot.pdf}
}{}


\clearpage
\section{General growth related plant properties}

\checkFile{side_height__mm_boxplot.pdf}{side_height__mm_nboxplot.pdf}
\ifthenelse{\boolean{isFile1} \or \boolean{isFile2}}{
	\subsection{Height}
	\begin{itemize}
	\item Plant height (px)
	\item Column name: side.height
	\item Unit: px
	\end{itemize}
	\ifthenelse{\boolean{isFile1}}{
		\loadImage{side_height__mm_boxplot.pdf}
	}{}
	\ifthenelse{\boolean{isFile2}}{
		\loadImage{side_height__mm_nboxplot.pdf}
	}{}
}{}


\checkFile{side_height_norm__mm_boxplot.pdf}{side_height_norm__mm_nboxplot.pdf}
\ifthenelse{\boolean{isFile1} \or \boolean{isFile2}}{
	\subsection{Height (zoom corrected)}
	\begin{itemize}
	\item Plant height (mm) (normalized to distance of left and right marker)
	\item Column name: side.height.norm
	\item Unit: mm
	\end{itemize}
	\ifthenelse{\boolean{isFile1}}{
		\loadImage{side_height_norm__mm_boxplot.pdf}
	}{}	
	\ifthenelse{\boolean{isFile2}}{	
		\loadImage{side_height_norm__mm_nboxplot.pdf}
	}{}
}{}


\checkFile{side_width__mm_boxplot.pdf}{side_width__mm_nboxplot.pdf}
\ifthenelse{\boolean{isFile1} \or \boolean{isFile2}}{
	\subsection{Width}
	\begin{itemize}
	\item Plant width (px) 
	\item Column name: side.width
	\item Unit: px
	\end{itemize}
	\ifthenelse{\boolean{isFile1}}{
		\loadImage{side_width__mm_boxplot.pdf}
	}{}
	\ifthenelse{\boolean{isFile2}}{	
		\loadImage{side_width__mm_nboxplot.pdf}
	}{}
}{}


\checkFile{side_width_norm__mm_boxplot.pdf}{side_width_norm__mm_nboxplot.pdf}
\ifthenelse{\boolean{isFile1} \or \boolean{isFile2}}{
	\subsection{Width (zoom corrected)}
	\begin{itemize}
	\item Plant width (mm) (normalized to distance of left and right marker) 
	\item Column name: side.width.norm
	\item Unit: mm
	\end{itemize}
	\ifthenelse{\boolean{isFile1}}{
		\loadImage{side_width_norm__mm_boxplot.pdf}
	}{}
	\ifthenelse{\boolean{isFile2}}{	
		\loadImage{side_width_norm__mm_nboxplot.pdf}
	}{}
}{}


\checkFile{side_area__px_boxplot.pdf}{side_area__px_nboxplot.pdf}
\ifthenelse{\boolean{isFile1} \or \boolean{isFile2}}{
	\subsection{Projected side area}
	\begin{itemize}
	\item Number of foreground pixels
	\item Colum name: side.area
	\item Unit: px
	\end{itemize}
	\ifthenelse{\boolean{isFile1}}{
		\loadImage{side_area__px_boxplot.pdf}
	}{}
	\ifthenelse{\boolean{isFile2}}{
		\loadImage{side_area__px_nboxplot.pdf}
	}{}
}{}


\checkFile{side_area_norm__mm2_boxplot.pdf}{side_area_norm__mm2_nboxplot.pdf}
\ifthenelse{\boolean{isFile1} \or \boolean{isFile2}}{
	\subsection{Projected side area (zoom corrected)}
	\begin{itemize}
	\item Number of foreground pixels from side camera (normalized to distance of left and right marker)
	\item Colum name: side.area.norm
	\item Unit: $mm^2$
	\end{itemize}
	\ifthenelse{\boolean{isFile1}}{
		\loadImage{side_area_norm__mm2_boxplot.pdf}
	}{}
	\ifthenelse{\boolean{isFile2}}{
		\loadImage{side_area_norm__mm2_nboxplot.pdf}
	}{}
}{}


\checkFile{top_area__px_boxplot.pdf}{top_area__px_nboxplot.pdf}
\ifthenelse{\boolean{isFile1} \or \boolean{isFile2}}{
	\subsection{Projected top area}
	\begin{itemize}
	\item Number of foreground pixels
	\item Colum name: top.area
	\item Unit: px
	\end{itemize}
	\ifthenelse{\boolean{isFile1}}{
		\loadImage{top_area__px_boxplot.pdf}
	}{}
	\ifthenelse{\boolean{isFile2}}{
		\loadImage{top_area__px_nboxplot.pdf}
	}{}
}{}


\checkFile{top_area_norm__mm2_nboxplot.pdf}{}
\ifthenelse{\boolean{isFile1}}{
	\subsection{Projected top area (zoom corrected)}
	\begin{itemize}
	\item Number of foreground pixels from top camera (normalized to distance of left and right marker)
	\item Colum name: top.area.norm
	\item Unit: $mm^2$
	\end{itemize}
	\loadImage{top_area_norm__mm2_nboxplot.pdf}
}{}

\clearpage
\section{\noindent Relative changes per day}
\begin{itemize}
\item Presented values in relative dependence.
\item Unit: \%
\end{itemize}


\checkFile{side_area_relativenboxplot.pdf}{}
\ifthenelse{\boolean{isFile1}}{
	\subsection{Projected side area}
	\begin{itemize}
	\item Relative growth of number of foreground pixels from side camera
	\item Colnum name: side.area.relative
	\end{itemize}
	\loadImage{side_area_relativenboxplot.pdf}
}{}


\checkFile{side_height_norm_relativenboxplot.pdf}{}
\ifthenelse{\boolean{isFile1}}{
	\subsection{Height}
	\begin{itemize}
	\item Relative growth of plant height in percent (normalized to distance of left and right marker)
	\item Colnum name: side.height.norm.relative
	\end{itemize}
	\loadImage{side_height_norm_relativenboxplot.pdf}
}{}


\checkFile{side_width_norm_relativenboxplot.pdf}{}
\ifthenelse{\boolean{isFile1}}{
	\subsection{Width}
	\begin{itemize}
	\item Relative plant width growth in percent (normalized to distance of left and right marker) 
	\item Colnum name: side.width.norm.relative
	\end{itemize}
	\loadImage{side_width_norm_relativenboxplot.pdf}
}{}

\checkFile{top_area_relativenboxplot.pdf}{}
\ifthenelse{\boolean{isFile1}}{
	\subsection{Projected top area}
	\begin{itemize}
	\item Relative growth of number of foreground pixels from top camera
	\item Colnum name: top.area.relative
	\end{itemize}
	\loadImage{top_area_relativenboxplot.pdf}
}{}


\checkFile{side_area_relativenboxplot.pdf}{}
\ifthenelse{\boolean{isFile1}}{
	\subsection{Projected side area}
	\begin{itemize}
	\item Relative growth of number of foreground pixels from side camera
	\item Colnum name: side.area.relative
	\end{itemize}
	\loadImage{side_area_relativenboxplot.pdf}
}{}

\checkFile{volume_iap_relativenboxplot.pdf}{}
\ifthenelse{\boolean{isFile1}}{
	\subsection{Digital biomass (IAP formula)}
	\begin{itemize}
	\item Digital biomass growth in percent (per day) based on the side.area and top.area observed from the visible light camera.
	\item Colnum name: volume.iap.relative
	\end{itemize}
	\loadImage{volume_iap_relativenboxplot.pdf}
}{}


\clearpage
\section{VIS ligth color histogram}


\checkFile{side_vis_hue_histogram_ratio_bin_1_0_25_side_vis_hue_histogram_ratio_bstackedOverallImage.tex}{top_vis_hue_histogram_ratio_bin_1_0_25_top_vis_hue_histogram_ratio_binstackedOverallImage.tex}
\ifthenelse{\boolean{isFile1} \or \boolean{isFile2}}{
	\subsection{Color histogram}
	\begin{itemize}
	\item Average activity of the normalized VIS images
	\item The darker the higher acitivity
	\item Column name: xxx
	\item Unit: %
	\end{itemize}
	\ifthenelse{\boolean{isFile1}}{
		\loadTex{side_vis_hue_histogram_ratio_bin_1_0_25_side_vis_hue_histogram_ratio_bstackedOverallImage}
	}{}	
	\ifthenelse{\boolean{isFile2}}{	
		\loadTex{top_vis_hue_histogram_ratio_bin_1_0_25_top_vis_hue_histogram_ratio_binstackedOverallImage}
	}{}
}{}


\checkFile{side_vis_normalized_histogram_ratio_bin_1_0_25_side_vis_normalized_his.tex}{}
\ifthenelse{\boolean{isFile1}}{
	\subsection{Color histogram (zoom corrected)}
	\begin{itemize}
	\item Average activity of the normalized VIS images
	\item The darker the higher acitivity
	\item Column name: xxx
	\item Unit: %
	\end{itemize}
	\loadTex{side_vis_normalized_histogram_ratio_bin_1_0_25_side_vis_normalized_his}
	}{}


\clearpage
\section{Fluorescence activity histogram}


\checkFile{side_fluo_histogram_bin_1_0_25_side_fluo_histogram_bin_2_25_51_side_flstackedOverallImage.tex}{top_fluo_histogram_bin_1_0_25_top_fluo_histogram_bin_2_25_51_top_fluo_stackedOverallImage.tex}
\ifthenelse{\boolean{isFile1} \or \boolean{isFile2}}{
	\subsection{Fluorescence spectra}
	\begin{itemize}
	\item Histogram of observed fluorescence colors
	\end{itemize}
	\ifthenelse{\boolean{isFile1}}{
		\loadTex{side_fluo_histogram_bin_1_0_25_side_fluo_histogram_bin_2_25_51_side_flstackedOverallImage}
	}{}
	\ifthenelse{\boolean{isFile2}}{	
		\loadTex{top_fluo_histogram_bin_1_0_25_top_fluo_histogram_bin_2_25_51_top_fluo_stackedOverallImage}
	}{}
}{}


\checkFile{side_fluo_normalized_histogram_bin_1_0_25_side_fluo_normalized_histogrstackedOverallImage.pdf}{}
\ifthenelse{\boolean{isFile1}}{
	\subsection{Fluorescence spectra (zoom corrected)}
	\begin{itemize}
	\item Histogram of observed fluorescence colors
	\end{itemize}
	\loadTex{side_fluo_normalized_histogram_bin_1_0_25_side_fluo_normalized_histogrstackedOverallImage}
}{}

\checkFile{side_fluo_histogram_ratio_bin_1_0_25_side_fluo_histogram_ratio_bin_2_2stackedOverallImage.tex}{top_fluo_histogram_ratio_bin_1_0_25_top_fluo_histogram_ratio_bin_2_25_stackedOverallImage.tex}
\ifthenelse{\boolean{isFile1} \or \boolean{isFile2}}{
	\subsection{Fluo ratio intensity (to do: check!!!!!)}
	\begin{itemize}
	\item xxx
	\item The darker the higher acitivity
	\item Column name: xxx
	\item Unit: %
	\end{itemize}
	\ifthenelse{\boolean{isFile1}}{
		\loadTex{side_fluo_histogram_ratio_bin_1_0_25_side_fluo_histogram_ratio_bin_2_2stackedOverallImage}
	}{}
	\ifthenelse{\boolean{isFile2}}{	
		\loadTex{top_fluo_histogram_ratio_bin_1_0_25_top_fluo_histogram_ratio_bin_2_25_stackedOverallImage}
	}{}
}{}


\checkFile{side_fluo_normalized_histogram_ratio_bin_1_0_25_side_fluo_normalized_hstackedOverallImage.tex}{}
\ifthenelse{\boolean{isFile1}}{
	\subsection{Fluo ratio intensity (zoom corrected) (to do: check!!!!!)}
	\begin{itemize}
	\item xxx
	\item The darker the higher acitivity
	\item Column name: xxx
	\item Unit: %
	\end{itemize}
	\loadTex{side_fluo_normalized_histogram_ratio_bin_1_0_25_side_fluo_normalized_hstackedOverallImage}
}{}


\clearpage
\section{Near-infrared intensity histogram}


\checkFile{side_nir_histogram_bin_1_0_25_side_nir_histogram_bin_2_25_51_side_nhisstackedOverallImage.tex}{top_nir_histogram_bin_1_0_25_top_nir_histogram_bin_2_25_51_top_nir_hisstackedOverallImage.tex}
\ifthenelse{\boolean{isFile1} \or \boolean{isFile2}}{
	\subsection{Intensity histogram}
	\begin{itemize}
	\item Average intensity of near infrared (NIR)
	\item Represents the water content of the plant
	\item Column name: xxx
	\item Unit: %
	\end{itemize}
	\ifthenelse{\boolean{isFile1}}{
		\loadTex{side_nir_histogram_bin_1_0_25_side_nir_histogram_bin_2_25_51_side_nhisstackedOverallImage}
	}{}
	\ifthenelse{\boolean{isFile2}}{
		\loadTex{top_nir_histogram_bin_1_0_25_top_nir_histogram_bin_2_25_51_top_nir_hisstackedOverallImage}
	}{}
}{}


\checkFile{side_nir_normalized_histogram_bin_1_0_25_side_nir_normalized_histogramstackedOverallImage.tex}{}
\ifthenelse{\boolean{isFile1}}{
	\subsection{Intensity histogram (zoom corrected)}
	\begin{itemize}
	\item Average intensity of normalized near infrared (NIR)
	\item Represents the water content of the plant
	\item Column name: xxx
	\item Unit: %
	\end{itemize}
	\loadTex{side_nir_normalized_histogram_bin_1_0_25_side_nir_normalized_histogramstackedOverallImage}
}{}


\clearpage
\section{Plant structures}
% http://upload.wikimedia.org/wikipedia/commons/9/98/Maize_plant_diagram.svg


\checkFile{side_leaf_count_median__leafs_nboxplot.pdf}{}
\ifthenelse{\boolean{isFile1}}{
	\subsection{Number of leafs}
	\begin{itemize}
	\item Number of leafs-tips
	\item Colum name: side.leaf.count.median
	\item Unit: leafs
	\item Hint: leaf-tips are shown as red retangles in the image
	\end{itemize}
	\loadImage{side_leaf_count_median__leafs_nboxplot.pdf}
}{}


\checkFile{side_leaf_length_sum_norm_max__mm_nboxplot.pdf}{}
\ifthenelse{\boolean{isFile1}}{
	\subsection{Leaf lengths}
	\begin{itemize}
	\item Length of all leafs plus stem
	\item Column name: side.leaf.length.sum.norm.max
	\item Unit: mm
	\item Hint: the yellow line in the image
	\end{itemize}
	\loadImage{side_leaf_length_sum_norm_max__mm_nboxplot.pdf}
}{}


\checkFile{side_bloom_count__tassel_nboxplot.pdf}{}
\ifthenelse{\boolean{isFile1}}{	
	\subsection{Flower detection}
	\begin{itemize}
	\item Number of tassel florets
	\item Colum name: side.bloom.count
	\item Hint: the number of blue retangles in the image
	\end{itemize}
	\loadImage{side_bloom_count__tassel_nboxplot.pdf}
}{}

\clearpage
\section{\noindent Wetness}

\begin{itemize}
\item Vales based on the NIR images
\end{itemize}

\checkFile{side_nir_wetness_average__percent_.pdf}{}
\ifthenelse{\boolean{isFile1}}{
	\subsection{Average wetness of side image}
	\begin{itemize}
	\item Average wetness of the plants from NIR side camera
	\item Column name: side.nir.wetness.av
	\item Unit: .
	\end{itemize}
	\loadImage{side_nir_wetness_average__percent_.pdf}
}{}


\checkFile{top_nir_wetness_average__percent_nboxplot.pdf}{}
\ifthenelse{\boolean{isFile1}}{	
	\subsection{Average wetness of top image}
	\begin{itemize}
	\item Average wetness of the plants from NIR top camera
	\item Column name: top.nir.wetness.av
	\item Unit: .
	\end{itemize}
	\loadImage{top_nir_wetness_average__percent_nboxplot.pdf}
}{}	
	
	
\checkFile{side_nir_wetness_plant_weight_drought_lossnboxplot.pdf}{}
\ifthenelse{\boolean{isFile1}}{	
	\subsection{Weighted loss through drought stress - side image}
	\begin{itemize}
	\item Number of foreground pixels from NIR side camera minus the weighted value of the plant
	\item weightOfPlant = fully wet: 1 unit, fully dry: 1/7 unit
	\item Column name: side.nir.wetness.plant\_weight\_drought\_loss
	\item Unit: .
	\end{itemize}
	\loadImage{side_nir_wetness_plant_weight_drought_lossnboxplot.pdf}
}{}


\checkFile{top_nir_wetness_plant_weight_drought_lossnboxplot.pdf}{}
\ifthenelse{\boolean{isFile1}}{
	\subsection{Weighted loss through drought stress - top image}
	\begin{itemize}
	\item Number of foreground pixels from NIR top camera minus the weighted value of the plant
	\item weightOfPlant = fully wet: 1 unit, fully dry: 1/7 unit
	\item Column name: top.nir.wetness.plant\_weight\_drought\_loss
	\item Unit: .
	\end{itemize}
	\loadImage{top_nir_wetness_plant_weight_drought_lossnboxplot.pdf}
}{}


\checkFile{appendixImage.tex}{}
\ifthenelse{\boolean{isFile1}}{
	\clearpage
	\section{Appendix}
	\loadTex{appendixImage}
}{}


\vfill
\small{Leibniz Institute of Plant Genetics and Crop Plant Research, Group Image Analysis, Corrensstr. 3, 06466 Gatersleben, Germany}
\newline
\small{\textit{URL:} http://ba-13.ipk-gatersleben.de}
\end{document}