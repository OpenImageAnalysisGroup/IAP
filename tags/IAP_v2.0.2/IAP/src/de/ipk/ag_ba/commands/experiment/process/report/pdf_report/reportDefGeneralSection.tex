% 
% Author: Entzian, Klukas (bug-fixing)
%%%%%%%%%%%%%%%%%%%%%%%%%%%%%%%%%%%%%%%%%%%%%%%%%%%%%%%%%%%%%%%%%%%%%%%%%%%%%%%%%%%%%%%%%%%%


\begin{document}


\title{--experimentname-- \tabularnewline \vspace{10 mm} \large{Experiment coordinator: --coordinator--}}

\author{- Automated image analysis report created with IAP - \tabularnewline Research Group Image Analysis, IPK-Gatersleben}

\date{\today ~(\currenttime )}

\maketitle
\thispagestyle{empty}  
\begin{abstract}
This document is automatically created based on the automated image analysis performed inside the integrated analysis platform IAP. Currently the pipelines and data analysis tools are in beta status, 
which means that we are working on makeing the analysis procedures, statistical operations, diagram output and report
generation more stable and suitable to the needs of any user of the automated imaging infrastructure at the IPK.

If you have any hints, comments or suggestions for improvements, please don't hesitate to contact any member of the group image analysis.
\end{abstract}
\vfill
\small{\textit{E-mail address:} klukas@ipk-gatersleben.de}
\newline 
\small{\textit{Key words and phrases:} image analysis, phenotyping, \textit{Arabidopsis}, barley, maize }
 
\addtocounter{page}{-1}
\clearpage
\tableofcontents

\clearpage
\pagestyle{headings}
\section{Experiment info} 
\subsection{Experiment properties}
\begin{center}
	\begin{tabular}{|p{3cm}|p{13cm}|}
	\hline
	{\textbf{Experiment}} & --experimentnameShort--\tabularnewline
	\hline
	\hline
	{\textbf{End experiment}} & --EndExp--\tabularnewline
	\hline
	{\textbf{Start analysis}} & --StartExp--\tabularnewline
	\hline
	{\textbf{End analysis}} & --StorageDate--\tabularnewline
	\hline
	{\textbf{Numeric values}} & --NumExp-- \tabularnewline
	\hline
	{\textbf{Images}} & --ImagesExp-- \tabularnewline
	\hline
	{\textbf{Storage}} & --StorageExp-- \tabularnewline
	\hline
	{\textbf{Sequence}} & --SequenceEXP-- \tabularnewline
	\hline
	{\textbf{Defined Outliers}} & --Outliers-- \tabularnewline
	\hline
	\hline 
	\end{tabular}
\end{center}

\subsection{Remarks}
\begin{center}
	\begin{tabular}{|p{16.5cm}|}
	\hline 
	{\textbf{Notes}}\tabularnewline
	\hline 
	\hline 
	--RemarkExp--\tabularnewline
	\hline
	\hline  
	\end{tabular}
\end{center}


\resetBoolean
\checkFileNoReset{conditionsFirstFilter.tex}
\checkFileNoReset{conditionsSecondFilter.tex}
\ifthenelse{\boolean{isFile1}}{
	\subsection{Grouping of selected experiment factors}
	
	\resetBoolean
	\checkFileNoReset{conditionsFirstFilter.tex}
	\ifthenelse{\boolean{isFile1}} {
		\subsubsection{Primary filter}
		\begin{center}
		\loadTex{conditionsFirstFilter}
		\end{center}
	}{}
	
	\resetBoolean
	\checkFileNoReset{conditionsSecondFilter.tex}
	\ifthenelse{\boolean{isFile1}} {
		\subsubsection{Secondary filter}
		\begin{center}
		\loadTex{conditionsSecondFilter}
		\end{center}
	}{}
}{}

\resetBoolean
\checkFileNoReset{unitTable.tex}
\ifthenelse{\boolean{isFile1}}{
	\subsection{Units of the used descriptors}
	
	\begin{center}
	\loadTex{unitTable}
	\end{center}
}{}


\subsection{Full list of experiment factors}
\begin{center}
	\begin{longtable}{|p{0.7cm}|p{2.2cm}|p{2.2cm}|p{2.2cm}|p{2.2cm}|p{2.2cm}|}
	%|p{2.2cm}
	
	%This is the header for the first page of the table...
	
	\hline
	{\textbf{ID}} & 
	{\textbf{Species}} & 
	{\textbf{Genotype}} & 
	{\textbf{Variety}} &
	{\textbf{Treatment}} & 
% 	{\textbf{Sequence}} & 
	{\textbf{Growth conditions}}
	\tabularnewline
	\hline
	\hline
	\endfirsthead
	
	 %This is the header for the remaining page(s) of the table...
	 
	\hline
	{\textbf{ID}} & 
	{\textbf{Species}} & 
	{\textbf{Genotype}} & 
	{\textbf{Variety}} &
	{\textbf{Treatment}} & 
% 	{\textbf{Sequence}} & 
	{\textbf{Growth conditions}}
	\tabularnewline
	\hline
	\hline
	\endhead
	
	%This is the footer for all pages except the last page of the table...
	
	\multicolumn{6}{l}{{Continued on Next Page\ldots}} 
	\tabularnewline
	\endfoot
	
	%This is the footer for the last page of the table...
	
	\hline \hline
	\endlastfoot

	--factorlist--

	\end{longtable}
\end{center}
